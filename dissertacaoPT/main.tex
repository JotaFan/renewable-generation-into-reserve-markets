%Documento elaborado através da modificação de Template_ Tese_ MAEG por RJF. 
%Com contribuições e modificações dos alunos do DEGGE e Phd student 
%Angelo Soares (arsoares@fc.ul.pt) com orientação de Carla M. Silva.
%11/11/2021
% Para escrever neste documento, expandam a secção Capitulos e escrevam no capitulo apropriado.
% Após a escrita, seleccionem "main.tex" e recompilem. Tudo o que escreverem nos capitulos irá 
% aparecer. 

% ------------------------------------------------------------------------------------------------------%
% ------------------------------------------ ALTERAR CAPA ----------------------------------------------%
% ------------------------------------------------------------------------------------------------------%
%A Neste main.tex as únicas alterações que precisam de fazer são referentes à Capa da
%dissertação. Sigam as indicações seguintes: Façam ctrl+F para procurar "includepdf" e leiam 
%as notas por baixo do \includepdf.

% ------------------------------------------------------------------------------------------------------%

%Outras Notas: 
%1. Ler o conteudo em "library.bib".
%2. O GOOGLE é vosso amigo, todas e quaisquer modificações necessárias, seja para introduzir um
%certo tipo de formatação, seja para colocar uma figura ou tabela com especificações particulares,
%usem o GOOGLE, existem N exemplos para todo o tipo de alteração em latex. Com quase copy+paste
%code.
% ------------------------------------------ IMPORTANTE -----------------------------------------------%
%3. Neste documento existem algumas colisoes de pacotes, nomeadamente o tocloft e o book report
%e a maneira %como o chapter foi elaborado. Exemplo: Se os alunos quiserem colocar o prefixo "1."
%antes da introdução, ou seja "1. Introdução". No Indice ficaria "1 1. Introdução".
%ALTERNATIVA: Escrevam a dissertação na totalidade. No fim, façam download do PDF (cliquem segundo
%Icone após o Recompile). Com esse ficheiro PDF, abram-no no PDF reader (adobe acrobat pro) e 
%editem acrescentando o número do capitulo para ficar igual ao que está no Indice "1. Introdução".
%É a forma mais simples de resolver.

%Se alguem por curiosidade encontrar outra solução usando o tocloft package, como por exemplo, 
%ocultar o número do capitulo no TOC sem usar \chapter* ou interferir com o hyperref enviem essa
%info para o email supramencionado, muito obrigado.
%Boa sorte


\documentclass[11pt,a4paper,twoside,openright]{book}
% ------------------------------------------------------------------------------------------------------%
% ------------------------------------------- PREAMBLE -------------------------------------------------%
% ------------------------------------------------------------------------------------------------------%
\usepackage{mathptmx} % Fonte mais próxima de Times New Roman
% Alterar margens do documento
\usepackage[margin=2.5cm]{geometry}
\renewcommand{\baselinestretch}{1.15}
\usepackage{setspace}
\usepackage{fancyhdr}
\pagestyle{fancy}
\renewcommand{\chaptermark}[1]{\markboth{\MakeUppercase{\thechapter. #1 }}{}}
\renewcommand{\sectionmark}[1]{\markright{\thesection\ #1}}
\fancyhf{}
\fancyhead[RO]{\small{\rightmark}}
\fancyhead[LE]{\small{\leftmark}}
\setlength{\headheight}{12pt}
\addtolength{\topmargin}{-10.5pt}
\fancyfoot[C]{\thepage}
\renewcommand{\headrulewidth}{0pt}
\renewcommand{\footrulewidth}{0pt}
\addtolength{\headheight}{0.5pt}
\fancypagestyle{plain}{
  \fancyhead{}
  \renewcommand{\headrulewidth}{0pt}
}

\usepackage[backend=biber, natbib=true, style=numeric, sorting=none]{biblatex} %Numeric style
%\usepackage[backend=biber,style=authoryear,maxcitenames=1,autocite=inline,natbib=true]{biblatex}
%\addbibresource{library} %O ficheiro .bib que a bibliografia vai usar
%\bibliography{library}
%\addbibresource{library.bib} %O ficheiro .bib que a bibliografia vai usar
%\bibliography{library.bib}
\addbibresource{library.bib}

%\bibliographystyle XXX
%\usepackage[numbers]{natbib}

% Package que permite incluir imagens
\usepackage{graphicx}

% Comentar para não incluir a lista de figuras e tabelas no índice
\usepackage[nottoc]{tocbibind}

% Para citar referências
%\usepackage{natbib}
%\setcitestyle{numbers}
%\bibliographystyle{humannat}
%\setcitestyle{authoryear}

%Indentar primeiro parágrafo
\usepackage{indentfirst}

% Incluir até à subsecção no índice
\setcounter{tocdepth}{3}
\setcounter{secnumdepth}{3}

% Permite remover os espaços entre itens
\usepackage{enumitem}

% Caption para imagens
\usepackage[font=footnotesize]{caption}

% Referenciar anexos
\usepackage[toc,page]{appendix}

%Adicionar pdfs
\usepackage{pdfpages}

%Alterar nome de capitulo para a referencia
\usepackage{fmtcount}
\usepackage{url}

%Para expressões Matemáticas
\usepackage{amsmath}
\usepackage{bm}

%Bibliotecas
%\usepackage{biblatex}
%\usepackage[portuguese]{babel} %Can be switched to english
\usepackage{csvsimple}
\usepackage[utf8]{inputenc}
\usepackage{float}
\raggedbottom
\usepackage{subcaption}
\usepackage{eurosym}
\usepackage{booktabs}
\usepackage[T1]{fontenc}
\usepackage{csquotes}
\usepackage{tabularx}
\usepackage[nottoc]{tocbibind}
%\usepackage[scaled]{helvet}
%\renewcommand*\familydefault{\sfdefault}

%Comando que cria abreviaturas usado no Nomenclatura
\newcommand{\abbreviations}[1]{%
\vspace{12pt}\noindent{\selectfont\textbf{Abreviaturas}\par\vspace{6pt}\noindent {\fontsize{9}{9}\selectfont #1}\par}}
%Criar Siglas
\newcommand{\siglas}[1]{%
\vspace{12pt}\noindent{\selectfont\textbf{Siglas e acrónimos}\par\vspace{6pt}\noindent {\fontsize{9}{9}\selectfont #1}\par}}
%Cria simbologia
\newcommand{\simbolos}[1]{%
\vspace{12pt}\noindent{\selectfont\textbf{Simbologia}\par\vspace{6pt}\noindent {\fontsize{9}{9}\selectfont #1}\par}}

\makeatletter
\def\cleardoublepage{\clearpage\if@twoside \ifodd\c@page\else
    \hbox{}
    \thispagestyle{plain}
    \newpage
    \if@twocolumn\hbox{}\newpage\fi\fi\fi}
\makeatother \clearpage{\pagestyle{plain}\cleardoublepage}

\providecommand{\keywords}[1]{\textbf{Keywords:} #1}
\providecommand{\keywordsP}[1]{\textbf{Palavras chave:} #1}

%Remover Chapter N on the header
\usepackage{titlesec}
\usepackage{lipsum}
\titleformat{\chapter}[display]
   {\normalfont\huge\bfseries}{\chaptertitlename\ \thechapter}{20pt}{\Huge}   
\titlespacing*{\chapter}{0pt}{-20pt}{40pt}
\titleformat{\chapter}[display]{\normalfont\bfseries}{}{0pt}{\huge}

%\numberwithin{section}{chapter}
\usepackage{multirow}
\usepackage{chngcntr}  
\usepackage{tocloft}
\usepackage[hidelinks]{hyperref}
\hypersetup{
    colorlinks=true,
    linkcolor=black,  % color of internal links
    filecolor=black,  % color of links to local 
      urlcolor=blue,
}
%\counterwithout{figure}{chapter}
%\counterwithout{table}{chapter}

\renewcommand{\cftfigpresnum}{\textbf{Figura\ }}
\renewcommand{\cfttabpresnum}{\textbf{Tabela\ }}

\newlength{\mylenf}
\settowidth{\mylenf}{\cftfigpresnum}
\setlength{\cftfignumwidth}{\dimexpr\mylenf+1.5em}
\setlength{\cfttabnumwidth}{\dimexpr\mylenf+1.5em}

\sloppy

\makeatletter
\newcommand\listoftablesandfigures{%
    \chapter*{List of Tables and Figures}%
    \phantomsection
\@starttoc{lof}%
\bigskip
\@starttoc{lot}}
\makeatother

\PassOptionsToPackage{hyphens}{url}\usepackage{hyperref}

% ------------------------------------------------------------------------------------------------------%
% ------------------------------------------- END - PREAMBLE -------------------------------------------%
% ------------------------------------------------------------------------------------------------------%

\begin{document}


% Capa
\newpage
\thispagestyle{empty}

%\setmainfont[Path = fonts/]{arial_narrow.ttf} 

\centering{\fontsize{12.2}{14.4}\selectfont UNIVERSIDADE DE LISBOA}\\
\centering{\fontsize{12.2}{14.4}\selectfont FACULDADE DE CIÊNCIAS}\\
\centering{\fontsize{12.2}{14.4}\selectfont DEPARTAMENTO DE ENGENHARIA GEOGRÁFICA,GEOFÍSICA E ENERGIA}

\vspace{1.3cm}

\begin{figure}[h]
 \centering
 \includegraphics[width = 6.99cm,height = 3.3cm]{Imagens/logo_fcul.png}
\end{figure}

\vspace{3.45cm}

%\setmainfont[Path = fonts/]{arial_narrow_bold.ttf}

\centering{\fontsize{17}{20.4}\selectfont \textbf{Participação da geração renovável no mercado de reservas de um sistema eléctrico}}

%\setmainfont[Path = fonts/]{arial_narrow.ttf}

\vspace{4cm}

\centering{\fontsize{14}{16.8}\selectfont João Pedro Passagem dos Santos}

\vspace{1.3cm}

%\setmainfont[Path = fonts/]{arial_narrow_bold.ttf}

\centering{\fontsize{13}{15.6}\selectfont \textbf{Mestrado em Engenharia da Energia e Ambiente}}\\

%\setmainfont[Path = fonts/]{arial_narrow.ttf}

%\centering{\fontsize{13}{15.6}\selectfont Especialização em Bioinformática}

\vspace{0.5cm}


\vspace{0.5cm}

\centering{\fontsize{13}{15.6}\selectfont Dissertação orientada por:}\\
\centering{\fontsize{13}{15.6}\selectfont Doutor Hugo Algarvio}\\
\centering{\fontsize{13}{15.6}\selectfont Professora Doutora Ana Estanqueiro}\\

\vspace{2.9cm}

\centering{\fontsize{14}{16.8}\selectfont 2024}

\label{Capa}



\let\cleardoublepage\clearpage


\clearpage \thispagestyle{empty}\mbox{}\clearpage

\frontmatter
% RESUMO
\newpage
\thispagestyle{plain}
\chapter{Resumo}
\justifying

A crescente penetração de fontes de energia renováveis variáveis no tempo, \gls{vRES}, no sistema eléctrico, como a solar e a eólica, está a transformar significativamente os mercados de eletricidade, devido à sua natureza intermitente e imprevisível. Esta imprevisibilidade torna as previsões de produção e consumo de energia mais desafiantes, especialmente porque os mercados de eletricidade fecham entre 1 e 37 horas antes da entrega real da energia, o que pode resultar em discrepâncias entre a energia contratada e a energia realmente produzida ou consumida. Manter o equilíbrio entre a oferta e a procura em tempo real é vital para a segurança e estabilidade da rede, função que recai principalmente sobre os operadores de redes de transporte (\gls{TSO}).\par
Os \gls{TSO} utilizam mercados de reserva de energia, onde adquirem de forma simétrica potência secundária ascendente e descendente, com base em previsões de procura para as horas subsequentes. No entanto, essa abordagem simétrica pode ser ineficaz diante das flutuações das energias renováveis, levando à necessidade de ajustes mais dinâmicos e precisos.\par
Neste contexto, o presente trabalho propõe primeiramente um estudo do parâmetro variável por hora da fórmula do \gls{TSO} português para a previsão de reserva necessária ($\rho$), onde usando os dados históricos horários no período 2010 a 2019, é calculado o $\rho$ horário que resulte num menor erro médio na previsão, onde são obtidos resultados com erro inferior a 5\%.\par
Este trabalho propõe de seguida um modelo baseado em técnicas de \textit{machine learning} para calcular dinamicamente as reservas de potência secundária, tanto ascendente como descendente, integrando despachos programados para o dia seguinte, previsões de produção de \gls{vRES}, a procura prevista e outras características operacionais.\par
Utilizando dados operacionais abertos do \gls{TSO} espanhol, o modelo foi treinado com dados no período de 2014 a 2023,  e validado com dados de referência de 2019 a 2022. A metodologia proposta demonstra uma melhoria significativa na utilização das reservas de potência secundária, com um aumento de aproximadamente 43\% na eficiência das reservas ascendentes e cerca de 36\% nas reservas descendentes. Este avanço contribui para uma gestão mais eficiente e equilibrada do sistema elétrico, especialmente em cenários com elevada penetração de \gls{vRES}.\par




\vspace{0.5cm} %adiciona um espaço de 0.5cm entre o texto e as palavras chave.

\keywordsP{sistemas de reserva, mercados de energia, redes neuronais, previsões}
%reparem que # necessita de um \ para que o latex o interprete correctamente como um caracter especial. Isto também acontece com % por exemplo.
\label{resumo}

%Abstract
\newpage
\thispagestyle{plain}
\chapter{Abstract}
Abstract should have the relevant information of the developed work. One sentence with the framework of the research. One sentence with the goal and method. One sentence with the main conclusions.

\vspace{0.5cm} %adds a 0.5cm space between the text and the keywords.

\keywords{up to six different from the title and helpful to find related subjects (like \# in social media)} 
%notice that # requires a \ so that latex correctly interprets it as a special character. This also happens with % for example.
\label{abstract}

% AGRADECIMENTOS
\newpage
\thispagestyle{plain}
\chapter{Agradecimentos}
Opcional, embora no caso de dissertações que decorram no âmbito de projetos financiados, por exemplo, pela FCT ou programas europeus devem ser mencionados aqui a referência e nome do projeto, e mais alguma informação conforme as regras de publicitação do projeto em questão.

\vspace{10mm} %ele aceita diversos tipos units, neste caso 10mm, 0.1cm.

\flushleft{Nome do Autor}
\label{agradecimentos}

%Nomenclatura
\newpage
\thispagestyle{plain}
\chapter{Nomenclatura}
% Lista de siglas, acrónimos, abreviaturas e simbologia apresentadas por ordem alfabética.

% %Utilizar sempre \\ para forçar a mudança de linha.
% %Utilizar o & para forçar a mudança de "coluna", como numa tabela
% \abbreviations{\noindent 
% \begin{tabular}{@{}ll}
% (A/F) & Relação mássica ar/combustível\\
% pme	& Pressão média efectiva\\
% vol & Volume\\
% \end{tabular}
% }
\vspace{12pt}
\printglossary[type=\acronymtype, title={Siglas e acrónimos}, style=customstyle]


\simbolos{\noindent 
\begin{tabular}{@{}ll}
Hz & Hertz\\
MW & Megawatts\\
MWh & Megawatt-hora\\
\end{tabular}
}
%The use of \( \) avoids obscure compiling errors in latex when using certain mathematical notation symbols in a non-mathematical environment (text).
%https://tex.stackexchange.com/questions/510/are-and-preferable-to-dollar-signs-for-math-mode

%Nota: List of greek letters and others https://pt.overleaf.com/learn/latex/List_of_Greek_letters_and_math_symbols
\label{ch:nomenclatura}

% ÍNDICE
% ÍNDICE (Não é preciso fazer nada, faz update automaticamente)
\clearpage
\thispagestyle{plain}
\renewcommand{\contentsname}{Índice}
\tableofcontents
\clearpage
\thispagestyle{plain}
\listoffigures
\clearpage
\thispagestyle{plain}
\listoftables
\clearpage \thispagestyle{plain}\mbox{}\clearpage

\mainmatter
\setlength{\parindent}{15pt} %Comentar retira a indentação inicial nos chapters
\pagenumbering{arabic} %numeros em numeração árabe

% INTRODUÇÃO
\newpage
\thispagestyle{plain}
\chapter{Introdução}

\section{Enquadramento\label{se:enquadramento}}
Esta dissertação enquadra-se no âmbito do projeto \href{https://traderes.eu/}{TradeRES}, que visa o estudo de um sistema de mercado eléctrico capaz de atender às necessidades da sociedade num sistema quase totalmente renovável, tendo as características para se integrar nos \href{https://ods.pt/ods/}{ \gls{ODS}} \ref{fig:ODS}.\par
O estudo da acessibilidade das energias renováveis ao mercado vigente integra-se nos \gls{ODS} n$^{\circ}$7, “Energia Renováveis e Acessíveis”, indo directamente de encontro a um dos pontos deste objectivo: 7.2.1 “Peso das energias renováveis no consumo total final de energia”. Por meio deste objectivo, a participação das renováveis no mercado faz também cumprir, embora indiretamente, o objectivo n$^{\circ}$8 “Trabalho Digno e Crescimento Económico”, através do ponto 8.4, onde, neste último, é dada primazia à eficiência dos recursos globais no consumo e na produção. Esta contribuição indireta ocorre através da diminuição do uso de energias não limpas, justificadas por um maior uso das renováveis, melhorando a gestão de recursos, e baixando o consumo de recursos naturais não renováveis.\par
Por último, no âmbito do presente estudo, podemos igualmente incluir o objectivo n$^{\circ}$13, “Acção Climática”, no qual, referimos, não só a diminuição de consumo de recursos finitos, mas ainda, a melhor gestão de recursos renováveis, promovendo o planeamento e estratégias de combate a emissões de gases de efeito estufa.\par

\begin{figure}[h]
    \centering
    \includegraphics{Imagens/DesenvolvimentoSustentavel.jpg}
    \caption{Objectivos de Desenvolvimento Sustentável da ONU}
    \label{fig:ODS}
\end{figure}

\section{Objetivos e Perguntas de Pesquisa\label{se:objetivos}}

Foram aprovadas a nível europeu (2020)\cite{52020DC0299} sugestões de alterações aos serviços de sistema, que serão seguidas pelos Estados-Membros. Nesta dissertação, será realizada a aplicação dessas sugestões, identificando as melhorias em relação ao \textit{design} actual e avaliando se as novas sugestões serão suficientes para garantir a operação de um sistema elétrico \textasciitilde100\% renovável, potencialmente identificando ações adicionais para garantir a robustez e segurança do sistema elétrico sem o uso de combustíveis fósseis.\par
A penetração das \gls{vRES} no sistema de energia eléctrica trouxe maior incerteza na previsão em mercados de energia, pois estas estão mais sujeitos a elementos não controláveis como a velocidade do vento ou a radiação solar incidente.\par
As seguintes perguntas servirão de guia nesta pesquisa:\par

\begin{enumerate}[label=\alph*)]
  \item Podemos reduzir a incerteza na produção criada pela participação das \gls{vRES} nos sistemas de energia? 
  \item A alocação dinâmica pode ter um efeito positivo no mercado de reservas?
  \item É possível prever a necessidade de reserva necessária baixando a alocação desperdiçada?
\end{enumerate}

Para responder às perguntas \textit{supra} referidas, utilizaremos dados de previsão de geração de energia renovável para estimar a energia necessária para alocação secundária. Actualmente, os valores de previsão desse mercado estão distantes do consumo real, o que resulta em alocações no dia anterior que não estão em conformidade com as necessidades reais.\par
O objetivo deste trabalho é criar métodos de previsão para o dia seguinte, da necessidade de alocação de banda de reserva secundária, de modo a alocar banda suficiente e, simultaneamente, baixar a alocação em excesso, usando dados históricos das mesmas.\par
Iremos explorar a optimização da fórmula de alocação de banda de reserva da REN, testando novos valores para o parâmetro horário da mesma.\par
Utilizando técnicas de \textit{machine learning} vamos criar um modelo para a previsão de alocação necessária do dia seguinte.\par
Previsões mais exactas tornam possível uma melhor gestão das alocações, resultando num menor gasto de recursos energéticos e financeiros.\par

\section{Organização do Documento \label{se:organização}}

Este documento está dividido em capítulos. Sendo que os primeiros apresentam uma introdução às ideias e temas no 1, o estado de arte dos temas na literatura publicada, seguido de uma contextualização do tema do trabalho proposto no capítulo 2. Dentro da contextualização é de forma geral apresentado os mercados de energia, os sistemas de reserva, e os métodos de previsão para os mesmos, dentro destes o uso de fórmulas e o uso de \textit{machine learning}, formulando aqui a motivação e caminho de estudo.\par
No capítulo 3 apresentamos no sub-capítulo Ferramentas as bibliotecas criadas em \textit{python} para o presente estudo.
Segue o sub-capítulo Métodos onde abordamos os diferentes estudos presentes, como serão dirigidos e condições a alcançar. Dividindo o trabalho em estudos distintos para o tipo de previsões apresentadas no capítulo 2.
Métricas e Dados intitula o capítulo 4 que começa numa dissecção das métricas aplicadas ao longo das experiências e como estas influenciam a mesma, terminando num estudo geral dos dados utilizados, seus tratamentos e elações iniciais de análise. Apresentado também o que é usado como treino e como validação para as experiências.
No capítulo 5 são apresentados os resultados da experiência completa, incluindo as métricas apresentadas, apresentado gráficos de séries temporais das previsões conseguidas. Dando realce aos melhores modelos e optimizações conseguidos.
Termina com um breve capítulo conclusivo dando um pouco mais de contexto aos resultados, apresentando possíveis caminhos futuros de melhoria dos mesmos e discutindo o impacto de \textit{machine learning} no futuro das energias renováveis e consequentemente nos mercados de reserva.

% Os dois capítulos seguintes apresentam os dois diferentes estudos. No \hyperref[ch:estudo_1]{capítulo 4} é definido e apresentado o resultado do estudo da estimativa do parâmetro $\rho$ da fórmula de estimativa da \gls{REN}.\par
% No \hyperref[ch:estudo_2]{capítulo 5} explora-se o segundo estudo, o dimensionamento dinâmico da alocação necessária. São apresentados os dados utilizados com um estudo preliminar sobre os mesmos, e o tratamento necessário para usar nos modelos.\par
% No \hyperref[ch:ferramentas]{capítulo 6} as ferramentas de programação criadas para realizar a mesma.\par
% Os 3 capítulos seguintes são os descritivos da experiência em si. \hyperref[ch:metricas]{Capítulo 7} são as métricas usadas e criadas para a validação da experiência, \hyperref[ch:metodos]{capítulo 8} é a estrutura e parametrização da mesma, e \hyperref[ch:resultados_discussao]{capítulo 9} apresenta os resultados.\par
% Termina com um \hyperref[ch:conclusao]{capítulo conclusivo} onde são avaliadas as experiências como um todo, e o seu impacto no âmbito dos mercados de reserva.\par
\label{ch:intro}

% Revisão bibliográfica
\newpage
\thispagestyle{plain}
\chapter{Revisão bibliográfica}

A análise de séries temporais é um tema recorrente em pesquisa. Especialmente para previsões.\\
Desde as previsões para mercados de acções\cite{Bhandari2022}, fenomenos meteorologicos\cite{Wang2019}, e especialmente mercados energéticos, onde se quer ter em consideração o impacto das gerações mais voláteis. \\
As energias renováveis, devido à sua natureza, são as produçoes mais voláteis, logo alvo de estudo ideal para estas tecnologias\cite{Lu2015}
, energia eolica\cite{Sun2022}, energia solar\cite{Rajasundrapandiyanleebanon2023}, aplicabilidade dos vários sistemas\cite{Ahmad2020}, procura\cite{Antonopoulos2020}.
Sendo que cada problema já apresenta arquiteturas e soluções diferentes, como a geração de energia fotovoltaica em casas pode ser melhor prevista com LSTM\cite{Costa2022} mas tambem com uso de SVM\cite{Meenal2018}
As várias faces destas tecnologias estão optimamente apresentadas em\cite{Benti2023}



Para o estudo de previsões de séries temporais chega a ser o caso se pesquisar primeiramente com \textit{deep learning}, antes de procurar outras soluções.\\
Em \cite{Elsayed} é visto o impacto dessa decisão, e se realmente compensa emergir em \text{machine learning}. O trabalho conclui que modelos simples, com alguma engenharia de atributos inteligente, consegue competir, ou mesmo passar as qualidades de redes neuronais profundas. \\
Esta conclusão mostra também que por vezes a procura por modelos mais complexos não compensa, e que cada problema/dataset deve ter a sua própria investigação e conclusão, consoante a quantidade/qualidade de recursos disponiveis. \\

Os sistema de reserva de frequência do mercado espanhol já foram alvo de análise predictivas com modelos de redes neuronais profundas \cite{miota2023}.\\
Neste trabalho procurou-se prever o preço da banda de reserva secundaria, sendo que os melhores modelos atigiram metricas de MASE de cerca de 64\%, o que não foi considerado um bom resultado. \\
Para o trabalho presente, isto mostra que a resolução linear destes dados pode ser dificil. O trabalho utilizou 32 variaveis abertas pela TSO espanhola, e tambem conclui que o aumento de numero de variveis não melhorou os modelos. Para este trabalho o impacto deste conhecimento é pouco, visto estudarmos com um conjunto fixo de variaveis, e sendo que queremos perceber se conseguimos prever com esse conjunto.\\
Embora estes dados tragam um estudo parecido, usando dados da mesma fonte, o objectivo é ligeiremente diferente.








\label{ch:revisao}

% Contextos
\newpage
\thispagestyle{plain}
\chapter{Contextos\label{ch:contextos}}

\section{Mercado de Serviços de Sistema \label{se:servicos_sistema}}
%\cite{Lopes2021}
%\cite{Watson1984}
%\citep{Schweppe1988}

% O mercado de serviço de sistema é parte integrante dos mercados de energia e mantêm responsabilidade sobre a segurança do mesmo.\cite{dgegmss}.
% Serve para garantir o equilíbrio entre a energia produzido e a consumida. Esta qualidade e segurança é controlada através da frequência e da potência activa, controlo de tensão e potência reactiva, arranque automático e outras técnicas de sistemas \cite{Rassid2017} \cite{Carneiro2016}. \\
% Neste caso de estudo estamos interessados nos serviços de controlo de frequência. A nível europeu estes serviços são impostos pela \gls{ENTSO-E}, e a operação dos mesmos é da responsabilidade dos \gls{TSO} nacionais.\\
% Para manter o controlo de frequência o gestor de sistema deverá manter reservas para responder às diferenças entre a energia consumida e produzida na rede, que deve ser mantida em equilíbrio. Quando o serviço de sistema precisa de actuar para manter a frequência no seu valor nominal, 50Hz na Europa, isto é feito alterando a potência activa dos geradores.  \\
% Quando é necessário um aumento na potência chama-se a isto Banda de Reserva/Regulação a Subir, e quando é necessária uma diminuição chama-se à mesma a Descer.
% Para isto, no mercado ibérico, a tarefa é dividida em três reservas, primária, secundária e terciária. Esta divisão assenta no tempo de resposta que os sistemas precisam de ter, e na capacidade de actuação (MWh/Hz). \\

% A reserva primária é a primeira a ser activada em resposta a distúrbios na rede, como desvios de frequência, e deve agir em questão de segundos para estabilizar o sistema. A reserva secundária, que funciona como um sistema de segurança para a reserva primária, é regulada pelo mercado de banda das reservas secundárias, e sua alocação ocorre no dia anterior à sua necessidade. A reserva terciária é usada para complementar as reservas anteriores e para lidar com desvios de potência activa de longa duração.
% O valor alocado de reserva secundária tem um custo para as operadoras, como tal a previsão do mesmo é importante para a gestão destes sistemas de segurança. Estas previsões são feitas através de estatísticas dos sistemas, e tendo em conta as áreas de balanço que os mesmos têm.\\
% Assim como no mercado de energia de balanço, onde a participação de \gls{vRES} é cada vez mais incentivada, também nos serviços de reserva há uma crescente necessidade de adaptar esses mercados para acomodar a variabilidade e a imprevisibilidade inerente das \gls{vRES}, como a energia eólica e solar. A falta de harmonização nos mercados de balanço, particularmente na forma como os preços de desequilíbrios são determinados, apresenta um desafio significativo. Embora a participação de produtores de energia renovável nesses mercados seja tecnicamente viável, existem restrições que visam garantir a segurança e a estabilidade da rede, além de questões económicas que precisam ser abordadas para tornar esses mercados mais atrativos para todos os participantes.\\

% Uma previsão precisa das necessidades de reservas secundárias é crucial, pois impacta diretamente os custos e recursos das operadoras.
% Estas previsões são feitas usando fórmulas. Que por si só não prevêem a variabilidade dos sistemas de produção de energia renovável. Esta variabilidade sendo dificilmente previsível, tem sido alvo de estudo com modelos de \textit{machine learning}.\\
% Com o aumento de dados históricos destes mercados começa a ser possível estudar os mesmos para criar métodos empíricos para estas previsões.\\
% Com bons resultados apresentados em estudo de energias renováveis, a aplicação dos mesmos métodos para as reservas de sistema parece um passo natural. \\


\subsection{Introdução ao Mercado de Serviços de Sistema \label{se:intro_servicos_sistema}}

O mercado de serviços de sistema é uma componente fundamental dos mercados de energia, desempenhando um papel crucial na manutenção da segurança e estabilidade das redes elétricas \cite{dgegmss}. Esses serviços são essenciais para garantir que a produção e o consumo de energia permaneçam em equilíbrio, um requisito vital para o funcionamento seguro e eficiente de qualquer sistema eléctrico. A principal função dos serviços de sistema é assegurar a qualidade da energia fornecida, monitorizando parâmetros críticos como a frequência, a potência activa e reactiva, controlando a tensão na rede, arranque automático e outras técnicas de sistemas. Esse controlo é realizado através da coordenação entre os geradores e os consumidores, com o objetivo de responder rapidamente a variações na oferta e na procura de energia \cite{Rassid2017} \cite{Carneiro2016}.\\
No contexto europeu, a regulação desses serviços é coordenada pela \gls{ENTSO-E}, que estabelece os requisitos e normas para a operação dos sistemas de energia, e a operação dos mesmos é da responsabilidade dos \gls{TSO} nacionais. Essas reservas são activadas conforme necessário para manter a frequência da rede no seu valor nominal de 50Hz, ajustando a potência activa dos geradores em resposta a variações imprevistas na demanda ou na oferta de energia.\\
As reservas de frequência são divididas em três categorias principais: primária, \gls{FCR}, secundária, \gls{aFRR} e terciária, \gls{mFRR}. Cada uma com funções específicas e tempos de resposta distintos. A reserva primária é activada automaticamente e de forma quase instantânea, dentro de segundos após um distúrbio na rede, para estabilizar rapidamente a frequência. A reserva secundária entra em ação logo em seguida, substituindo gradualmente a reserva primária e ajustando a frequência de volta ao seu valor programado. Finalmente, a reserva terciária é utilizada para corrigir desvios de longo prazo e libertar as outras reservas para possíveis eventos futuros, completando o ciclo de controlo da frequência e assegurando que o sistema retorne a um estado de equilíbrio estável.\\
A harmonização dos mercados europeus de eletricidade, especialmente nos mercados diários, intradiários e de balanço, é uma realidade em desenvolvimento que busca reduzir custos e melhorar as condições de participação para todos os envolvidos \cite{Algarvio2019}. No entanto, a integração das \gls{vRES}, como a eólica e a solar, apresenta desafios adicionais devido à sua natureza intermitente e dependente de condições climáticas. Embora tecnicamente viável, devido a este paradigma de imprevisibilidade e ao facto de serem fontes não despacháveis, a participação dessas fontes nos mercados de balanço enfrenta restrições significactivas para garantir a segurança e a estabilidade da rede.\\
A atual infraestrutura dos mercados de serviços de sistema precisa, portanto, ser adaptada para acomodar essas novas fontes de energia. Uma parte essencial dessa adaptação é o desenvolvimento de métodos mais robustos para prever a necessidade de reservas, que levem em consideração a variabilidade das \gls{vRES}. Actualmente, as previsões são baseadas principalmente em fórmulas criadas pelas operadoras, mas esta abordagem muitas vezes falha em capturar a complexidade e a incerteza associadas à produção renovável. Assim, há uma crescente exploração de técnicas avançadas, como o uso de modelos de \textit{machine learning}, para melhorar a precisão das previsões e otimizar a gestão das reservas.
Além disso, a evolução para um mercado pan-europeu harmonizado de serviços de sistema envolve não apenas a uniformização de regras e requisitos técnicos, mas também a criação de incentivos económicos que tornem a participação atraente para todos os tipos de produtores de energia, incluindo os renováveis. Isso é particularmente importante, uma vez que os mercados de balanço são fundamentais para garantir que as redes elétricas possam operar de forma estável e segura, mesmo com altas penetrações de \gls{vRES}. Ao permitir que essas fontes renováveis participem de forma mais activa e competitiva nos mercados de balanço, espera-se não apenas reduzir os custos de operação dos sistemas eléctricos, mas também aumentar a viabilidade económica das \gls{vRES}.\\
Com a crescente dependência de fontes de energia renovável e a necessidade de sistemas eléctricos mais resilientes e flexíveis, o papel dos serviços de sistema continuará a expandir-se e a evoluir, exigindo inovações tanto na gestão técnica quanto na regulação econômica dos mercados de energia.\\


\subsection{Estrutura e Funcionamento das Reservas de Frequência \label{se:reservas_freq}}


A reserva primária, \gls{FCR}, é o primeiro nível de resposta e é acionada automaticamente em questão de segundos após a detecção de um desvio de frequência, que pode ocorrer devido a falhas na produção ou variações repentinas na procura. Esta reserva é activada até 15 segundos após o distúrbio e permanece activa por cerca de 30 segundos, ou até que a reserva secundária possa assumir o controlo. A \gls{FCR} é geralmente suportada por geradores que possuem capacidade técnica para resposta rápida, como hidroelétricas e algumas unidades térmicas. Este serviço é obrigatório para todos os geradores conectados à rede que possuem a capacidade técnica necessária, e não é remunerado em muitos mercados europeus, incluindo o mercado ibérico.\\
A reserva secundária \gls{aFRR} entra em ação logo após a activação da reserva primária, com o objetivo de restaurar a frequência da rede ao seu valor programado de 50 Hz e libertar a \gls{FCR} para responder a possíveis distúrbios subsequentes. A \gls{aFRR} é activada automaticamente até 30 segundos após o desvio inicial e pode levar até 15 minutos para corrigir completamente o desequilíbrio. Este tipo de reserva é contratado em mercados específicos de banda de reserva, nos quais os geradores submetem ofertas para fornecer a capacidade necessária.\\
A reserva terciária \gls{mFRR} é o último nível de resposta e é utilizada principalmente para corrigir desequilíbrios de longo prazo e libertar a \gls{aFRR} para outros usos. Ao contrário das reservas primária e secundária, a \gls{mFRR} é activada manualmente pelos \gls{TSO} e pode levar até 15 minutos a estar completamente activa. Esta reserva é frequentemente utilizada para ajustar a geração ou o consumo de energia de acordo com desvios significativos e prolongados, que não podem ser compensados de forma eficaz pelas reservas de resposta mais rápida. A \gls{mFRR} é geralmente suportada por geradores que podem oferecer flexibilidade em suas operações, como algumas centrais térmicas e hidroelétricas de grande porte.\\

\subsection{Previsão de Necessidades de Reservas \label{se:pred_impact_vres}}

A previsão das necessidades de reservas de frequência é uma componente essencial na gestão eficiente dos sistemas eléctricos, especialmente num cenário de crescente penetração das \gls{vRES}.\\
O uso de técnicas de \textit{machine learning} tem sido explorado como uma solução promissora para melhorar essas previsões. Estes modelos podem analisar grandes volumes de dados, identificar padrões complexos e ajustar previsões em tempo real, levando em consideração factores como mudanças nas condições meteorológicas e padrões de consumo de energia. Ao incorporar a variabilidade das \gls{vRES} nos modelos de previsão, é possível reduzir a incerteza e melhorar a alocação das reservas de frequência, resultando numa operação mais eficiente do sistema eléctrico.\\
Outro factor crítico na previsão das necessidades de reservas de frequência é a coordenação entre diferentes mercados e operadores de sistemas. A harmonização dos mercados europeus de balanço, incluindo a padronização das regras de oferta, leilão e remuneração, pode facilitar a integração das \gls{vRES} e melhorar a eficiência geral do sistema. Com regras claras e uniformes, os produtores de energia renovável têm maior incentivo para participar activamente dos mercados de reservas, fornecendo capacidade adicional para apoiar a estabilidade da rede. Isso é particularmente relevante em mercados onde as \gls{vRES} ainda enfrentam barreiras significactivas para a participação, como regras complexas de licitação ou altos requisitos de capacidade mínima para participação.\\
Apesar dos avanços na previsão de necessidades de reservas, ainda existem desafios consideráveis. A precisão das previsões pode ser limitada pela qualidade dos dados disponíveis, bem como pela capacidade dos modelos de capturar todas as variáveis relevantes que afetam a operação da rede. Além disso, a crescente interconexão dos sistemas eléctricos e o aumento da troca de energia entre países exigem uma abordagem coordenada e colaboractiva para a previsão de reservas, considerando tanto as condições locais quanto regionais.\\
O desenvolvimento contínuo de técnicas avançadas de previsão e a integração de soluções baseadas em dados serão fundamentais para enfrentar esses desafios. À medida que mais dados históricos se tornam disponíveis e os modelos de previsão evoluem, espera-se que a gestão das reservas de frequência se torne cada vez mais eficiente, contribuindo para um sistema eléctrico mais resiliente e capaz de integrar altos níveis de energias renováveis. Isso não apenas reduzirá os custos operacionais, mas também contribuirá para a segurança energética e para a transição para um sistema energético mais sustentável.\\


%\section{\gls{MIBEL} \label{se:mibel}}
%\cite{Bessa2012}
%\cite{Fernandes2016}
%\citep{Agostini2021}


\thispagestyle{plain}
\section{Arquitecturas de Modelos\label{se:arquiteturas_modelos}}

Grande parte da literatura sobre previsões em modelos de apredizagem apresenta as mesmas arquiteturas, sendo que são depois aprimoradas consoate os dados e o problema. \\
Apresento aqui as arquitecturas mais usadas em previsões, como também algumas usadas noutros ramos tentado prever a compatibilidade neste problema. \\
% TODO: meter citaçao para o uso de cada uma em previsões
Neste trabalho vamos usar arquiteturas de \gls{FCNN}, \gls{CNN}, \gls{LSTM} e Transformer.\\




\subsection{FCNN\label{se:fcnn_sec}}

A arquitetura mais simples \gls{FCNN}, Redes Neuronais Totalmente Conectadas , é constituída por camadas em que cada neurónio está ligado a todos os neurónios da camada seguinte. Isto significa que cada caraterística de entrada tem um peso associado, e esses pesos são aprendidos durante o treino. A saída de cada neurónio é calculada através da aplicação de uma função de ativação à soma ponderada das suas entradas.\\
Cada neurónio gera uma operação, inicialmente aleatória, para tentar reproduzir uma função que traduza a entrada na saída ideal.\\
Esta arquitectura tem como base o Perceptão inicialmente proposto em \cite{Rosenblatt1958}. Este apresentava um Perceptão que fazia uma decisão binária baseado nas somas pesadas de todas as entradas.\\
A ideia é a base utilizada actualmente, mas apresentava algumas limitações, e muita computação, o proposto por \cite{Minsky1969}, eleva a ideia com a introdução da função de activação e o bias. A utilização mais recorrente actual é a proposta em \cite{Haykin1999}.


\begin{figure}[H]
	\centering
	\resizebox{\linewidth}{!}{\begin{tikzpicture}[scale=2.4, transform shape]
    % Draw input nodes
    \foreach \h [count=\hi ] in {$x_2$,$x_1$}{%
          \node[input] (f\hi) at (0,\hi*1.25cm-1.5 cm) {\h};
        }
    % Dot dot dot ... x_n
    \node[below=0.62cm] (idots) at (f1) {\vdots};
    \node[input, below=0.62cm] (last_input) at (idots) {$x_n$};
    % Draw summation node
    \node[functions] (sum) at (4,0) {\Huge$\sum$};
    \node[below=0.69cm] at (sum) {$\sum_{i=0}^n w_ix_i$};
    % Draw edges from input nodes to summation node
    \foreach \h [count=\hi ] in {$w_2$,$w_1$}{%
          \path (f\hi) -- node[weights] (w\hi) {\h} (sum);
          \draw[->] (f\hi) -- (w\hi);
          \draw[->] (w\hi) -- (sum);
        }
    % Dot dot dot ... w_n
    \node[below=0.05cm] (wdots) at (w1) {\vdots};
    \node[weights, below=0.45cm] (last_weight) at (wdots) {$w_n$};
    % Add edges for last node and last weight etc
    \path[draw,->] (last_input) -- (last_weight);
    \path[draw,->] (last_weight) -- (sum);
    % Draw node for activation function
    \node[functions] (activation) at (7,0) {};
    \node[small_input, below=1cm] (bias) at (activation) {bias};
    \path[draw,->] (bias) -- (activation);

    % Place activation function in its node
    \begin{scope}[xshift=7cm,scale=1.25]
        \addaxes
        % flexible selection of activation function
        \relu
        % \stepfunc
    \end{scope}
    % Connect sum to relu
    \draw[->] (sum) -- (activation);
    \draw[->] (activation) -- ++(1,0);
    % Labels
    \node[above=1cm]  at (f2) {Entradas};
    \node[above=1cm] at (w2) {Pesos};
    \node[above=1cm] at (activation) {Função de activação};

    % Neuron
    \node[draw, dashed, fit= (w2) (last_weight) (activation) (bias), inner sep=0.5em] (square){};
    \node[below=1.5cm] at (square) {Neurónio};

\end{tikzpicture}}
	\caption{Ilustração de um neurónio. Adaptado de \cite{Haykin1999}}
	\label{fig:neuronio}
\end{figure}



\subsection{CNN\label{se:cnn_sec}}
% TODO: add cites do conv 
As Redes Neuronais Convolucionais (\gls{CNN}) diferem das \gls{FCNN} no sentido em que os filtros (neurónios) não são criados aleatoriamente, mas sim cada filtro trata de uma parte da camada de entrada. Nas convoluções é criada uma janela móvel que percorre a camada, criando um saída desse conjunto de pontos. Esta janela move-se sempre subsequentemente.\\
Esta operação é normalmente feita na dimensão (ou dimensões) em que queremos perceber padrões.\
Nos nossos dados a convolução será na dimensão temporal.\\
Se tivermos uma matriz com nove passos temporais (N,9,1), se o tamanho da janela de convolução for 3, teremos uma saída de tamanho 6 (N, 6, 1).\\
\begin{figure}[H]
	\centering
	\resizebox{\linewidth}{!}{
\begin{tikzpicture}
    % Time series
    \matrix (M1) [matrix of math nodes, nodes={draw, minimum size=1cm, anchor=center}, 
    column sep=-\pgflinewidth, row sep=-\pgflinewidth,
    ] {
        \node[fill=red, draw=red] (M1-1-1) {1}; & \node[fill=red, draw=red] (M1-2-1) {2};
        & \node[fill=red, draw=red] (M1-3-1) {3}; &
        4 & 5 & \node[draw=yellow] (M1-6-1) {6}; & \node[draw=yellow] (M1-7-1) {7};
        & \node[draw=yellow] (M1-8-1) {8}; & 9 \\
    };
    
    % Kernel
    \matrix (M2) [below=of M1, matrix of math nodes, nodes={draw, minimum size=1cm, anchor=center}, 
    column sep=-\pgflinewidth, row sep=-\pgflinewidth,
    ] {
        \node[fill=red, draw=red] (M2-1-1) {6}; & 9 & 12 & 15 & 18 & \node[draw=yellow] (M2-5-1) {21}; & 24 \\
    };

    \draw[dashed, red] (M1-1-1.north west) -- (M2-1-1.north west);
    \draw[dashed, red] (M1-3-1.north east) -- (M2-1-1.north east);
    \draw[dashed, red] (M1-1-1.south west) -- (M2-1-1.south west);
    \draw[dashed, red] (M1-3-1.south east) -- (M2-1-1.south east);

    \draw[dashed, yellow] (M1-6-1.north west) -- (M2-5-1.north west);
    \draw[dashed, yellow] (M1-8-1.north east) -- (M2-5-1.north east);
    \draw[dashed, yellow] (M1-6-1.south west) -- (M2-5-1.south west);
    \draw[dashed, yellow] (M1-8-1.south east) -- (M2-5-1.south east);

    
    % Titles
    \node [right=1cm, align=center, font=\bfseries] at (M1.east) {Série Temporal};
    \node [right=1cm, align=center, font=\bfseries] at (M2.east) {Filtro};
    \end{tikzpicture}}
	\caption{Ilustração da operação de Convolução}
	\label{fig:conv_layer1D}
\end{figure}

Anteriormente ignoramos o número de filtros. Mas as convoluções criam o número pedido de filtros para cada janela temporal. Aqui cada filtro vai funcionar como na camada \gls{FCNN}, onde cada um começa com uma operação pseudo aleatória. Esta operação normalmente é feita na dimensão dos atributos.\\
Ou seja, a quantidade de filtros que esta camada irá produzir por convolução.\\
Se tivermos a mesma entrada que anteriormente mas com 4 atributos (N, 9, 4), e se definir o número de filtros para 2 teremos uma saída (N, 6, 2).\\
Ou seja, dois filtros por cada janela temporal.\\


\begin{figure}[H]
	\centering
	\resizebox{\linewidth}{!}{\begin{tikzpicture}[scale=2]
    \matrix (M1) [matrix of math nodes, nodes={draw, minimum size=1cm, anchor=center}, 
    column sep=-\pgflinewidth, row sep=-\pgflinewidth,]
{
    \node[draw=red] (M1-1-1) {1}; & \node[draw=red] (M1-1-2) {2}; & \node[draw=red] (M1-1-3) {3}; & 4 & 5 & 6 & 7 & 8 & 9\\
    \node[draw=red] (M1-2-1) {4}; & \node[draw=red] (M1-2-2) {5}; & \node[draw=red] (M1-2-3) {6}; & 7 & 8 & 9 & 10 & 11 & 12\\
    \node[draw=red] (M1-3-1) {7}; & \node[draw=red] (M1-3-2) {8}; & \node[draw=red] (M1-3-3) {9}; & 10 & 11 & 12 & 13 & 14 & 15\\
\node[draw=red] (M1-4-1) {10}; & \node[draw=red] (M1-4-2) {11};
& \node[draw=red] (M1-4-3) {12}; & 13 & 14 & 15 & 16 & 17 & 18\\
};

\matrix (M2) [matrix of math nodes, nodes={draw, minimum size=1cm, anchor=center}, 
column sep=-\pgflinewidth, row sep=0.3cm,
    right=of M1]
{
    \node[fill=red, draw=red] (M2-1-1) {78}; & 90 & 102 & 114 & 126 & \node[draw=yellow] (M2-1-5) {138}; & 150 \\
    \node[fill=red, draw=red] (M2-2-1) {6}; & 9 & 12 & 15 & 18 & \node[draw=yellow] (M2-2-5) {21}; & 24 \\
    };

% Titles
\node [above=0.5cm, align=center, font=\bfseries] at (M1.north) {Série Temporal};
\node [above=0.5cm, align=center, font=\bfseries] at (M2.north) {Filtros};
 
\draw[dashed, red] (M1-1-1.north west) -- (M2-1-1.north west);
\draw[dashed, red] (M1-1-1.north west) -- (M2-2-1.north west);

\draw[dashed, red] (M1-1-3.north east) -- (M2-1-1.north east);
\draw[dashed, red] (M1-1-3.north east) -- (M2-2-1.north east);

\draw[dashed, red] (M1-4-1.south west) -- (M2-1-1.south west);
\draw[dashed, red] (M1-4-1.south west) -- (M2-2-1.south west);


\draw[dashed, red] (M1-4-3.south east) -- (M2-1-1.south east);
\draw[dashed, red] (M1-4-3.south east) -- (M2-2-1.south east);



% TODO: add legenda em pequeno
%\node [right=1cm, align=center, font=\bfseries] at (matrix1.west) {Atributos};
%\node [right=1cm, align=center, font=\bfseries] at (matrix1.south) {tempo};


\end{tikzpicture}}
	\caption{Ilustração da camada de Convolução}
	\label{fig:conv_layer}
\end{figure}

As convoluções podem realizar as operações em mais dimensões, é comum usar 2D para imagens, e 3D para vídeos. Neste trabalho apenas trabalhamos com convoluções 1D.\\

\subsubsection{UNET\label{se:unet_sec}}

Um desenho especial de \gls{CNN}, normalmente usando em modelação de imagens, e primeiro proposto em \cite{Shelhamer2014}, a arquitectura UNET passa por criar uma rede de expansão dos filtros, usando convoluções, e de seguida uma rede de contracção dos mesmo, até aos tamanhos pretendidos.\\
Nas suas ligações UNET junta informação de filtros passados (não de nível temporal mas de rede neuronal) para realçar informação já trabalhada, e assim identificar padrões de vários contextos diferentes.\\
É chamada assim pois é uma rede (NET) que forma um U na sua expansão, contracção e ligações entre estes.\\
Em cada camada de encoding vai usando convulucões para criar novos filtros e diminuir a dimensionalidade, enquanto que na fase de decoding vai usar convoluções para aumentar a dimensionalidade e diminuir o número de filtros, adicionando a camada decoder de tamanho análogo.\\

\begin{figure}[H]
	\centering
	\resizebox{\linewidth}{!}{\begin{tikzpicture}[ node distance = 2cm, auto, block/.style={ rectangle, draw, align=center, minimum width=2cm, minimum height=1cm }, line/.style={ draw, -latex' } ]
    % Encoder (Contracting Path)
    \node [block] (input) {Input};
    \node [block, below right=of input, xshift=-2cm] (enclayer1) {Enconding1};
    \node [block, below right=of enclayer1, xshift=-1.8cm] (enclayer2) {Enconding2};
    \node [block, below right=of enclayer2, xshift=-1.6cm] (enclayer3) {Enconding3};
    % (None, 168, 18) 
    
    \node [block, below right=of enclayer3] (up1) {Enconding4};
    
    % Decoder (Expanding Path)
    \node [block, above right=of up1] (declayer1) {Decoding1};
    \node [block, above right=of declayer1, xshift=-1.6cm] (declayer2) {Decoding2};
    \node [block, above right=of declayer2, xshift=-1.8cm] (declayer3) {Decoding3};
    \node [block, above right=of declayer3, xshift=-2cm] (output) {Output};
    
    % Skip Connection
    % \draw [line] (pool1) -- ++(0,-1) -| (up1);
    
    % Connections
    \draw [line] (input) -- (enclayer1);
    \draw [line] (enclayer1) -- (enclayer2);
    \draw [line] (enclayer2) -- (enclayer3);
    \draw [line] (enclayer3) -- (up1);
    \draw [line] (up1) -- (declayer1);
    
    
    \draw [line] (declayer1) -- (declayer2);
    \draw [line] (declayer2) -- (declayer3);
    \draw [line] (declayer3) -- (output);
    
    
    \draw [line] (input) -- (output);
    \draw [line] (enclayer1) -- (declayer3);
    \draw [line] (enclayer2) -- (declayer2);
    \draw [line] (enclayer3) -- (declayer1);
    
    
    \end{tikzpicture}}
	\caption{Ilustração uma rede UNET.}
	\label{fig:unet_graph}
\end{figure}


\subsection{RNN\label{se:rnn_sec}}

As Redes Neuronais Recorrentes (RNN) são projetadas para processar sequências de dados, onde a ordem dos elementos é importante. Elas funcionam passando informações de um neurónio para outro em uma cadeia, o que permite que cada neurónio seja influenciado pelo estado anterior da rede.\\
Isso é feito através de loops internos que permitem à rede "lembrar" informações de etapas anteriores. No entanto, as RNNs enfrentam dificuldades ao tentar lembrar informações de longo prazo, devido ao problema conhecido como desvanecimento do gradiente, onde os gradientes se tornam muito pequenos e impedem a atualização eficaz dos pesos da rede.\\

\subsubsection{LSTM\label{se:lstms_sec}}

As redes \gls{LSTM} são um tipo especial de RNN projetado para superar os problemas de memória de longo prazo encontrados nas RNNs. Isto é conseguido através de uma estrutura de célula que mantém informações ao longo do tempo, permitindo que a rede lembre detalhes importantes mesmo após muitos passos no tempo.\\
As \gls{LSTM}s usam mecanismos de portão para controlar o fluxo de informações, permitindo que elas ignorem informações irrelevantes e mantenham as informações relevantes. Isso torna-as particularmente eficazes em tarefas que exigem o entendimento de dependências de longo prazo em dados sequenciais.\\


O uso de \gls{LSTM} para previsões é uma área comum, mas aqui é seguido através das ideas partilhas em \cite{Hewamalage2021}, e reforçado pelo uso em previsões energéticas demonstados em \cite{Costa2022} \\


\subsection{Transformer\label{se:transformer_sec}}

Os Transformers são um tipo de arquitetura de modelo que utiliza mecanismos de atenção para pesar a importância de diferentes partes de um dado de entrada, primeiro apresentado em \cite{Vaswani2017}.\\
Em vez de processar os dados sequencialmente, como as RNNs, os Transformers processam todos os elementos do dado de entrada simultaneamente. Isso é feito através de um mecanismo de atenção que calcula uma pontuação de atenção para cada par de elementos no dado de entrada, indicando quão relevante um elemento é para o outro. Essas pontuações de atenção são então usadas para ponderar a contribuição de cada elemento ao resultado final. \\
Isso permite aos Transformers capturar dependências de longo alcance nos dados de forma eficiente, tornando-os extremamente eficazes para tarefas de processamento de linguagem natural, como tradução automática e sumarização de texto.\\
% TODO: ref para cahtgpt e dall-e e assim
Este tipo de desenho é a base para os modelos generativos mais conhecidos como o chatGPT para linguagem ou o Dall-E para imagens.\\




\label{ch:contexto}

% Dados
\newpage
\thispagestyle{plain}
\chapter{Dados}

\section{Dados Utilizados \label{se:dados_crus}}

Os dados em estudo são do mercado energético espanol, retirados do site :
https://www.esios.ree.es/es


\csvautotabular{../data/indicators_metadata.csv}


\subsection{Aquisição dos Dados (depth 2)}

No ambito da automatização destes dados foi modificado o repositorio ESIOS (creditar, como?) para ser usado como uma biblioteca de python, aberta, em pypi.
Sendo uma ferramenta mais facilmente acessivel para a extrair dados do mercado espanhol:
https://pypi.org/project/pyesios/



\section{Estudo dos dados  \label{se:dados_estudo}}

Os dados que propunho a prever são: "UpwardUsedSecondaryReserveEnergy", "DownwardUsedSecondaryReserveEnergy"

\begin{figure}[H]
  \centering
  \includegraphics[width=0.8\textwidth]{../plots/targets_timeseries.png}
  \caption{Serie Temporal dos dados alvo}
  \label{fig:target_timeseries}
\end{figure}

Para termos uma melhor percepção dos mesmos segue algumas janelas temporais mais pequenas.

\begin{figure}[H]
  \centering
  \includegraphics[width=0.8\textwidth]{../plots/target_timeseries_windows.png}
  \caption{Janelas Temporais dos dados alvo}
  \label{fig:target_timeseries_windows}
\end{figure}


Estas mostram claramente que ambos os atributos mantêm um comportamento tanto discreto, como linear, isto é, que ou existe algum valor, ou é zero, e se existe valor este tem comportamento linear.



A distribuição destes dados é claremente exponencial. O que é importante para a escolha de alguns parametros no modelação

		
\begin{figure}[H]
  \centering
  \includegraphics[width=\textwidth]{../plots/target_histograms.png}
  \caption{Frequência dos dados alvos}
\end{figure}


\subsection{Correlações (depth 2)}

Os modelos vão depender bastante de correlação entre variaveis.
Nesta secção queremos tentar identificar se há visiveis relações entre as variaveis, e se há relações temporais  visiveis nas colunas alvo.



\begin{figure}[H]
  \centering
  \includegraphics[width=\textwidth]{../plots/feature_correlation.png}
  \caption{Correlação entre atributos}
\end{figure}

As correlações entre variveis parecem muitos escassas o que apresenta já que a previção deste dados usando estas variaveis vai ser um problema dificil.
Por norma é feito uma seleção de  atributos baseado nestas correlações, eliminando assim os atributos que ajudam menos, ou ate prejudicam os modelos.



\begin{figure}[H]
  \centering
  \includegraphics[width=\textwidth]{../plots/autocorrelation.png}
  \caption{Serie Temporal dos dados alvo}
\end{figure}


A autocorrelação, em ambos os "targets", é mais forte nas 3 horas mais proximas, e nos pontos com diferença de 12 e 24 horas.
É de notar que estes valores são baixos, prometendo já tambem uma baixa regressividade temporal.
Outro ponto a denotar é que os objectos não têm um comportamento completamente linear, i.e., parece existir um comportamento discreto na questão ser alocado ou não esta reservas secundárias, e caso seja alocado, aí existir alguma linearidade.
Logo qualquer tipo de modelação terá de resolver primeiramente este problema.

Estas relações mostram que em termos de atributos usados vai ser um desafio complicado para qualquer tipo de modelo.

No âmbito desta disserteção queremos verificar a qualidade das previsões usando estes mesmo atributos, logo, não será feita seleção dos mesmos.

A nível da relação temporal, a maior parte dos modelos que iremos testar aplica um janela na dimensão temporal, usando todos os valores nessa janela, e aplicando os pesos nessas distancias que mais se enquadram. Logo também não é relevante escolher apenas as distancias temporais com maior correlação, pois os modelos vão fazer essa pesagem.

\subsection{Agrupamento \label{se:clustering}}

Uma das possibilidades na modelção será a utilização de grupo de valores, classes, em conjunto com a linearidade.
Devido ao comportamente não exclusivamente linear (tenho de procurar um nome para isto) é tambem estudado as possiveis agregassões (clustering) em que podemos dividir os valores em classe.
Tendo por base que uma das classes é o valor zero, devido ao comportamento não linear desta série, vamos apenas testas quantos, e quais, as melhores classes em que podemos dividir os dados.


preccesos de clsutering elbow e sillpute e merdas, fazer 3 pagindas disto.

clustering
	deu cerca de 15
	5 seria um bom numero de clusters pelos estudos
	mas segundo o tipo de problema 3 + 1(zero) foi o ideal

Ambos os casos apontam um assintota na relação interna dos clusters, a partir de cerca de 5 clusters, sendo que o melhor valor dos verificados seria com 15 clusters.
Para a nossa questão, queremos algumas classes, mas quanto menos classes mais facil será para os modelos correctammente identificar a que classe pertence. Logo para os valores apresentados, escolhemos 5 clusters, sendo que este já pode ser um numero elevado de classes, logo usemos 3 clusters se os modelos tiverem muuita dificuldade com 5.



\section{Tratamento dos dados  \label{se:data_treatment}}

Normalizaçao
A normalizaçao foi deixada por ser aprendida nos modelos, sendo que todos têm como segunda camada, uma de normalização.

Limpeza

Podemos ver pelos graficos seguintes que a existem alguns outliers, sendo estes definidos como 3 desvios padrão de distância à média.

graphhhhhhhhh


No caso de "" e "" existe um grande salto nos valores logo o aqui os dados são partidos e estudados os outliers nas duas diferentes zonas.
A limpeza destes dados é feita apenas substituindo pelo ................................................

Fica também como motivo de estudo se a remoção deste outliers ajuda na modelação.


Estudemos tambem o caso de dados em falta. Alguns destes atributos têm certas entras vazias, e como podemos ver alguns não têm alguns anos inteiros.
Como queremos usar o maximo de dados possiveis iremos usar tecnicas de imputing nesses dados 
............................................
imputing
https://arxiv.org/abs/2102.03340

cleaning
	outliers
		naquela parte que parte fazer a limpeza em duas vezes
		melhora se tirarmos os outliers?
		
	imputing de missing data
	
features adicionais:
	apenas as temporais	

Por ultimo foi adicionado ao dados mais atributos, sendo eles todos de cariz temporal. É adicionado atributos correspondentes à hora, ao dia do ano, ao dia da semana, ao dia do mês, mês, ano. TODO


\section{Considerações adicionais  \label{se:dados_plus}}

Talvez aqui uma secçao para finalizar e mostrar algumas coisas

\label{ch:dados}

% Arquiteturas a estudar
\newpage
\thispagestyle{plain}
\chapter{Arquitecturas de Modelos\label{ch:arquiteturas_modelos}}

Grande parte da literatura sobre previsões em modelos de apredizagem apresenta as mesmas arquiteturas, sendo que são depois aprimoradas consoate os dados e o problema. \\
Apresento aqui as aquiteturas mais usadas em previsões, como tambem algumas usadas noutros ramos tentado prever a compatibilidade neste problema. \\
As arquitecturas irão seguir um esquema logíco comum, um bloco de camadas de entrada, um bloco principal e um por fim um bloco interpretativo. \\
As dimensionalidades destas camadas é o que irá formar as diferentes arquitecturas em estudo. \\

\section{Camadas\label{se:layers}}

Para uma construção de modelos usando a ferramenta \href{https://keras.io/}{keras} a unidade básica são as camadas. Estas representação um operação, com uma entrada, e uma saida, e com possiveis parametrizaçoes específicas. \\
Estas camadas ligadas entre si, perfazem um \"profundo\" de camadas neuronais, chamado profundo pois tem mais  que uma camada. \\

Apresento aqui as camadas utilizadas nos modelos aplicados.\\

\subsection{Dense\label{se:dense_layer}}

A camada dense pega num input, cria um numero de neurónios, \textit{N}, também chamado numero de filtros, onde cada neuronio (filtro), recebe informação de cada uma das entradas, e todos os neuronios ligam a todas as dimensões de saida. \\
Cada neuronio gera uma operação, inicialmente aleatoria, para tentar reproduzir uma função que traduza a entrada na saida ideal. \\
Esta camade é altamente influenciada pelo \textit{Perceptão} inicialmente proposto por Franck Rosenblatt\cite{Rosenblatt1958}. Este apresentava um \textit{Perceptão} que fazia uma decisão binária baseado na somas pesadas de todas as entradas. \\
A idea é a base utilizada actualmente, mas apresentava alguma limitações, e muita computação, o proposto por Minsky and Papert\cite{Minsky1969}, eleva a idea com a introdução da funcção de activação e o bias. \\
A utilizção mais recorrente actual é a proposta por Haykin\cite{Haykin1999}, que baseada nas anteriores tem a seguinte apresentação:

\begin{figure}[H]
	\centering
	\includegraphics[width=0.8\textwidth]{Imagens/percepton.png}
	\caption[Ilustração da camada de Dense.]{Adaptado de \cite{Haykin1999}}
	\label{fig:dense}
\end{figure}


\subsection{Convolution\label{se:conv_layer}}

A camada de convoluções difere da dense no sentido em que os filtros (neuronios) não são criados aleatoriamente, mas sim cada filtro trata de uma parte da camada de entrada.
Cada filtro é criado a partir de operações para aquela zona da entrada. Ao conjunto destes filtros dá-se o nome de \textit{feature map}, mapa de atributos, onde normalmente cada filtro aprende a \"ver\" um conceito diferente da entrada. \\
No caso de convoluções em séries temporais, como neste trabalho, os filtros são criados em convoluções temporais\cite{sss}. Pegando no exemplo do trabalho em questão, se usarmos um tamanho da janela de 3 (horas), cada filtro criado terá representaçoes de todos os conjuntos subsequentes de 3 horas. \\
(ref)(TODO: descrever melhor, e provavelmente fazer eu uma imagem para explicar em timeseries).

\begin{figure}[H]
	\centering
	\includegraphics{Imagens/conv_layer.png}
	\caption{Ilustrção da camada de Convulução}
	\label{fig:conv_blcok}
\end{figure}


\subsection{MaxPooling\label{se:max_pooling}}

As camadas de pooling fazem operaçoes para redimensionar os filtros anteriores. \\
Esta camada usada é MaxPooling, que escolhe o maior valor dentro da janela de strides, e aplica na saida. \\
Outros exemplos são Average Pooling ou global Pooling. \\
https://arxiv.org/pdf/2203.01016.pdf

(ref) na imagem
\begin{figure}[H]
	\centering
	\includegraphics{Imagens/pooling.jpg}
	\caption{Ilustrção do efeito da camada de Pooling}
	\label{fig:pooling}
\end{figure}


\subsection{\href{https://keras.io/api/layers/regularization_layers/dropout/}{Dropout}\label{se:dropout}}

Dropout é uma camada que elimina/ignora alguns dos neuronios da camada anterior. Este procedimente impede o overfitting, ajudando na generalização. \\
(ref) na imagem
\begin{figure}[H]
	\centering
	\includegraphics{Imagens/dropout.png}
	\caption{Ilustrção do efeito da camada de dropout}
	\label{fig:dropout}
\end{figure}



\section{Blocos\label{se:blocos}}

Todas as arquiteturas em análise irão ter por base um bloco de camadas neuronais. A formação dessas arquitecturas passa pelas diferentes maneiras que se pode utilizar o bloco principal. Repetições em serie ou em paralelo são um exemplo. \\

\subsection{Bloco Dense\label{se:dense}}

O bloco dense sendo ele o mais simples é formado por duas camadas Dense \ref{se:dense_layer} \cite{}, em que a primeira apresenta um numero maior de filtros que a segunda. \\
Estas camadas não são mais do que uma criação de filtros aleatórios combinando as entradas, para criar todos os filtros de saida. São a base das camadas intrepretativas. A acumulação em série (stacked) de camadas de dense está ligada a melhorias nas capacidades predictivas dos modelos \cite{VLHelen2021}. \\
Exemplo ilustrativo do nosso bloco basico onde entrariam 16 filtros na primeira camada e para finalizar o bloco com 2 filtros \\

\begin{figure}[H]
	\centering
	\includegraphics{Imagens/dense_layer.png}
	\caption{Bloco Dense}
	\label{fig:dense_blcok}
\end{figure}

\begin{figure}[H]
	\centering
	\includegraphics{Imagens/dense_block.png}
	\caption{Ilustrção do bloco de Dense}
	%\label{fig:dense_blcok}
\end{figure}

\subsection{Bloco CNN\label{se:cnn}}

Bloco de CNN é aqui definido como uma convolução na dimensão temporal seguido de camadas para combater o overfitting, MaxPooling e Dropout. \\
Normalmente usada em processamentos de imagens, o uso de convuluções temporais é tambem por si mesmo uma ideia forte. \\

explicar om que e CNN
imagem

Usada tambem as ideias de attention, residual e o que eu chamei broad

%figura otima: https://www.researchgate.net/publication/344229502_A_Novel_Deep_Learning_Model_for_the_Detection_and_Identification_of_Rolling_Element-Bearing_Faults/figures?lo=1
TODO citar
\begin{figure}[H]
	\centering
	\includegraphics{Imagens/cnn_block.png}
	\caption{Ilustrção do bloco de CNN}
	\label{fig:cnn_block}
\end{figure}


\subsection{Bloco LSTM\label{se:lstm}}

O uso de LSTM para previsões é uma area comum, mas aqui é seguido através das ideas partilhas em \cite{Hewamalage2021}, e reforçado pelo uso em previsões energéticas demonstados em \cite{Costa2022} \\
O bloco LSTM é a aplicaçao das RNN, aqui sendo apenas definido como uma camada de LSTM. \\
Estes blocos mantêm dentro de si ligações a diferentes camadas temporais, e cada filtro criado, mantêm uma "memória" dos filtros passados. \\
Bastante utilizado em modelação de linguagem.

imagem


\section{Arquiteturas \label{se:arquitecturas}}

\subsection{Vanilla \label{se:vannila}}

O termo "Vanilla" aqui é aplicado para aquitecturas que apenas usam um bloco de cada, um de entrada, um principal, e um interpretativo. \\
Como exemplo a arquitetura de "VanillaCNN"

imagem da mesma

\subsection{Stacked\label{se:stacked}}

Stacked refere-se a "amontoado" onde se utiliza o bloco principal várias vezes em série.E apenas um bloco de  entrada e um interpretativo. \\
Como exemplo a arquitetura de "StackedCNN"

imagem da mesma

\subsection{MultiHead\label{se:multihea}}

Multihead é o termo para quando os blocos de entrada e principais são repetidos paralelamente, um caminho para cada atributo, ou uma outra paralelização à escolha. Sendo depois concatenadas essas camadas e passadas juntas para a camada interpretativa. \\
Aqui foi usado sempre a paralelização por atributos, e ao invês de fazer Mulithead no sentido de multiplas entradas, para simplicidade de programação, foi feito um paralização interna no modelo, apos a camada de entrada, onde a mesma é repetida para cada atributo. \\
Foi testado a diferença, e para os dados usados não havia diferenças de qualidade, mas sim em tempo de treino, logo a mais rapida foi a escolhida. \\
Como exemplo a arquitetura de "MultiheadCNN"

imagem da mesma

\subsection{MultiTail\label{se:multitail}}

Esta arquitectura tem o mesmo conceito que a anterior a nivel de paralelização, mas neste caso esta é feita apenas na camada interpretativa. Sendo que o resultado do bloco principal é repetido para criar a paralelização. \\
Neste caso foi paralelizado com o numero de tempos a prever, 24 horas, 24 objectos de saida destas modelos.  \\
A grande diferença desta arquitectura para a "Vanilla" que preve 24 horas, é que aqui cada hora tem o seu proprio valor de função de perda, logo o modelo como que está a treinar 24 modelos diferentes, e no caso "Vanilla" a função de perda é ùnica e é a media do erro das horas todas. \\
Como exemplo a arquitetura de "MultiTailCNN"

imagem da mesma

\subsection{UNET\label{se:UNET}}

Normalmente usando em modelção de imagens, a arquitectura UNET passa por criar uma rede de expansão dos filtros, usando convoluções, e de seguida uma rede de contracção dos mesmo, até aos tamanhos pretendidos.\\
O bloco principal contextualmente o mesmo que o CNN.\\
Nas suas ligações UNET junta informação de filtros passados (não de nivel temporal mas de rede neuronal) para realçar informação já trabalhada, e assim identificar padrões de vários contextos diferentes.\\
É habitual tambem adicionar aos blocos principais portões de atenção, portões residuais. Estas duas tecnicas são tambem estudadas aqui.\\
É chamada assim pois é uma rede (NET) que forma um U na sua expansão e contracção.\\

Como exemplo a arquitetura de "UNET"

imagem da mesma


\section{Considerações adicionais\label{se:modelos_plus}}

Aqui e dizer que os modelos utilizados para teste sao as combinacoes deste blocos nestas aquiteturas.

Imagens de layers criadas com 
dense
http://alexlenail.me/NN-SVG/index.html
\label{ch:arch}




% Métodos
\newpage
\thispagestyle{plain}
\chapter{Métodos}

Para responder a primeira questão estudou-se o comportamento do parâmetro p na equação publicada pela ENTSO-E para a Banda de Regulação Secundária a Subir \cite{CMVM2018}:
\begin{equation}
\label{eq:eq_entso-e}
    BRSsubir = p \times \sqrt{a \times Lmax + b^2} - b 
\end{equation}
TODO: explicação das variaveis

Aplicando os dados XXXX (dados históricos do mercado MIBEL), 

Usandos os dados XXXXX para o calculo directo do parâmetro p e comparando com o mesmo parâmetro apresentado na tese de Célia Carneiro \cite{Carneiro2016}, temos a seguinte distribuição de valores:

\includegraphics{Imagens/Histogram p parameter.png}
\includegraphics{Imagens/p por Hora.png}

Verifica-se que não têm qualquer relação entre si.

Para a normalização deste parâmetro à Hora, estudou-se o erro entre a Banda a Subir calculada através das normalizações e a Banda a Subir disponível nos dados. 

\begin{table}[H]
\centering
\caption{Isto é um exemplo de uma tabela. Se fôr igual(copiada) a outro autor deve ser pedido autorização para reproduzir.}
\begin{tabular}{p{5cm}p{3.5cm}p{2cm}p{2cm}}
\toprule %thicker line
\textbf{Normalização/Erro} & \textbf{MAE} & \textbf{RMSE} & \textbf{Mediana AE} \\ \hline
\multirow{1}{*}{media}   & 23.03 & 29.15 & 19.04            \\ \hline
\multirow{1}{*}{mediana}   & 22.95 & 29.14 & 18.93            \\ \hline
\multirow{1}{*}{media ponderada}   & 23.39 & 29.57 & 19.28            \\ \hline
\end{tabular}
\end{table}

A normalização que traz erros mais baixos à Banda é a mediana.
Comparando com os p \cite{Carneiro2016}: 

\includegraphics{Imagens/p por Hora norm.png}

Podemos verificar que os valores não coincidem.
Comparando as bandas calculadas:

\includegraphics{Imagens/Banda a Subir.png}

Retiramos as médias dos erros percentuais e podemos observar em:

\includegraphics{Imagens/Erro per Banda a Subir.png}

Que conclui que o p calculado agora tem uma prestação bastante mais perto da realidade. 
Onde cerca de 25\% dos casos cai dentro da margem de erro de 5\% na Banda. E na outra tese apenas 10\% cai dentro dessa margem de erro.



Aqui descrever qual o método seguido para responder às perguntas de investigação, o método pode incluir o recurso a uma metodologia reconhecida internacionalmente e já publicada por exemplo numa norma ISO\cite{fet_skaar_2006}.
As figuras e tabelas colocadas devem ser sempre referidas no meio do texto tal como todas as referências que aparecem listadas no fim do documento. 



\section{Modelos estatiscos  \label{se:dados_estudo}}

\subsection{subsection ARIMA (depth 2)}

\subsection{subsubsection Gerador de dados (depth 2)}
distribuiçao

clustering




\section{Forecat  \label{se:dados_estudo}}

\subsection{subsection Construtor de modelos (depth 2)}

\subsection{subsubsection Gerador de dados (depth 2)}
distribuiçao

clustering


\section{Forecat  \label{se:dados_estudo}}

\subsection{subsection Construtor de modelos (depth 2)}

\subsection{subsubsection Gerador de dados (depth 2)}
distribuiçao

clustering



\label{ch:metodos}

% Resultados e discussão
\newpage
\thispagestyle{plain}
\chapter{Resultados e discussão}

Os resultados das experiências são apresentados por experiência, sendo que cada um vai testando e eliminando parâmetros na modelação.\\
Após análise inicial na modelação, foi concluído que tentar modelar apenas um dos atributos de cada vez leva a melhores resultados do que tentar modelar os dois no mesmo modelo.\\
Foi também observado que os dois atributos em questão são análogos, logo os sistemas que melhor representam um dos atributos também são semelhantemente eficazes no outro. \\
Assim, todas as experiências foram realizadas usando apenas um dos atributos alvo, sendo este o "UpwardUsedSecondaryReserveEnergy", e o atributo de alocação comparativo é "SecondaryReserveAllocationAUpward".\\

\section{Métricas\label{se:metricas}}

Para a comparação efetiva dos modelos, iremos usar as seguintes métricas: RMSE, erro absoluto, r2 score, percentagem ótima, alocação em falta, alocação a mais.\\
As métricas de decisão final são aquelas que representam melhor o objetivo de reduzir os custos de alocação das reservas secundárias, portanto, são as métricas de alocação em falta e em excesso, sendo que a soma delas é o erro absoluto.\\

\begin{equation}
\label{eq:rmse}
   % RMSE (y, \hat{y}) = \sqrt{\frac{\sum_{i=0}^{N - 1} (y_i - \hat{y}_i)^2}{N}}
\end{equation}

\begin{equation} \label{eq:abse} 
    \text{Erro Absoluto} = \sum_{i=0}^{N - 1} \left| y_i - \hat{y}_i \right| 
\end{equation}

%\begin{equation} \label{eq:r2score} R^2\ \text{score} = 1 - \frac{\sum_{i=0}^{N - 1} (y_i - \hat{y}_i)^2}{\sum_{i=0}^{N - 1} (y_i - \bar{y})^2} \end{equation}

\begin{equation} 
    \label{eq:miss_alloc} 
    \text{Alocação em falta} = 
    \begin{cases} 
        0 & ,\text{if } \hat{y} \geq y \\
        y - \hat{y} & ,\text{otherwise} 
    \end{cases} 
\end{equation}

\begin{equation} 
    \label{eq:opt_perc} 
    \text{Percentagem ótima} = \frac{1}{N} \sum_{i=1}^{n} 1 [\hat{y}_i \geq y_i  \&  \hat{y}_i \leq \text{alloc}]
\end{equation}

Onde $\hat{y}$ são as previsões dos modelos, $y$ são os valores reais utilizados (UpwardUsedSecondaryReserveEnergy), e \text{alloc} são os valores alocados (SecondaryReserveAllocationAUpward). \\

\section{Experiências\label{se:experiments}}

Para a comparação, o erro absoluto para o ano de 2021 na alocação feita é de 3889367.4. \\

\subsection{Arquiteturas e números de épocas\label{se:archs_epocs}}

\subsubsection{Arquiteturas\label{se:archs_res}}

\begin{table}[H]
\caption{Resultados Arquiteturas}            
%\resizebox{\linewidth}{!}{\begin{tabular}{lrrlrrrrrrr}
\toprule
name & rmse & abs erro & erro comp & r2 score & mape score & alloc missing & alloc surplus & optimal percentage & better allocation & beter percentage \\
\midrule
VanillaDense & 1200.20 & 10073188.28 & False & -41.27 & 192.50 & 487.24 & 10072701.04 & 1.13 & 0.77 & 1.17 \\
StackedCNN & 453.12 & 3717806.99 & True & -5.03 & 82.67 & 52038.16 & 3665768.83 & 35.04 & 34.48 & 38.34 \\
UNET & 183.58 & 1166186.91 & True & 0.01 & 17.02 & 687071.63 & 479115.27 & 60.39 & 60.39 & 85.47 \\
VanillaCNN & 174.96 & 1124205.72 & True & 0.10 & 16.45 & 614355.90 & 509849.82 & 61.79 & 61.79 & 86.18 \\
\bottomrule
\end{tabular}
}
\end{table}

\subsubsection{Épocas\label{se:epocas_res}}

\subsection{Funções de Perda (Loss) \label{se:loss_exp}}

As funções de perda foram estudadas com duas arquiteturas diferentes, de modo a conseguir ter uma melhor noção do impacto das mesmas. \\

\begin{table}[H]
\caption{Resultados Funções de perda}        
%\resizebox{\linewidth}{!}{\begin{tabular}{lrrlrrrrrrr}
\toprule
name & rmse & abs erro & erro comp & r2 score & mape score & alloc missing & alloc surplus & optimal percentage & better allocation & beter percentage \\
\midrule
UNETmse & 180.84 & 1269085.40 & True & 0.04 & 23.32 & 521000.39 & 748085.00 & 68.78 & 68.78 & 87.31 \\
StackedCNNmae & 198.29 & 1083872.29 & True & -0.15 & 5.93 & 937786.28 & 146086.01 & 43.65 & 43.65 & 83.32 \\
StackedCNNmape & 232.60 & 1257578.80 & True & -0.59 & 1.01 & 1252220.35 & 5358.46 & 19.15 & 19.15 & 80.60 \\
StackedCNNmse & 176.71 & 1103152.68 & True & 0.08 & 14.81 & 673158.53 & 429994.14 & 58.75 & 58.75 & 85.68 \\
StackedCNNwl & 2002.81 & 17092114.23 & False & -116.72 & 320.21 & 0.00 & 17092114.23 & 0.05 & 0.00 & 0.05 \\
UNETmae & 195.49 & 1083463.60 & True & -0.12 & 7.14 & 887646.74 & 195816.87 & 45.95 & 45.95 & 83.70 \\
StackedCNNmsle & 225.42 & 1761315.40 & True & -0.49 & 40.41 & 320813.27 & 1440502.13 & 79.46 & 79.46 & 89.93 \\
StackedCNNmsde & 612.31 & 4978280.05 & False & -10.00 & 81.86 & 3325165.78 & 1653114.27 & 18.27 & 18.06 & 30.14 \\
UNETmsle & 211.40 & 1142731.21 & True & -0.31 & 4.70 & 1048097.40 & 94633.80 & 39.64 & 39.64 & 82.11 \\
\bottomrule
\end{tabular}
}
\end{table}

A nivel de percentagem de modelo melhor, elas esão todas bastante renhidas, mas ha uma claro vantagem quando vemos a percentagem de melhor alocação ou de alocação optima.\\
Sendo que a perda que avança é a de MSLE (Mean Square Log Error).\\
Esta perda é tendicionalmente usada para distribuições exponencias, e onde temos bastantes outliers, que devem ser considerados. O que é o caso no nosso problema.\\
Esta experiência valida a observação feito na análise estatistica sobre o mesmo.\\

\subsection{Hiperparametrização\label{se:hiper}}

\subsubsection{Activação\label{se:activ}}

\begin{table}[H]
\caption{Resultados Ativação}    
%\resizebox{\linewidth}{!}{\begin{tabular}{lrrlrrrrrrr}
\toprule
name & rmse & abs erro & erro comp & r2 score & mape score & alloc missing & alloc surplus & optimal percentage & better allocation & beter percentage \\
\midrule
StackedCNN relu relu & 207.66 & 1113406.16 & True & -0.27 & 3.90 & 1028475.81 & 84930.35 & 38.64 & 38.64 & 82.31 \\
StackedCNN tanh tanh & 236.24 & 1289434.20 & True & -0.64 & 0.85 & 1288660.90 & 773.30 & 11.47 & 11.47 & 80.40 \\
StackedCNN relu tanh & 236.24 & 1289434.20 & True & -0.64 & 0.85 & 1288660.90 & 773.30 & 11.47 & 11.47 & 80.40 \\
StackedCNN relu softsign & 236.31 & 1290138.00 & True & -0.64 & 0.86 & 1289376.80 & 761.20 & 11.29 & 11.29 & 80.40 \\
StackedCNN linear tanh & 236.24 & 1289434.20 & True & -0.64 & 0.85 & 1288660.90 & 773.30 & 11.47 & 11.47 & 80.40 \\
StackedCNN softsign linear & 1112.70 & 8936608.01 & False & -35.34 & 178.94 & 8185.60 & 8928422.41 & 7.99 & 7.50 & 8.60 \\
StackedCNN softsign softsign & 236.24 & 1289436.03 & True & -0.64 & 0.85 & 1288662.99 & 773.04 & 11.00 & 11.00 & 80.39 \\
StackedCNN softplus elu & 294.00 & 2287913.46 & True & -1.54 & 52.62 & 273134.72 & 2014778.74 & 79.27 & 79.25 & 88.22 \\
StackedCNN softplus softplus & 393.20 & 3191326.78 & True & -3.54 & 72.87 & 84140.16 & 3107186.62 & 69.22 & 69.15 & 71.84 \\
StackedCNN relu selu & 262.76 & 1986551.87 & True & -1.03 & 43.04 & 261245.83 & 1725306.04 & 77.31 & 77.17 & 86.49 \\
StackedCNN linear elu & 297.39 & 2368520.21 & True & -1.60 & 55.86 & 177934.14 & 2190586.07 & 87.47 & 87.47 & 92.53 \\
StackedCNN softsign tanh & 236.24 & 1289434.20 & True & -0.64 & 0.85 & 1288660.90 & 773.30 & 11.47 & 11.47 & 80.40 \\
StackedCNN softplus exponential & 461.92 & 3782992.35 & True & -5.26 & 82.96 & 50599.52 & 3732392.83 & 32.55 & 31.88 & 35.55 \\
StackedCNN softplus selu & 287.86 & 2249601.96 & True & -1.43 & 53.57 & 218301.68 & 2031300.27 & 84.44 & 84.44 & 91.23 \\
StackedCNN softplus softsign & 236.36 & 1291380.19 & True & -0.64 & 0.90 & 1290725.89 & 654.30 & 9.39 & 9.39 & 80.40 \\
StackedCNN softplus linear & 208.37 & 1113888.40 & True & -0.27 & 3.74 & 1032928.56 & 80959.84 & 38.23 & 38.23 & 82.30 \\
StackedCNN softsign relu & 203.70 & 1100693.28 & True & -0.22 & 4.92 & 986235.38 & 114457.90 & 41.64 & 41.64 & 82.69 \\
StackedCNN softsign exponential & 508.90 & 4187396.73 & False & -6.60 & 91.15 & 32147.81 & 4155248.92 & 25.81 & 24.91 & 28.09 \\
StackedCNN linear softsign & 236.24 & 1289434.20 & True & -0.64 & 0.85 & 1288660.90 & 773.30 & 11.47 & 11.47 & 80.40 \\
StackedCNN relu exponential & 542.88 & 4482255.28 & False & -7.65 & 96.23 & 31472.22 & 4450783.06 & 22.04 & 21.01 & 24.00 \\
StackedCNN softsign selu & 2053.61 & 17501165.80 & False & -122.77 & 323.70 & 73467.70 & 17427698.09 & 0.01 & 0.00 & 2.98 \\
StackedCNN tanh softplus & 203.53 & 1097751.79 & True & -0.22 & 4.71 & 986626.52 & 111125.27 & 41.31 & 41.31 & 82.73 \\
StackedCNN tanh linear & 543.57 & 4512608.50 & False & -7.67 & 97.01 & 27030.50 & 4485578.00 & 20.81 & 19.78 & 22.73 \\
StackedCNN relu elu & 481.58 & 3739379.02 & True & -5.81 & 83.92 & 84681.14 & 3654697.88 & 50.13 & 49.65 & 53.43 \\
StackedCNN linear exponential & 457.52 & 3752241.97 & True & -5.14 & 83.28 & 52642.15 & 3699599.81 & 29.83 & 29.21 & 33.15 \\
StackedCNN relu softplus & 187.41 & 1193440.68 & True & -0.03 & 17.36 & 716295.64 & 477145.04 & 59.08 & 59.08 & 85.03 \\
StackedCNN linear selu & 559.84 & 4606574.05 & False & -8.20 & 98.93 & 35641.95 & 4570932.10 & 22.53 & 21.82 & 24.82 \\
StackedCNN softplus relu & 380.03 & 3059764.47 & True & -3.24 & 69.45 & 99012.20 & 2960752.27 & 70.70 & 70.58 & 74.00 \\
StackedCNN linear relu & 614.15 & 4962488.83 & False & -10.07 & 100.79 & 66478.05 & 4896010.78 & 19.20 & 18.28 & 23.60 \\
StackedCNN linear softplus & 321.88 & 2577000.96 & True & -2.04 & 59.88 & 149354.46 & 2427646.50 & 88.59 & 88.59 & 92.51 \\
StackedCNN tanh relu & 548.29 & 4556413.79 & False & -7.82 & 97.85 & 26362.43 & 4530051.37 & 20.66 & 19.65 & 22.52 \\
StackedCNN softsign softplus & 900.71 & 6351850.55 & False & -22.81 & 117.73 & 22655.51 & 6329195.04 & 17.30 & 16.55 & 18.83 \\
StackedCNN linear linear & 238.22 & 1866499.25 & True & -0.67 & 43.41 & 292863.60 & 1573635.64 & 81.02 & 81.02 & 90.41 \\
StackedCNN tanh softsign & 236.24 & 1289436.20 & True & -0.64 & 0.85 & 1288663.18 & 773.02 & 11.00 & 11.00 & 80.39 \\
StackedCNN softsign elu & 205.05 & 1105893.27 & True & -0.23 & 4.88 & 995748.76 & 110144.51 & 41.17 & 41.17 & 82.57 \\
StackedCNN relu linear & 431.38 & 3464311.65 & True & -4.46 & 76.14 & 133075.54 & 3331236.11 & 45.08 & 44.54 & 51.57 \\
StackedCNN softplus tanh & 236.24 & 1289434.20 & True & -0.64 & 0.85 & 1288660.90 & 773.30 & 11.47 & 11.47 & 80.40 \\
\bottomrule
\end{tabular}
}
\end{table}

\subsubsection{Optimizaçao\label{se:opt}}

\begin{table}[H]
\caption{Resultados Optimizadores}    
%\resizebox{\linewidth}{!}{\begin{tabular}{lrrlrrrrrrr}
\toprule
name & rmse & abs erro & erro comp & r2 score & mape score & alloc missing & alloc surplus & optimal percentage & better allocation & beter percentage \\
\midrule
StackedCNN Adadelta & 1946.60 & 16304588.30 & False & -110.20 & 298.49 & 121783.10 & 16182805.20 & 0.37 & 0.37 & 6.57 \\
StackedCNN Adafactor & 209.16 & 1127287.01 & True & -0.28 & 4.25 & 1036653.47 & 90633.54 & 39.16 & 39.16 & 82.23 \\
StackedCNN Adagrad & 2281.10 & 19250481.97 & False & -151.71 & 354.33 & 50132.10 & 19200349.87 & 0.29 & 0.27 & 3.51 \\
StackedCNN Adam & 507.51 & 4103373.73 & False & -6.56 & 90.96 & 44467.79 & 4058905.95 & 36.50 & 35.86 & 38.94 \\
StackedCNN AdamW & 208.26 & 1114855.55 & True & -0.27 & 3.82 & 1033146.63 & 81708.92 & 38.60 & 38.60 & 82.34 \\
StackedCNN Adamax & 1085.65 & 8916100.25 & False & -33.59 & 177.57 & 64682.73 & 8851417.52 & 3.48 & 2.92 & 6.39 \\
StackedCNN Ftrl & 242.89 & 1847009.05 & True & -0.73 & 39.01 & 341650.54 & 1505358.52 & 77.52 & 77.52 & 89.31 \\
StackedCNN Nadam & 534.57 & 4388148.16 & False & -7.39 & 92.41 & 35224.69 & 4352923.47 & 22.85 & 21.81 & 25.02 \\
StackedCNN RMSprop & 501.82 & 3995286.60 & False & -6.39 & 87.47 & 165111.98 & 3830174.63 & 30.90 & 30.17 & 38.94 \\
StackedCNN SGD & 207.90 & 1125396.48 & True & -0.27 & 4.88 & 1021725.80 & 103670.68 & 40.59 & 40.59 & 82.27 \\
\bottomrule
\end{tabular}
}
\end{table}


\subsection{Janelas Temporais}

\begin{table}[H]
\caption{Resultados Janelas Temporais}
%\resizebox{\linewidth}{!}{\begin{tabular}{lrrlrrrrrrr}
\toprule
name & rmse & abs erro & erro comp & r2 score & mape score & alloc missing & alloc surplus & optimal percentage & better allocation & beter percentage \\
\midrule
StackedCNN 48X 24Y & 479.11 & 3989558.86 & False & -5.75 & 86.62 & 43807.71 & 3945751.15 & 29.86 & 29.19 & 32.72 \\
StackedCNN 98X 1Y & 445.28 & 3671816.03 & True & -4.83 & 81.15 & 54078.28 & 3617737.75 & 37.10 & 36.50 & 40.30 \\
StackedCNN 98X 24Y & 1167.21 & 9770108.13 & False & -39.09 & 191.54 & 1578.73 & 9768529.40 & 2.25 & 1.83 & 2.41 \\
StackedCNN 24X 1Y & 209.47 & 1149047.56 & True & -0.29 & 4.43 & 1055993.70 & 93053.86 & 39.55 & 39.55 & 82.21 \\
StackedCNN 98X 4Y & 495.81 & 4120357.48 & False & -6.23 & 88.95 & 36888.60 & 4083468.88 & 27.15 & 26.22 & 29.58 \\
StackedCNN 24X 4Y & 453.22 & 3779420.50 & True & -5.04 & 82.35 & 52131.12 & 3727289.38 & 33.55 & 32.94 & 36.88 \\
StackedCNN 24X 12Y & 450.27 & 3753076.87 & True & -4.96 & 82.29 & 53778.38 & 3699298.49 & 34.86 & 34.35 & 38.25 \\
StackedCNN 24X 24Y & 468.78 & 3907629.32 & True & -5.46 & 84.62 & 48042.45 & 3859586.87 & 26.35 & 25.65 & 29.52 \\
StackedCNN 168X 4Y & 247.11 & 1941513.71 & True & -0.79 & 45.26 & 272577.02 & 1668936.69 & 82.07 & 82.07 & 90.72 \\
StackedCNN 48X 1Y & 510.34 & 4249569.03 & False & -6.65 & 91.65 & 35151.96 & 4214417.07 & 23.58 & 22.63 & 26.05 \\
StackedCNN 24X 8Y & 209.71 & 1150163.22 & True & -0.29 & 4.41 & 1058328.49 & 91834.73 & 39.32 & 39.32 & 82.24 \\
StackedCNN 48X 4Y & 443.47 & 3672502.43 & True & -4.78 & 80.47 & 56035.63 & 3616466.80 & 33.50 & 33.02 & 36.99 \\
StackedCNN 48X 12Y & 451.66 & 3755277.12 & True & -4.99 & 82.42 & 51133.01 & 3704144.12 & 35.83 & 35.34 & 39.13 \\
StackedCNN 48X 8Y & 425.76 & 3528446.38 & True & -4.33 & 78.29 & 63798.09 & 3464648.29 & 35.06 & 34.73 & 38.87 \\
StackedCNN 98X 8Y & 401.18 & 3287580.95 & True & -3.73 & 74.38 & 76725.24 & 3210855.71 & 51.78 & 51.65 & 55.36 \\
StackedCNN 98X 12Y & 622.31 & 5179835.43 & False & -10.41 & 108.86 & 19985.94 & 5159849.49 & 17.78 & 16.78 & 19.33 \\
StackedCNN 168X 24Y & 352.28 & 2757367.10 & True & -2.64 & 65.61 & 155940.39 & 2601426.71 & 70.78 & 70.65 & 75.40 \\
StackedCNN 168X 8Y & 277.45 & 2189089.58 & True & -1.26 & 51.96 & 226599.69 & 1962489.89 & 84.47 & 84.47 & 91.59 \\
StackedCNN 168X 12Y & 404.20 & 3219562.53 & True & -3.79 & 70.85 & 90454.58 & 3129107.95 & 67.74 & 67.30 & 71.01 \\
StackedCNN 168X 1Y & 568.14 & 4694571.86 & False & -8.47 & 100.18 & 23253.65 & 4671318.21 & 20.47 & 19.49 & 22.15 \\
\bottomrule
\end{tabular}
}        
\end{table}

\subsection{Classificação}


\begin{table}[H]
\caption{Resultados Janelas Temporais}
%\resizebox{\linewidth}{!}{\begin{tabular}{lrrlrrrrrrrrr}
\toprule
name & rmse & abs erro & erro comp & r2 score & mape score & alloc missing & alloc surplus & optimal percentage & better allocation & beter percentage & acc & acc aloc \\
\midrule
StackedCNNClusterLinear 4 arch & 207.47 & 1111659.96 & True & -0.26 & 3.81 & 1027024.21 & 84635.75 & 38.66 & 38.66 & 82.34 & 0.40 & 0.29 \\
StackedCNNClusterLinear 4 interpr & 207.45 & 1111808.59 & True & -0.26 & 3.84 & 1027318.24 & 84490.36 & 38.77 & 38.77 & 82.38 & NaN & NaN \\
StackedCNNClusterLinear 6 arch & 208.03 & 1113342.39 & True & -0.27 & 3.72 & 1032371.26 & 80971.13 & 38.48 & 38.48 & 82.28 & 0.33 & 0.10 \\
StackedCNNClusterLinear 6 interpr & 208.23 & 1114478.16 & True & -0.27 & 3.72 & 1034020.09 & 80458.06 & 38.55 & 38.55 & 82.26 & NaN & NaN \\
StackedCNNClusters 4 arch & 206.56 & 1106934.65 & True & -0.25 & 4.11 & 1017487.36 & 89447.30 & 39.57 & 39.57 & 82.43 & 0.40 & 0.29 \\
StackedCNNClusters 6 arch & 206.94 & 1109766.69 & True & -0.26 & 4.12 & 1019180.77 & 90585.93 & 39.49 & 39.49 & 82.45 & 0.33 & 0.10 \\
\bottomrule
\end{tabular}
}        
\end{table}


\subsection{Pesos}



\begin{table}[H]
\caption{Resultados Janelas Temporais}
%\resizebox{\linewidth}{!}{\begin{tabular}{lrrlrrrrrrr}
\toprule
name & rmse & abs erro & erro comp & r2 score & mape score & alloc missing & alloc surplus & optimal percentage & better allocation & beter percentage \\
\midrule
StackedCNN delta mean & 177.29 & 1204388.33 & True & 0.08 & 19.99 & 508061.06 & 696327.28 & 66.69 & 66.69 & 87.72 \\
StackedCNN no weight & 178.20 & 1100714.22 & True & 0.07 & 14.09 & 697157.86 & 403556.35 & 57.49 & 57.49 & 85.36 \\
VanillaCNN delta mean & 4061.70 & 34669933.86 & False & -483.15 & 634.39 & 0.00 & 34669933.86 & 0.00 & 0.00 & 0.00 \\
VanillaCNN no weight & 1146.52 & 9109107.48 & False & -37.58 & 185.61 & 25950.43 & 9083157.05 & 11.90 & 11.55 & 12.81 \\
\bottomrule
\end{tabular}
}        
\end{table}




Aqui devem constar gráficos e sua análise crítica e ligação com a secção 2 da revisão bibliográfica no sentido de comparar valores e discutir diferenças, por exemplo. A legenda dos gráficos deve seguir a das figuras, isto é, porque também são figuras.\\





% Para introduzir figuras
\begin{figure}[H]
\centering
\includegraphics[width=150pt, keepaspectratio]{Imagens/FigA.png}
% legenda da figura por baixo
\caption{Exemplo de como considerar um gráfico}
\label{fig:graficoecampl} % referência única da figura
\end{figure}


\label{ch:resultados_discussao}

% Conclusões e sugestões futuras
\newpage
\thispagestyle{plain}
\chapter{Conclusões e sugestões futuras}

Primeiramente podemos ver pela análise estatistica\ref{tb:statitics_scores} e aplicando a idea\cite{Elsayed}, simples modelos estatisticos conseguiriam resolver o problema em mãos, melhor do que o que é utilizado actualmente\ref{tb:benchmark_val}, e melhor do que muitos dos modelos profundos que testamos. \\
E se considerarmos ainda que os modelos estatiscos apresentados que apresentam estes resultados, utilizam apenas a variavel em questão, e não todos os outros atributos, a nível de aplicabilidade já são um melhoria. \\




Aqui são dadas as respostas às perguntas de investigação formuladas na secção \ref{se:objetivos}. Não fazer aqui a discussão dos resultados. Essa discussão deve ser feita no capitulo \ref{ch:resultados_discussao}. Não esquecer de indicar sugestões futuras para que um colega possa dar continuidade ao trabalho desenvolvido. 
\label{ch:conclusao}

% Referências Bibliográficas
\renewcommand{\bibname}{Referências} %comando para rename a BIBLIOGRAFIA para REFERENCIAS
\addcontentsline{toc}{chapter}{\numberline{8}Referências}%
%\bibliographystyle{babplain}      % basic style, author-year citations
\printbibliography

% Referências Bibliográficas
\appendix
\newpage
\thispagestyle{plain}
\input{Capitulos/Tese-9-Anexos}
\label{ch:Anexos}

\end{document}
