\chapter{Anexos}

% Para introduzir figuras
\begin{figure}[H]
\centering
\includegraphics[width=150pt, keepaspectratio]{Imagens/FigA.png}
% legenda da figura por baixo
\caption{Exemplo de como considerar um gráfico nos anexos.}
\label{fig:grafico1} % referência única da figura
\end{figure}

\setlength\tabcolsep{4pt}
\begin{table}[H]
\centering
\caption{Isto é um exemplo de uma tabela. Se fôr igual(copiada) a outro autor deve ser pedido autorização para reproduzir.}
\begin{tabular}{p{5cm}p{3.5cm}p{2cm}p{2cm}}
\toprule %thicker line
\textbf{Title 1} & \textbf{Title 2} & \textbf{Title 3} & \textbf{Title 4} \\ \hline
\multirow{3}{*}{entry 1}   & data      & data    & data             \\
                           & data      & data    & data             \\
                           & data      & data    & data             \\ \hline
\multirow{2}{*}{entry 2}   & data      & data    & data             \\
                           & data      & data    & data             \\ \hline
\multirow{4}{*}{entry 3}   & data      & data    & data             \\
                           & data      & data    & data             \\
                           & data      & data    & data             \\
                           & data      & data    & data             \\ \hline
entry 4                    & data      & data    & data             \\ \hline 
\end{tabular}
\end{table}