\chapter{Resultados e discussão}

Os resultados das experiências são apresentados por experiência, sendo que cada um vai testando e eliminando parametros na modulção.
Após análise inicial na modelação foi concluido que usar tentar modelar apenas um dos atributos de cada vez leva a melhor resultados que tentar modelar os dois no mesmo modelo.
Foi também visto que os dois atributos em causa são analogos, logo os sistemas que melhor represetam um dos atributos tambem são semelhantemente eficazes no outro. 
Assim todas as experiências foram realizadas usado apenas um dos atributos alvo, sendo este o "UpwardUsedSecondaryReserveEnergy", e em que o atributo de alocação comparativo é  "SecondaryReserveAllocationAUpward".

\section{Métricas  \label{se:metricas}}

A nivel de comparação efectiva de modelos iremos usar as seguintes metricas: RMSE, erro absoluto, r2 score, percentagem optima, alocação em falta, alocação a mais.
Sendo que as métricas de decisão final são as que representam melhor o objectivo: baixar os custo de alocação das reservas secundárias. logo serão as metricas de alocção em falta e em demasia, sendo que a soma das mesmas é o erro absoluto.

\begin{equation}
\label{eq:rmse}
    RMSE (y, \hat{y}) = \sqrt{\frac{\sum_{i=0}^{N - 1} (y_i - \hat{y}_i)^2}{N}}
\end{equation}

\begin{equation} \label{eq:abse} 
    \text{Absolute\ Error} = \sum_{i=0}^{N - 1} \left| y_i - \hat{y}_i \right| 
\end{equation}

\begin{equation} \label{eq:r2score} R^2\ \text{score} = 1 - \frac{\sum_{i=0}^{N - 1} (y_i - \hat{y}i)^2}{\sum{i=0}^{N - 1} (y_i - \bar{y})^2} \end{equation}


\begin{equation} 
    \label{eq:eq1} 
    \text{alocação em falta} = 
    \begin{cases} 
        0 & ,\text{if } \hat{y} \geq y \\
        y - \hat{y} & ,\text{otherwise} 
    \end{cases} 
\end{equation}

\begin{equation} 
    \label{eq:eq1} 
    \text{optimal percentage} = \frac{1}{N} \sum_{i=1}^{n} 1 [\hat{y}_i \geq y_i  \&  \hat{y}_i \leq alloc]
    \text{(not: \cite{Iverson1962})}
\end{equation}


Onde \textit{$\hat{y}$} são as previsões dos modelos, \textit{y} são os valores reais utilizados, "UpwardUsedSecondaryReserveEnergy", e \textit{alloc} são os valores alocados "SecondaryReserveAllocationAUpward"


\section{Experiência  \label{se:metricas}}


Aqui devem constar gráficos e sua análise crítica e ligação com a secção 2 da revisão bibliográfica no sentido de comparar valores e discutir diferenças, por exemplo. A legenda dos gráficos deve seguir a das figuras, isto é, porque também são figuras.

% Para introduzir figuras
\begin{figure}[H]
\centering
\includegraphics[width=150pt, keepaspectratio]{Imagens/FigA.png}
% legenda da figura por baixo
\caption{Exemplo de como considerar um gráfico}
\label{fig:grafico1} % referência única da figura
\end{figure}
