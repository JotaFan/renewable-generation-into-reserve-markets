\chapter{Resultados e discussão}

Os resultados das experiências são apresentados por experiência, sendo que cada um vai testando e eliminando parâmetros na modelação.\\
Após análise inicial na modelação, foi concluído que tentar modelar apenas um dos atributos de cada vez leva a melhores resultados do que tentar modelar os dois no mesmo modelo.\\
Foi também observado que os dois atributos em questão são análogos, logo os sistemas que melhor representam um dos atributos também são semelhantemente eficazes no outro. \\
Assim, todas as experiências foram realizadas usando apenas um dos atributos alvo, sendo este o "UpwardUsedSecondaryReserveEnergy", e o atributo de alocação comparativo é "SecondaryReserveAllocationAUpward".\\

\section{Métricas\label{se:metricas}}

Para a comparação efetiva dos modelos, iremos usar as seguintes métricas: RMSE, erro absoluto, r2 score, percentagem ótima, alocação em falta, alocação a mais.\\
As métricas de decisão final são aquelas que representam melhor o objetivo de reduzir os custos de alocação das reservas secundárias, portanto, são as métricas de alocação em falta e em excesso, sendo que a soma delas é o erro absoluto.\\

\begin{equation}
\label{eq:rmse}
   % RMSE (y, \hat{y}) = \sqrt{\frac{\sum_{i=0}^{N - 1} (y_i - \hat{y}_i)^2}{N}}
\end{equation}

\begin{equation} \label{eq:abse} 
    \text{Erro Absoluto} = \sum_{i=0}^{N - 1} \left| y_i - \hat{y}_i \right| 
\end{equation}

%\begin{equation} \label{eq:r2score} R^2\ \text{score} = 1 - \frac{\sum_{i=0}^{N - 1} (y_i - \hat{y}_i)^2}{\sum_{i=0}^{N - 1} (y_i - \bar{y})^2} \end{equation}

\begin{equation} 
    \label{eq:miss_alloc} 
    \text{Alocação em falta} = 
    \begin{cases} 
        0 & ,\text{if } \hat{y} \geq y \\
        y - \hat{y} & ,\text{otherwise} 
    \end{cases} 
\end{equation}

\begin{equation} 
    \label{eq:opt_perc} 
    \text{Percentagem ótima} = \frac{1}{N} \sum_{i=1}^{n} 1 [\hat{y}_i \geq y_i  \&  \hat{y}_i \leq \text{alloc}]
\end{equation}

Onde $\hat{y}$ são as previsões dos modelos, $y$ são os valores reais utilizados (UpwardUsedSecondaryReserveEnergy), e \text{alloc} são os valores alocados (SecondaryReserveAllocationAUpward). \\

\section{Experiências\label{se:experiments}}

Para a comparação, o erro absoluto para o ano de 2021 na alocação feita é de 3889367.4. \\

\subsection{Arquiteturas e números de épocas\label{se:archs_epocs}}

\subsubsection{Arquiteturas\label{se:archs_res}}

\begin{table}[H]
\caption{Resultados Arquiteturas}            
%\resizebox{\linewidth}{!}{\begin{tabular}{lrrlrrrrrrr}
\toprule
name & rmse & abs erro & erro comp & r2 score & mape score & alloc missing & alloc surplus & optimal percentage & better allocation & beter percentage \\
\midrule
VanillaDense & 1200.20 & 10073188.28 & False & -41.27 & 192.50 & 487.24 & 10072701.04 & 1.13 & 0.77 & 1.17 \\
StackedCNN & 453.12 & 3717806.99 & True & -5.03 & 82.67 & 52038.16 & 3665768.83 & 35.04 & 34.48 & 38.34 \\
UNET & 183.58 & 1166186.91 & True & 0.01 & 17.02 & 687071.63 & 479115.27 & 60.39 & 60.39 & 85.47 \\
VanillaCNN & 174.96 & 1124205.72 & True & 0.10 & 16.45 & 614355.90 & 509849.82 & 61.79 & 61.79 & 86.18 \\
\bottomrule
\end{tabular}
}
\end{table}

\subsubsection{Épocas\label{se:epocas_res}}

\subsection{Funções de Perda (Loss) \label{se:loss_exp}}

As funções de perda foram estudadas com duas arquiteturas diferentes, de modo a conseguir ter uma melhor noção do impacto das mesmas. \\

\begin{table}[H]
\caption{Resultados Funções de perda}        
%\resizebox{\linewidth}{!}{\begin{tabular}{lrrlrrrrrrr}
\toprule
name & rmse & abs erro & erro comp & r2 score & mape score & alloc missing & alloc surplus & optimal percentage & better allocation & beter percentage \\
\midrule
UNETmse & 180.84 & 1269085.40 & True & 0.04 & 23.32 & 521000.39 & 748085.00 & 68.78 & 68.78 & 87.31 \\
StackedCNNmae & 198.29 & 1083872.29 & True & -0.15 & 5.93 & 937786.28 & 146086.01 & 43.65 & 43.65 & 83.32 \\
StackedCNNmape & 232.60 & 1257578.80 & True & -0.59 & 1.01 & 1252220.35 & 5358.46 & 19.15 & 19.15 & 80.60 \\
StackedCNNmse & 176.71 & 1103152.68 & True & 0.08 & 14.81 & 673158.53 & 429994.14 & 58.75 & 58.75 & 85.68 \\
StackedCNNwl & 2002.81 & 17092114.23 & False & -116.72 & 320.21 & 0.00 & 17092114.23 & 0.05 & 0.00 & 0.05 \\
UNETmae & 195.49 & 1083463.60 & True & -0.12 & 7.14 & 887646.74 & 195816.87 & 45.95 & 45.95 & 83.70 \\
StackedCNNmsle & 225.42 & 1761315.40 & True & -0.49 & 40.41 & 320813.27 & 1440502.13 & 79.46 & 79.46 & 89.93 \\
StackedCNNmsde & 612.31 & 4978280.05 & False & -10.00 & 81.86 & 3325165.78 & 1653114.27 & 18.27 & 18.06 & 30.14 \\
UNETmsle & 211.40 & 1142731.21 & True & -0.31 & 4.70 & 1048097.40 & 94633.80 & 39.64 & 39.64 & 82.11 \\
\bottomrule
\end{tabular}
}
\end{table}

A nivel de percentagem de modelo melhor, elas esão todas bastante renhidas, mas ha uma claro vantagem quando vemos a percentagem de melhor alocação ou de alocação optima.\\
Sendo que a perda que avança é a de MSLE (Mean Square Log Error).\\
Esta perda é tendicionalmente usada para distribuições exponencias, e onde temos bastantes outliers, que devem ser considerados. O que é o caso no nosso problema.\\
Esta experiência valida a observação feito na análise estatistica sobre o mesmo.\\

\subsection{Hiperparametrização\label{se:hiper}}

\subsubsection{Activação\label{se:activ}}

\begin{table}[H]
\caption{Resultados Ativação}    
%\resizebox{\linewidth}{!}{\begin{tabular}{lrrlrrrrrrr}
\toprule
name & rmse & abs erro & erro comp & r2 score & mape score & alloc missing & alloc surplus & optimal percentage & better allocation & beter percentage \\
\midrule
StackedCNN relu relu & 207.66 & 1113406.16 & True & -0.27 & 3.90 & 1028475.81 & 84930.35 & 38.64 & 38.64 & 82.31 \\
StackedCNN tanh tanh & 236.24 & 1289434.20 & True & -0.64 & 0.85 & 1288660.90 & 773.30 & 11.47 & 11.47 & 80.40 \\
StackedCNN relu tanh & 236.24 & 1289434.20 & True & -0.64 & 0.85 & 1288660.90 & 773.30 & 11.47 & 11.47 & 80.40 \\
StackedCNN relu softsign & 236.31 & 1290138.00 & True & -0.64 & 0.86 & 1289376.80 & 761.20 & 11.29 & 11.29 & 80.40 \\
StackedCNN linear tanh & 236.24 & 1289434.20 & True & -0.64 & 0.85 & 1288660.90 & 773.30 & 11.47 & 11.47 & 80.40 \\
StackedCNN softsign linear & 1112.70 & 8936608.01 & False & -35.34 & 178.94 & 8185.60 & 8928422.41 & 7.99 & 7.50 & 8.60 \\
StackedCNN softsign softsign & 236.24 & 1289436.03 & True & -0.64 & 0.85 & 1288662.99 & 773.04 & 11.00 & 11.00 & 80.39 \\
StackedCNN softplus elu & 294.00 & 2287913.46 & True & -1.54 & 52.62 & 273134.72 & 2014778.74 & 79.27 & 79.25 & 88.22 \\
StackedCNN softplus softplus & 393.20 & 3191326.78 & True & -3.54 & 72.87 & 84140.16 & 3107186.62 & 69.22 & 69.15 & 71.84 \\
StackedCNN relu selu & 262.76 & 1986551.87 & True & -1.03 & 43.04 & 261245.83 & 1725306.04 & 77.31 & 77.17 & 86.49 \\
StackedCNN linear elu & 297.39 & 2368520.21 & True & -1.60 & 55.86 & 177934.14 & 2190586.07 & 87.47 & 87.47 & 92.53 \\
StackedCNN softsign tanh & 236.24 & 1289434.20 & True & -0.64 & 0.85 & 1288660.90 & 773.30 & 11.47 & 11.47 & 80.40 \\
StackedCNN softplus exponential & 461.92 & 3782992.35 & True & -5.26 & 82.96 & 50599.52 & 3732392.83 & 32.55 & 31.88 & 35.55 \\
StackedCNN softplus selu & 287.86 & 2249601.96 & True & -1.43 & 53.57 & 218301.68 & 2031300.27 & 84.44 & 84.44 & 91.23 \\
StackedCNN softplus softsign & 236.36 & 1291380.19 & True & -0.64 & 0.90 & 1290725.89 & 654.30 & 9.39 & 9.39 & 80.40 \\
StackedCNN softplus linear & 208.37 & 1113888.40 & True & -0.27 & 3.74 & 1032928.56 & 80959.84 & 38.23 & 38.23 & 82.30 \\
StackedCNN softsign relu & 203.70 & 1100693.28 & True & -0.22 & 4.92 & 986235.38 & 114457.90 & 41.64 & 41.64 & 82.69 \\
StackedCNN softsign exponential & 508.90 & 4187396.73 & False & -6.60 & 91.15 & 32147.81 & 4155248.92 & 25.81 & 24.91 & 28.09 \\
StackedCNN linear softsign & 236.24 & 1289434.20 & True & -0.64 & 0.85 & 1288660.90 & 773.30 & 11.47 & 11.47 & 80.40 \\
StackedCNN relu exponential & 542.88 & 4482255.28 & False & -7.65 & 96.23 & 31472.22 & 4450783.06 & 22.04 & 21.01 & 24.00 \\
StackedCNN softsign selu & 2053.61 & 17501165.80 & False & -122.77 & 323.70 & 73467.70 & 17427698.09 & 0.01 & 0.00 & 2.98 \\
StackedCNN tanh softplus & 203.53 & 1097751.79 & True & -0.22 & 4.71 & 986626.52 & 111125.27 & 41.31 & 41.31 & 82.73 \\
StackedCNN tanh linear & 543.57 & 4512608.50 & False & -7.67 & 97.01 & 27030.50 & 4485578.00 & 20.81 & 19.78 & 22.73 \\
StackedCNN relu elu & 481.58 & 3739379.02 & True & -5.81 & 83.92 & 84681.14 & 3654697.88 & 50.13 & 49.65 & 53.43 \\
StackedCNN linear exponential & 457.52 & 3752241.97 & True & -5.14 & 83.28 & 52642.15 & 3699599.81 & 29.83 & 29.21 & 33.15 \\
StackedCNN relu softplus & 187.41 & 1193440.68 & True & -0.03 & 17.36 & 716295.64 & 477145.04 & 59.08 & 59.08 & 85.03 \\
StackedCNN linear selu & 559.84 & 4606574.05 & False & -8.20 & 98.93 & 35641.95 & 4570932.10 & 22.53 & 21.82 & 24.82 \\
StackedCNN softplus relu & 380.03 & 3059764.47 & True & -3.24 & 69.45 & 99012.20 & 2960752.27 & 70.70 & 70.58 & 74.00 \\
StackedCNN linear relu & 614.15 & 4962488.83 & False & -10.07 & 100.79 & 66478.05 & 4896010.78 & 19.20 & 18.28 & 23.60 \\
StackedCNN linear softplus & 321.88 & 2577000.96 & True & -2.04 & 59.88 & 149354.46 & 2427646.50 & 88.59 & 88.59 & 92.51 \\
StackedCNN tanh relu & 548.29 & 4556413.79 & False & -7.82 & 97.85 & 26362.43 & 4530051.37 & 20.66 & 19.65 & 22.52 \\
StackedCNN softsign softplus & 900.71 & 6351850.55 & False & -22.81 & 117.73 & 22655.51 & 6329195.04 & 17.30 & 16.55 & 18.83 \\
StackedCNN linear linear & 238.22 & 1866499.25 & True & -0.67 & 43.41 & 292863.60 & 1573635.64 & 81.02 & 81.02 & 90.41 \\
StackedCNN tanh softsign & 236.24 & 1289436.20 & True & -0.64 & 0.85 & 1288663.18 & 773.02 & 11.00 & 11.00 & 80.39 \\
StackedCNN softsign elu & 205.05 & 1105893.27 & True & -0.23 & 4.88 & 995748.76 & 110144.51 & 41.17 & 41.17 & 82.57 \\
StackedCNN relu linear & 431.38 & 3464311.65 & True & -4.46 & 76.14 & 133075.54 & 3331236.11 & 45.08 & 44.54 & 51.57 \\
StackedCNN softplus tanh & 236.24 & 1289434.20 & True & -0.64 & 0.85 & 1288660.90 & 773.30 & 11.47 & 11.47 & 80.40 \\
\bottomrule
\end{tabular}
}
\end{table}

\subsubsection{Optimizaçao\label{se:opt}}

\begin{table}[H]
\caption{Resultados Optimizadores}    
%\resizebox{\linewidth}{!}{\begin{tabular}{lrrlrrrrrrr}
\toprule
name & rmse & abs erro & erro comp & r2 score & mape score & alloc missing & alloc surplus & optimal percentage & better allocation & beter percentage \\
\midrule
StackedCNN Adadelta & 1946.60 & 16304588.30 & False & -110.20 & 298.49 & 121783.10 & 16182805.20 & 0.37 & 0.37 & 6.57 \\
StackedCNN Adafactor & 209.16 & 1127287.01 & True & -0.28 & 4.25 & 1036653.47 & 90633.54 & 39.16 & 39.16 & 82.23 \\
StackedCNN Adagrad & 2281.10 & 19250481.97 & False & -151.71 & 354.33 & 50132.10 & 19200349.87 & 0.29 & 0.27 & 3.51 \\
StackedCNN Adam & 507.51 & 4103373.73 & False & -6.56 & 90.96 & 44467.79 & 4058905.95 & 36.50 & 35.86 & 38.94 \\
StackedCNN AdamW & 208.26 & 1114855.55 & True & -0.27 & 3.82 & 1033146.63 & 81708.92 & 38.60 & 38.60 & 82.34 \\
StackedCNN Adamax & 1085.65 & 8916100.25 & False & -33.59 & 177.57 & 64682.73 & 8851417.52 & 3.48 & 2.92 & 6.39 \\
StackedCNN Ftrl & 242.89 & 1847009.05 & True & -0.73 & 39.01 & 341650.54 & 1505358.52 & 77.52 & 77.52 & 89.31 \\
StackedCNN Nadam & 534.57 & 4388148.16 & False & -7.39 & 92.41 & 35224.69 & 4352923.47 & 22.85 & 21.81 & 25.02 \\
StackedCNN RMSprop & 501.82 & 3995286.60 & False & -6.39 & 87.47 & 165111.98 & 3830174.63 & 30.90 & 30.17 & 38.94 \\
StackedCNN SGD & 207.90 & 1125396.48 & True & -0.27 & 4.88 & 1021725.80 & 103670.68 & 40.59 & 40.59 & 82.27 \\
\bottomrule
\end{tabular}
}
\end{table}


\subsection{Janelas Temporais}

\begin{table}[H]
\caption{Resultados Janelas Temporais}
%\resizebox{\linewidth}{!}{\begin{tabular}{lrrlrrrrrrr}
\toprule
name & rmse & abs erro & erro comp & r2 score & mape score & alloc missing & alloc surplus & optimal percentage & better allocation & beter percentage \\
\midrule
StackedCNN 48X 24Y & 479.11 & 3989558.86 & False & -5.75 & 86.62 & 43807.71 & 3945751.15 & 29.86 & 29.19 & 32.72 \\
StackedCNN 98X 1Y & 445.28 & 3671816.03 & True & -4.83 & 81.15 & 54078.28 & 3617737.75 & 37.10 & 36.50 & 40.30 \\
StackedCNN 98X 24Y & 1167.21 & 9770108.13 & False & -39.09 & 191.54 & 1578.73 & 9768529.40 & 2.25 & 1.83 & 2.41 \\
StackedCNN 24X 1Y & 209.47 & 1149047.56 & True & -0.29 & 4.43 & 1055993.70 & 93053.86 & 39.55 & 39.55 & 82.21 \\
StackedCNN 98X 4Y & 495.81 & 4120357.48 & False & -6.23 & 88.95 & 36888.60 & 4083468.88 & 27.15 & 26.22 & 29.58 \\
StackedCNN 24X 4Y & 453.22 & 3779420.50 & True & -5.04 & 82.35 & 52131.12 & 3727289.38 & 33.55 & 32.94 & 36.88 \\
StackedCNN 24X 12Y & 450.27 & 3753076.87 & True & -4.96 & 82.29 & 53778.38 & 3699298.49 & 34.86 & 34.35 & 38.25 \\
StackedCNN 24X 24Y & 468.78 & 3907629.32 & True & -5.46 & 84.62 & 48042.45 & 3859586.87 & 26.35 & 25.65 & 29.52 \\
StackedCNN 168X 4Y & 247.11 & 1941513.71 & True & -0.79 & 45.26 & 272577.02 & 1668936.69 & 82.07 & 82.07 & 90.72 \\
StackedCNN 48X 1Y & 510.34 & 4249569.03 & False & -6.65 & 91.65 & 35151.96 & 4214417.07 & 23.58 & 22.63 & 26.05 \\
StackedCNN 24X 8Y & 209.71 & 1150163.22 & True & -0.29 & 4.41 & 1058328.49 & 91834.73 & 39.32 & 39.32 & 82.24 \\
StackedCNN 48X 4Y & 443.47 & 3672502.43 & True & -4.78 & 80.47 & 56035.63 & 3616466.80 & 33.50 & 33.02 & 36.99 \\
StackedCNN 48X 12Y & 451.66 & 3755277.12 & True & -4.99 & 82.42 & 51133.01 & 3704144.12 & 35.83 & 35.34 & 39.13 \\
StackedCNN 48X 8Y & 425.76 & 3528446.38 & True & -4.33 & 78.29 & 63798.09 & 3464648.29 & 35.06 & 34.73 & 38.87 \\
StackedCNN 98X 8Y & 401.18 & 3287580.95 & True & -3.73 & 74.38 & 76725.24 & 3210855.71 & 51.78 & 51.65 & 55.36 \\
StackedCNN 98X 12Y & 622.31 & 5179835.43 & False & -10.41 & 108.86 & 19985.94 & 5159849.49 & 17.78 & 16.78 & 19.33 \\
StackedCNN 168X 24Y & 352.28 & 2757367.10 & True & -2.64 & 65.61 & 155940.39 & 2601426.71 & 70.78 & 70.65 & 75.40 \\
StackedCNN 168X 8Y & 277.45 & 2189089.58 & True & -1.26 & 51.96 & 226599.69 & 1962489.89 & 84.47 & 84.47 & 91.59 \\
StackedCNN 168X 12Y & 404.20 & 3219562.53 & True & -3.79 & 70.85 & 90454.58 & 3129107.95 & 67.74 & 67.30 & 71.01 \\
StackedCNN 168X 1Y & 568.14 & 4694571.86 & False & -8.47 & 100.18 & 23253.65 & 4671318.21 & 20.47 & 19.49 & 22.15 \\
\bottomrule
\end{tabular}
}        
\end{table}

\subsection{Classificação}


\begin{table}[H]
\caption{Resultados Janelas Temporais}
%\resizebox{\linewidth}{!}{\begin{tabular}{lrrlrrrrrrrrr}
\toprule
name & rmse & abs erro & erro comp & r2 score & mape score & alloc missing & alloc surplus & optimal percentage & better allocation & beter percentage & acc & acc aloc \\
\midrule
StackedCNNClusterLinear 4 arch & 207.47 & 1111659.96 & True & -0.26 & 3.81 & 1027024.21 & 84635.75 & 38.66 & 38.66 & 82.34 & 0.40 & 0.29 \\
StackedCNNClusterLinear 4 interpr & 207.45 & 1111808.59 & True & -0.26 & 3.84 & 1027318.24 & 84490.36 & 38.77 & 38.77 & 82.38 & NaN & NaN \\
StackedCNNClusterLinear 6 arch & 208.03 & 1113342.39 & True & -0.27 & 3.72 & 1032371.26 & 80971.13 & 38.48 & 38.48 & 82.28 & 0.33 & 0.10 \\
StackedCNNClusterLinear 6 interpr & 208.23 & 1114478.16 & True & -0.27 & 3.72 & 1034020.09 & 80458.06 & 38.55 & 38.55 & 82.26 & NaN & NaN \\
StackedCNNClusters 4 arch & 206.56 & 1106934.65 & True & -0.25 & 4.11 & 1017487.36 & 89447.30 & 39.57 & 39.57 & 82.43 & 0.40 & 0.29 \\
StackedCNNClusters 6 arch & 206.94 & 1109766.69 & True & -0.26 & 4.12 & 1019180.77 & 90585.93 & 39.49 & 39.49 & 82.45 & 0.33 & 0.10 \\
\bottomrule
\end{tabular}
}        
\end{table}


\subsection{Pesos}



\begin{table}[H]
\caption{Resultados Janelas Temporais}
%\resizebox{\linewidth}{!}{\begin{tabular}{lrrlrrrrrrr}
\toprule
name & rmse & abs erro & erro comp & r2 score & mape score & alloc missing & alloc surplus & optimal percentage & better allocation & beter percentage \\
\midrule
StackedCNN delta mean & 177.29 & 1204388.33 & True & 0.08 & 19.99 & 508061.06 & 696327.28 & 66.69 & 66.69 & 87.72 \\
StackedCNN no weight & 178.20 & 1100714.22 & True & 0.07 & 14.09 & 697157.86 & 403556.35 & 57.49 & 57.49 & 85.36 \\
VanillaCNN delta mean & 4061.70 & 34669933.86 & False & -483.15 & 634.39 & 0.00 & 34669933.86 & 0.00 & 0.00 & 0.00 \\
VanillaCNN no weight & 1146.52 & 9109107.48 & False & -37.58 & 185.61 & 25950.43 & 9083157.05 & 11.90 & 11.55 & 12.81 \\
\bottomrule
\end{tabular}
}        
\end{table}




Aqui devem constar gráficos e sua análise crítica e ligação com a secção 2 da revisão bibliográfica no sentido de comparar valores e discutir diferenças, por exemplo. A legenda dos gráficos deve seguir a das figuras, isto é, porque também são figuras.\\





% Para introduzir figuras
\begin{figure}[H]
\centering
\includegraphics[width=150pt, keepaspectratio]{Imagens/FigA.png}
% legenda da figura por baixo
\caption{Exemplo de como considerar um gráfico}
\label{fig:graficoecampl} % referência única da figura
\end{figure}

