\chapter{Arquitecturas de Modelos}

Grande parte da literatura sobre previsões em modelos de apredizagem apresenta as mesmas arquiteturas, sendo que são depois aprimoradas consoate os dados e o problema.

Apresento aqui as aquiteturas mais usadas em previsões, como tambem algumas usadas noutros ramos tentado prever a compatibilidade neste problema 

\section{Blocos  \label{se:blocos}}

\subsection{subsection Conv (depth 2)}

Usada tambem as ideias de attention, residual e o que eu chamei broad
	
\subsection{subsection CNN (depth 2)}

\subsection{subsection LSTM (depth 2)}

\section{CNN \label{se:dados_crus}}

dizer de onde veio

package para sacar

\section{LSTM  \label{se:dados_estudo}}
\subsection{subsection title (depth 2)}
\subsubsection{subsubsection title (depth 3)}
distribuiçao

clustering
	deu cerca de 15
	5 seria um bom numero de clusters pelos estudos
	mas segundo o tipo de problema 3 + 1(zero) foi o ideal

autocorrelation
	periodissidade
	feature coleration
\section{Arquitetura  \label{se:dados_tratamento}}

\subsection{subsection title (depth 2)}


\section{Vanilla  \label{se:dados_tratamento}}

\section{Stacked  \label{se:dados_tratamento}}


\section{MultiHead  \label{se:dados_tratamento}}
\section{MultiTail  \label{se:dados_tratamento}}


\section{UNET  \label{se:dados_tratamento}}

Usado normalmente em modelçao de imagem. 

\section{Considerações adicionais  \label{se:dados_plus}}

Aqui e dizer que os modelos utilizados para teste sao as combinacoes deste blocos nestas aquiteturas.

