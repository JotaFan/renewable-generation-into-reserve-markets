\chapter{Revisão bibliográfica}

A análise de séries temporais é um tema recorrente em pesquisa. Especialmente para previsões.\\
Desde as previsões para mercados de acções\cite{Bhandari2022}, fenomenos meteorologicos\cite{Wang2019}, e especialmente mercados energéticos, onde se quer ter em consideração o impacto das gerações mais voláteis. \\
As energias renováveis, devido à sua natureza, são as produçoes mais voláteis, logo alvo de estudo ideal para estas tecnologias\cite{Lu2015}
, energia eolica\cite{Sun2022}, energia solar\cite{Rajasundrapandiyanleebanon2023}, aplicabilidade dos vários sistemas\cite{Ahmad2020}, procura\cite{Antonopoulos2020}.
Sendo que cada problema já apresenta arquiteturas e soluções diferentes, como a geração de energia fotovoltaica em casas pode ser melhor prevista com LSTM\cite{Costa2022} mas tambem com uso de SVM\cite{Meenal2018}
As várias faces destas tecnologias estão optimamente apresentadas em\cite{Benti2023}



Para o estudo de previsões de séries temporais chega a ser o caso se pesquisar primeiramente com \textit{deep learning}, antes de procurar outras soluções.\\
Em \cite{Elsayed} é visto o impacto dessa decisão, e se realmente compensa emergir em \text{machine learning}. O trabalho conclui que modelos simples, com alguma engenharia de atributos inteligente, consegue competir, ou mesmo passar as qualidades de redes neuronais profundas. \\
Esta conclusão mostra também que por vezes a procura por modelos mais complexos não compensa, e que cada problema/dataset deve ter a sua própria investigação e conclusão, consoante a quantidade/qualidade de recursos disponiveis. \\


Os sistemas de reservas de frequências dos mercado espanhol já foram alvo de estudo predictivo, através de redes neuronais profundas\cite{miota2023}.\\
Este trabalho propõe prever o custo da alocação e faz usando um elavado numero de atributos disponiveis pela ESIOS, 32 variaveis. Estes modelos tiveram de metrica MASE cerca de 64\%, o que não foi considerado um bom valor.\\
Este 

reserve allocation forecasting \\
A alocação de reservas 
ESTUDOS EM MERCADOS DE RESERVAS \cite{Rassid2017}
% \cite{Energy and reserve markets: interdependency in electricity systems with a high share of renewables}

energy forecasting with ML \\
Previsões são um dos problemas mais comuns a ser tratados com aprendizagem automática. \cite{} \\
No ambito de energias vários trabalhos vieram mostrar que o uso de \textit{machine learning} para previsões energéticas tem aplicabilidade \cite{Stassen} e muitos casos resultados melhores do que usados na industria currente. \cite{Ahmad2020} \cite{Antonopoulos2020} \\  
Como muitos trabalhos apresentados em \cite{Benti2023}, o uso destas técnicas está a crescer e a produzir frutos. Como concluido nesse trabalho as várias arquitecturas e modelos comuns de ML já foram aplicados em energia, especial nas areas de consumo e produção. \\



deep neural network on forecasting\\
O uso de redes neuronais profundas para previsões é tambem por si proprio alvo de estudo
\cite{miota2023}






