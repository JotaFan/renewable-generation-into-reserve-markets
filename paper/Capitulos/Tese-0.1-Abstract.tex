\chapter{Abstract}
\justifying

The growing penetration of time-varying renewable energy sources (\gls{vRES}) into the electricity system, such as solar and wind, is significantly transforming the electricity markets due to their intermittent and unpredictable nature. This unpredictability makes forecasting energy production and consumption more challenging, especially as electricity markets close between 1 and 37 hours before the actual delivery of energy, which can result in discrepancies between the energy contracted and the energy actually produced or consumed. Maintaining the balance between supply and demand in real time is vital for the security and stability of the grid, a task that falls mainly to \gls{TSO}.\par
TSOs use power reserve markets, where they symmetrically purchase upward and downward secondary power based on demand forecasts for subsequent hours. However, this symmetrical approach can be ineffective in the face of renewable energy fluctuations, leading to the need for more dynamic and precise adjustments.\par
In this context, this work first proposes a study of the hourly variable parameter of the Portuguese \gls{TSO} formula for forecasting the required reserve ($\rho$), where using historical hourly data for the period 2010 to 2019, the hourly $\rho$ that results in the lowest average error in the forecast is calculated, where results with an error of less than 5\% are obtained.\par
This work then proposes a model based on machine learning techniques to dynamically calculate secondary power reserves, both upward and downward, integrating scheduled day-ahead dispatches, \gls{vRES} production forecasts, forecast demand and other operational characteristics.\par
Using open operational data from the Spanish \gls{TSO}, the model was trained with data from 2014 to 2023, and validated with reference data from 2019 to 2022. The proposed methodology demonstrates a significant improvement in the utilization of secondary power reserves, with an increase of approximately 43\% in the efficiency of upstream reserves and around 36\% in downstream reserves. This advance contributes to more efficient and balanced management of the electricity system, especially in scenarios with high \gls{vRES} penetration.\par


\vspace{0.5cm} %adiciona um espaço de 0.5cm entre o texto e as palavras chave.

\keywordsP{reserve systems,energy markets, neural networks, forecast}
%notice that # requires a \ so that latex correctly interprets it as a special character. This also happens with % for example.