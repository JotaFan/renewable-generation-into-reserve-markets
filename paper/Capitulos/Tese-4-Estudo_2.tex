\chapter{Estudo 2: Dimensionamento dinâmico da potência alocada na reserva secundária}

\section{Dados Utilizados\label{se:dadosestudo}}

Os dados em estudo são do mercado energético espanol, retirados do site da \href{https://www.esios.ree.es/es}{ESIOS}.

\resizebox{\linewidth}{!}{\csvautotabular{tabelas/indicators_metadata.csv}}



\subsection{Aquisição dos Dados}

No ambito da automatização destes dados foi modificado o repositorio \href{https://github.com/SanPen/ESIOS}{ESIOS} para ser usado como uma biblioteca de python, aberta, em pypi.\\
Sendo uma ferramenta mais facilmente acessivel para a extrair dados do mercado espanhol, \href{https://pypi.org/project/pyesios/}{pyesios}. \\
No âmbito de automatizar o processo, foram feitas contribuições a esta ferramenta para tornar mais acessível, e uma ferramenta aberta de python\\


\newpage
\thispagestyle{plain}
\input{Capitulos/Tese-4-Estudo_2A} \label{se:dadoscrus}


\newpage
\thispagestyle{plain}
\input{Capitulos/Tese-4-Estudo_2B} \label{se:tratamentodados}

\subsection{Dados de treino}

Após o tratamento apresentado as estatísticas gerais dos dados usados para treinar o modelo são:

\begin{table}[H]
    \caption{Dados de Treino}    
    \resizebox{\linewidth}{!}{\begin{tabular}{lrrrr}
\toprule
 & média & desvio padrão & min & max \\
\midrule
DownwardUsedSecondaryReserveEnergy & 168.18 & 199.23 & 0.00 & 1721.40 \\
SecondaryReserveAllocationAUpward & 665.98 & 150.88 & 399.00 & 958.00 \\
SecondaryReserveAllocationADownward & 554.50 & 131.06 & 312.00 & 956.00 \\
UpwardUsedSecondaryReserveEnergy & 160.82 & 193.09 & 0.00 & 1654.80 \\
WindD+1DailyForecast & 5881.14 & 3480.52 & 66.13 & 20879.30 \\
PhotovoltaicD+1DailyForecast & 1676.31 & 2745.51 & 0.00 & 14925.30 \\
DemandD+1DailyForecast & 27933.38 & 4488.71 & 14170.00 & 41799.66 \\
TotalBaseDailyOperatingSchedulePBFGeneration & 27250.40 & 4608.74 & 13470.50 & 42707.60 \\
BaseDailyOperatingSchedulePBFSolarPV & 1737.79 & 2850.91 & 0.00 & 16358.90 \\
BaseDailyOperatingSchedulePBFWind & 6588.28 & 3637.80 & 308.60 & 21619.60 \\
BaseDailyOperatingShedulePBFTotalBalanceInterconnections & 266.26 & 2169.01 & -7817.00 & 6858.50 \\
\bottomrule
\end{tabular}
}
    \end{table}
