\subsubsection{Estatisticos}

Em estatística conseguimos encontrar vários métodos de estudo de séries temporais. Estes métodos são normalmente usados como primeira abordagem para fazer previsões.\par
Estes modelos podem ser \gls{AR}, onde vão fazer previsões baseados num número \textit{(p)} de dados anteriores. Estes modelos são construídos com a noção de que um valor é linearmente dependente de \textit{p} valores anteriores numa série temporal.\par

\begin{alignat*}{2} 
    & X_{t} : \text{Valor no } t \text{ a prever.} &\quad& p : \text{O número observaçoes anteriores.} \\
    & \varphi_{i} : \text{Coeficiente na observação } i. &\quad& q : \text{O número observaçoes anteriores.} \\
    & \epsilon_{i} : \text{Erro na observação } i. &\quad& \theta_{i} : \text{Coeficiente na observação } i \text{.} \\ 
    & \mu : \text{Média dos valores } X \text{.} 
\end{alignat*}

\bigskip
\gls{AR} \\

\begin{equation} \label{eq:ar} 
    X_{t} = \sum_{i=1}^{p}\varphi_{i} X_{t-i} 
\end{equation}
\smallskip

Outra família destes modelos são os de \gls{MA}, onde a média de um número de observações \textit{(q)} em conjunto com os erros \textit{($\epsilon$)} e os coeficientes \textit{($\theta$)} é usada para prever os valores seguintes.\par
\bigskip
\gls{MA} \\

\begin{equation} \label{eq:ma} 
    X_{t} = \mu + \sum_{i=1}^{q}(\theta_{i} \epsilon_{t-i}) + \epsilon_{t}
\end{equation}
\smallskip
Estes dois tipos de modelos podem ser juntos criando os modelos \gls{ARMA}. que junta as capacidades dos modelos anteriores.\par

\bigskip
\gls{ARMA} \\

\begin{equation} \label{eq:arma} 
    X_{t} = \sum_{i=1}^{p}\varphi_{i} X_{t-i}  + \mu + \sum_{i=1}^{q}(\theta_{i} \epsilon_{t-i}) + \epsilon_{t}
\end{equation}
\smallskip

Existem mais modelos de previsão estatística baseados nestes com algumas variações, mas para este trabalho, e apenas como ponto de comparação às redes neuronais, ficamos apenas por estes.\par
As variáveis em estudo por tipo de modelo foram retiradas das autocorrelações temporais usando os métodos de sugestão da ferramenta \hyperref[se:muaddib]{MuadDib}:\\


\begin{table}[h] \centering
\begin{tabular}{lrrrr}
    \toprule
     & p & q \\
    \midrule
    \gls{AR} & 1/ 2 / 23 / 24 / 25 / 48 / 144 / 168 / 192 / 336 & NA \\
    \gls{MA} & NA & 1 / 24 \\
    \gls{ARMA} & 1 & 1 \\
    \bottomrule
    \end{tabular}
    \label{tab:varsstats} 
    \caption{Variáveis de estudo dos modelos AR/MA}
\end{table}


Todos estes modelos foram testados usando o software disponível na package de python \href{https://www.statsmodels.org/stable/index.html}{statsmodel}, com a classe \href{https://www.statsmodels.org/stable/generated/statsmodels.tsa.arima.model.ARIMA.html}{ARIMA}.
