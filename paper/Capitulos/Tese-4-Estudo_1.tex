\chapter{Estudo 1: Estimativa do parâmetro $\rho$ da fórmula da REN}



Para responder a primeira questão estudou-se o comportamento do parâmetro p na equação publicada pela REN para a Banda de Regulação Secundária a Subir:

\begin{equation} \label{eq:BRREN} 
    BR = \rho \times \sqrt{a \times  L_{max} + b^{2}} - b 
\end{equation}
onde:
\begin{itemize}
  \item $BR$: Banda de Reserva  de regulação secundária necessária (MW).
  \item $\rho$: Paramêtro horário.
  \item $a$ e $b$: Coeficientes empiricos, $a$=10MW e $b$=150MW .
  \item $L_{max}$: Pico máximo de consumo (MW).
\end{itemize}

Aqui queremos descobrir qual o valor do parâmetro $\rho$ (por hora do dia) que melhor nos descreve os dados reais. Para isso estudamos os valores reais usados para a Banda de Reserva, os valores resultantes da proposta de $\rho$  em \cite{Carneiro2016} e os valores resultados do estudo aqui proposto. Aproximar o parâmetro $\rho$  utilizando os dados históricos. \\
Todos os dados necessários são disponibilizados pelo operador do sistema no \href{https://mercado.ren.pt/PT/Electr}{site da REN}, com exceção do consumo máximo expectável. Este parâmetro é então substituído pelo consumo real, como uma aproximação à formulação indicada previamente.\\
Os dados estudados contêm entradas horárias desde 2010 até ao fim de 2018. Com as seguintes variáveis:\\

\begin{table}[H] \centering \caption{Dados REN} \begin{tabular}{ll}
\toprule
Variável & Unidades \\
\midrule
BANDA SUBIR & MW \\
BANDA DESCER & MW \\
Consumo real & MW \\
Consumo Máximo ENTSO-E & MW \\
\bottomrule
\end{tabular}
 \end{table}

Na equação \ref{eq:BRREN} BR equivale à soma de BANDA SUBIR e BANDA DESCER, onde aqui é sempre considerado que Banda a subir são $\frac{2}{3}$ da Banda de Reserva total e a Banda a descer é o restante $\frac{1}{3}$. \\
Aqui iremos aplicar o mapa de parâmetro $\rho$ apresentado em \cite{Carneiro2016} na formula \ref{eq:BRREN} para o cálculo da Banda de Reserva Carneiro2016 como benchmark. \\

\begin{table}[H] \centering \caption{Valores de $\rho$ apresentado em \cite{Carneiro2016}} \begin{tabular}{ll}
\toprule
Hora & $\rho$ \\
\midrule
1/2/8/9/24 & 1,6 \\
3/7/10/11/19/20 & 1,4 \\
4 & 1,3 \\
5/6/12/13/14/15/16/17/18/21/22/23 & 1,2 \\
\bottomrule
\end{tabular}
 \end{table}


Por outro lado, usando os dados de consumo real, calculamos o $\rho$ ideal para cada entrada, de onde estudamos a melhor normalização dos mesmos para cada hora. \\
O cálculo deste $\rho_{proposto}$ é apenas a utilização da fórmula \ref{eq:BRREN} mas em função de $\rho$ : \\

\begin{equation} \label{eq:rhoproposed} 
    \rho  = \frac{(BR + b)}{\sqrt{a \times L_max + b^{2}}}
\end{equation}


Arredondando o $\rho_{proposto}$ a uma casa decimal, podemos verificar que o histograma das diferentes propostas difere bastante. Sendo que esta apresenta uma curva de distribuição normal. \\


\begin{figure}[H]
    \centering
    \includegraphics[width=\textwidth]{plots/Histograma_parametro_p.png}
    \caption{Histograma $\rho$}
  \end{figure}


Olhando as distribuições por hora: \\

\begin{figure}[H]
    \centering
    \includegraphics[width=\textwidth]{plots/Valor_do_parametro_p_hora.png}
    \caption{Valor do paramêtro $\rho$ (hora)}
  \end{figure}

O $\rho_{proposto}$ apresenta um grande variabilidade em todas as horas, embora de notar que em todas tem um maior peso perto da mediana. O $\rho$ de comparação embora sempre dentro da distribuição note-se que cai quase sempre em zonas com pouco peso nestes dados históricos. \\
Calculamos $\rho$ possiveis para proposta final usando as seguintes aproximações: média, mediana, e média ponderada ao consumo, e à banda. \\

As distribuições por hora são as seguintes:

\begin{figure}[H]
    \centering
    \includegraphics[width=\textwidth]{plots/Comparação_de_p_propostos_por_hora.png}
    \caption{Histograma $\rho$}
\end{figure}

Todas seguem um percurso semelhante ao longo do dia, o qual também pode ser extrapolado para Carneiro2016. A média e mediana destacam-se seguindo muito parecidas, enquanto que as ponderadas também parecidas entre elas são bastante mais discretas. \\
Para a escolha da normalização deste parâmetro à Hora, estudou-se o erro entre a Banda Reserva calculada através das normalizações e a Banda Reserva disponível nos dados. \\


\begin{figure}[H]
    \centering
    \includegraphics[width=\textwidth]{plots/Comparação_das_metricas_de_Erro.png}
    \caption{Histograma $\rho$}
\end{figure}


\begin{table}[H]
    \caption{Erros de Banda de Reserva por método de normalização $rho$}    
    \resizebox{\linewidth}{!}{\begin{tabular}{lrrrr}
\toprule
 & MAE (MW) & RMSE (MW) & MedianAE (MW) & MAPE (\%)\\
Normalização &  &  &  &  \\
\midrule
Carneiro2016 & 53.07 & 66.54 & 44.53 & 18.70 \\
média & 30.94 & 39.19 & 25.38 & 11.58 \\
mediana & 30.85 & 39.20 & 25.17 & 11.51 \\
média ponderada banda & 32.15 & 40.61 & 26.45 & 12.19 \\
média ponderada consumo & 31.54 & 39.91 & 26.20 & 11.73 \\
\bottomrule
\end{tabular}
}
    \end{table}


A normalização com erros mais baixos é a mediana. Com um erro médio (de todo o histórico) para o consumo real de 11.51\% o que comparando com o benchmark de 18.70\% é uma melhoria  bastante considerável. \\
Comparando as bandas calculadas a uma média em cada hora: \\


\begin{figure}[H]
    \centering
    \includegraphics[width=\textwidth]{plots/Média_historica_de_Banda_de_Reserva.png}
    \caption{Média historica de Banda de Reserva}
\end{figure}

Podemos ver que em termos de média horária, a Banda de Reserva calculada através do $\rho_{proposto}$ apresenta quase uma sobreposição por inteiro ao valor médio real. \\

Retiramos as médias dos erros percentuais e podemos observar: \\

\begin{figure}[H]
    \centering
    \includegraphics[width=\textwidth]{plots/Erro_médio_por_hora_Banda_de_Reserva.png}
    \caption{Erro médio por hora Banda de Reserva}
\end{figure}

Em termos de média diária o erro pelo método proposto está bem abaixo da margem de erro do 5\% na banda, em todas as horas. E na outra tese apenas 10\% cai dentro dessa margem de erro. \\

Como tal o $\rho_{proposto}$ a partir do estudo dos dados  históricos é: \

\begin{table}[H] \centering \caption{Valores de $\rho$ propostos} \begin{tabular}{rr}
\toprule
Hora & $\rho$ \\
\midrule
1 & 1.621694 \\
2 & 1.576623 \\
3 & 1.486929 \\
4 & 1.364176 \\
5 & 1.313958 \\
6 & 1.318832 \\
7 & 1.504499 \\
8 & 1.612361 \\
9 & 1.638188 \\
10 & 1.613728 \\
11 & 1.601277 \\
12 & 1.485861 \\
13 & 1.451995 \\
14 & 1.457233 \\
15 & 1.440454 \\
16 & 1.421988 \\
17 & 1.424636 \\
18 & 1.420682 \\
19 & 1.553086 \\
20 & 1.588201 \\
21 & 1.480219 \\
22 & 1.478815 \\
23 & 1.474412 \\
24 & 1.635658 \\
\bottomrule
\end{tabular}
 \end{table}

Em relação a perdas por arredondamento, apresento o resultado dos erros por arrendamento em cada um da casas possíveis, concluindo que até à primeira casa decimal, pode ser feito arredondamento do parâmetro $\rho$, sem influenciar muito o erro: \\


\begin{figure}[H]
    \centering
    \includegraphics[width=\textwidth]{plots/Erro_médio_por_hora_Banda_de_Reserva_Arredondamento.png}
    \caption{Erro médio por hora Banda de Reserva (Arredondamento)}
\end{figure}



