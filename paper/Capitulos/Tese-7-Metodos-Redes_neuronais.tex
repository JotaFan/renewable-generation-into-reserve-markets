\section{Redes Neuronais}

As redes neuronais podem ser descritas como uma função desconhecida f(x)=y onde durante o treino a função f é criada através da manipulação dos pesos da sua arquitetura usando os dados de treino, x, de forma a diminuir ao máximo uma função de perda . Sendo f'(x)=y' um modelo já treinado onde y' é a previsão, a função de perda fp(y, y') idealmente igual a 0, com y'=y.\\
Neste trabalho o x são todos os dados apresentados no capitulo \hyperref[ch:estudo_2]{Estudo 2}, em grupos de 128 (horas), e o y é a energia usada, "UpwardUsedSecondaryReserveEnergy" no modelo de previsão de energia a subir e "DownwardUsedSecondaryReserveEnergy" no modelo de previsão de energia a descer, nas 24 horas subsequentes. \\
Assim utilizamos os 168 horas (1 semana) para prever as 24 horas seguintes. A fp é um dos factores de estudo, assim como outros parâmetros dentro das arquiteturas de modelos, f. \\
As condições em estudo são feitas através da ferramenta \hyperref[se:muaddib]{MuadDib}, seguindo vários percursos entre as combinações possíveis, de modo a conseguir a combinação óptima, maior GPD Positivo.\\




\subsection{Arquitecturas}

% TODO: add ref to fcnn etc no capitulo a explicar isso
FCNN, CNN , RNN são as arquitecturas mais simples que vamos estudar. Estas vão apenas pegar nos blocos e vamos criar as mesmas "Vanila" e "Stacked" com 2 blocos (ex: StackedCNN) ou 6 blocos (ex: Stacked6CNN).\\
UNET, LSTM são arquiteturas mais complexas e pesadas. Como descrito anteriormente uma mais utilizada em análise de imagens, e outra em análise de texto respectivamente.\\
Transformers são as mais pesadas qualidade comum da família de "generative AI".

\subsection{Função de Perda}

Nos primeiros testes mais simples foi imediato a discrepância entre os erros da energia alocada em demasia e em falta. Sendo que estes erros estão em dimensões completamente diferentes.
\begin{figure}[H]
    \centering
    \includegraphics[width=\textwidth]{plots/allocs_results_shadow.png}
    \caption{Resultados de alocações totais em diferentes arquiteturas}
    \label{fig:resexparchs}
  \end{figure}

Na energia em falta, estamos a lidar com valores na dimensão de $10^{6}$ nos resultados, sendo que o benchmark está nos $10^{5}$10. Logo estão bastante acima do que queremos. Por outro lado na Energia em Demasia temos resultados na ordem dos $10^{6}$ e o benchmark está na ordem dos $10^{7}$. Isto dá-nos espaço para aumentar os resultados da Energia em Demasia mantendo-os ainda abaixo do benchmark para diminuir os resultados da Energia em Falta com objectivo de a ter também abaixo do benchmark.\\
Para combater esta desigualdade foram criadas várias funções de perda para atribuir melhor peso a ambas de modo a atingir melhor o objectivo geral.\\
De maneira que partimos esta experiência em duas partes. A primeira parte, Função de Perda Avançada, vai estudar diferentes maneiras de distribuir pesos entre a energia alocada em demasia e a em falta. A segunda vai escolher qual a melhor função de perda a aplicar nessa distribuição de pesos, ou vice-versa.\\


\subsubsection{Funções de Perda}
Depois de escolhidos os pesos nos diferentes grupos são testadas as funções a aplicar. Aqui vamos manter simples e testar apenas as mais comuns em problemas de regressão linear: Mean Absolute Error (MAE), Mean Squared Error (MSE), Mean Logarithmic Error (MLE).\\
MAE é usada no geral em problemas em que os dados têm um histograma linear, e um erro normalmente distribuído.\\
MSE é usado para atribuir maior peso aos erros maiores, do que no MAE. Fazendo com que o modelo se concentre mais em aprender a diminuir erros maiores.\\
MSLE é sugerido em dados que têm uma histograma exponencial.\\

% TODO: meter formulas? depende do espaço


\subsubsection{Função de Perda Avançada}\label{se:advancedloss}

Para escolher a melhor maneira de distribuir pesos foi criada uma função de perda com diferentes regras, que distribuem o peso da amostra.\\
\href{https://github.com/alquimodelia/alquitable/blob/main/alquitable/advanced_losses.py#L33}{Mirror Weights (Pesos Espelhados)}\\
Que vai distribuir os pesos da amostra consoante um rácio predefinido e o próprio erro da amostra.\\
Os pesos nas amostras vão ser divididos entre os erros negativos (alocação em demasia) e os positivos (alocação em falta). Consoante uma variável lógica,  uns terão peso 1 e os outros serão o próprio erro em absoluto. Dando assim um peso equivalente ao erro, quanto maior o erro maior o peso da amostra na função de perda, do lado da amostra escolhido (em demasia ou em falta).\\
Pode ser multiplicado um rácio tanto a um dos pesos como a outro, sendo estes rácios que irão equilibrar as diferenças entre a alocação em falta e a em demasia. E o sinal do rácio influencia qual o lado a ser multiplicado.\\
Este pesos são passados directamente à função de perda em uso.\\

% TODO: meter formulas? depende do espaço

\begin{figure}[H]
    \centering
    \includegraphics[width=\textwidth]{plots/ratio_mw.png}
    \caption{Resultados de alocações totais em diferentes racios}
    \label{fig:resexpratiomw}
  \end{figure}

Estas variações no rácio produzem diferentes dimensões nas alocações, modificando assim a sua posição em relação ao benchmark. Aqui para cada arquitetura o rácio ideal para o melhor GPD Positivo diferencia ligeiramente, tendo sido procurado com tentativa/erro baseado em assunções perante a aparente distribuição rácio/alocações.\\


\subsection{Função de Activação}

Como mostrado em \cite{Vaswani2017}, e \cite{Liu2022} , o uso de uma activação mais apropriada aos dados pode ser crucial para um salto na qualidade do modelo.\\
Vamos dividir as função de activação usadas nas camadas intermédias e a usada na camada final. Isto porque as camadas intermédias tendem a funcionar melhor com a mesma activação e a final é que mais define o valor que sai do modelo.\\
Esta experiência vai testar a combinações das seguintes activações nas duas variáveis descritas anteriormente: linear, relu, gelu.\\


\subsection{Pesos}
Esta experiência serve para testar diferentes pesos por amostra, não por grupo como na experiência anterior. Aqui os pesos são aplicados no momento da função de perda final.\\
Normalmente é usado para dar mais pesos a amostras com menor amostragem. Mais facilmente aplicável em modelos de classificação. Com este é um problema de regressão linear com séries temporais vamos testar aplicar os seguintes pesos, ou nenhum peso.\\
Este peso é multiplicado peso peso em \hyperref[se:advancedloss]{peso espelhados}.


\subsubsection{Temporais}
Aqui a primeira amostra tem o menor valor de peso (1) e todas as amostras seguintes incrementam 1. Dando mais peso consecutivamente a amostras mais recentes. É testado em vários casos de séries temporais onde o objectivo é prever o futuro. Podendo assim dar mais peso a tendências e valores mais recentes.\\

\subsubsection{Distância à média}
Neste peso cada amostra tem como valor a sua distância à média total dos dados. Vai servir para o modelo conseguir criar pesos relevantes a valores mais distantes à média.\\
Logo as amostras que tenham picos de valores têm um peso maior, forçando o modelo a aprender melhor estas ocasiões.

