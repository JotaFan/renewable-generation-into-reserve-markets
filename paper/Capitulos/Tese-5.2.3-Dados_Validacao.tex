\subsubsection{Dados de Validação}
Os dados de validação são os mesmos que os dados de treino, embora apenas durante os anos de 2019 a 2022, inclusive.\par
Usamos como benchmark as capacidades alocadas, "SecondaryReserveAllocationAUpward" e "SecondaryReserveAllocationADownward", e como validação e objectivo, \hyperref[se:metneuralnet]{y}, a própria energia usada, "UpwardUsedSecondaryReserveEnergy" e "DownwardUsedSecondaryReserveEnergy".

\textbf{Benchmark}

Como método de comparação a todas as experiências foi criado uma base que servirá de benchmark.\par
Este base não é nada mais do que a própria previsão feita pela entidade reguladora \gls{ESIOS}. Dentro do nossos dados são os valores nos campos "SecondaryReserveAllocationAUpward" e "SecondaryReserveAllocationADownward".\par
Para os dados utilizados, podemos ver a totalidade e comparação do benchmark (Energia Alocada) com a energia utilizada.\par

\begin{figure}[H]
    \centering
    \includegraphics[width=\textwidth]{plots/benchmark_alocacoes_validacao.png}
    \caption{Série Temporal dos dados de Benchmark c/ consumo real}
    \label{fig:benchmarktimeseries}
\end{figure}

Imediatamente podemos verificar que o método para prever a energia necessária actualmente está dentro de um espectro limitado de valores, sendo que esses valores estão perto dos valores de ponta na alocação para cima, e perto dos valores médios na alocação para baixo.\par
Isto deve-se ao facto de ser uma função fixa, baseado no dia em questão. Notamos também que a meio de 2022 houve uma mudança dessa função que limitou os alcances tornando os valores mais elevados. Devido à guerra na Ucrânia e à forte incerteza que esta trouxe aos mercados de eletricidade por causa da crise de gás na Europa, que aumentou significativamente o preço deste recurso e levou à adaptação dos consumidores e países, a \gls{REE} aumentou as necessidades de reserva secundária para responder a esta incerteza.\par
Do ponto de visto de dados faz sentido para diminuir a quantidade  de vezes em que não é alocada energia suficiente.\par
Mas o mais importante a notar é a forma estática destes métodos, dado a natureza flutuante dos da energia necessária este método apresenta frequentemente um erro grande.\par
Podemos ver em pormenor analisando algumas janelas temporais dentro do período de validação. Vendo o melhor e pior resultado, em termos de erro absoluto, em janelas temporais de ano, mês, semana e dia.\par


\begin{figure}[H]
    \centering
    \includegraphics[width=\textwidth]{plots/alocacoes_temporais_upward_dataset.png}
    \caption{Janelas temporais de benchmark energia a subir}
    \label{fig:benchmarktimewindowsup}
\end{figure}


\begin{figure}[H]
    \centering
    \includegraphics[width=\textwidth]{plots/alocacoes_temporais_downward_dataset.png}
    \caption{Janelas temporais de benchmark energia a descer}
    \label{fig:benchmarktimewindowsdown}
\end{figure}

Dentro destas janelas temporais conseguimos ter melhor a percepção da natureza estática deste modelo actual, e quão longe está dos valores reais necessários.\par

Os resultados a querer melhorar são:\\
\begin{table}[H]
    \caption{Resultados métricas benchmark}    
    \resizebox{\linewidth}{!}{\begin{table}[H] 
    \caption{Metric Results for Validation Data. \label{validation_res}}
    \newcolumntype{C}{>{\centering\arraybackslash}X}
    \begin{tabularx}{\textwidth}{CCCCC}
    \toprule
    & \textbf{RMSE}	& \textbf{SAE}	& \textbf{AllocM} & \textbf{AllocS}\\
    \midrule
    Up Allocation (MW) & 726.26 & 5787490.59 & 41080.10 & 5746410.49 \\
    Down Allocation (MW) & 794.53 & 6585513.97 & 15017.90 & 6570496.07 \\
        \bottomrule
    \end{tabularx}
    % \noindent{\footnotesize{\textsuperscript{1} Tables may have a footer.}}
\end{table}

}
    \label{tab:benchmarkmetrics}
    \end{table}

As correlações entre o método actual e a energia consumida podem ser vistas na figura abaixo:\\


\begin{figure}[H]
    \centering
    \includegraphics[width=\textwidth]{plots/correlation_heatmap_benchmark.png}
    \caption{Correlação entre benchmark e real}
    \label{fig:benchmarkcorr}
\end{figure}

% TODO: meter formula da previsão ENTO-e no capitulo estudo 2
As relações entre as energias alocadas são altas devido à natureza do \hyperref[]{método de previsão} enquanto que a correlação entre a energia alocada e a usada são bastante baixas com 21\% na alocaçao a descer e 2\% na alocação a subir.\par
O que não mostra uma ligaçao entre as alocações e a energia usada, mas apenas entre as energias alocadas.\par

