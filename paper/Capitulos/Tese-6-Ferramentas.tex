\chapter{Ferramentas}

\section{\href{https://github.com/alquimodelia/forecat/tree/main/forecat}{Forecat}\label{se:forecat}}


Com o propósito de desenvolver este estudo, e deixar ferramentas para a replicação do mesmo, foi criado uma biblioteca em python para desenhar as arquitecturas em estudo.\\

\subsection{Construtor de modelos}

Seguindo as arquitecturas descritas anteriormente esta ferramenta constrói os modelos automaticamente, sendo que precisamos apenas de fornecer os parâmetros variáveis.\\
O construtor assenta na ideia de três camadas abstratas de redes neuronais. A camada de entrada, a camada de bloco, e a camada interpretativa.\\
A camada de entrada recebe os dados e normaliza, podendo também fazer outras operações de preparação para a camada de bloco.\\
A camada de bloco é a camada descritiva da arquitetura, é a que tem as operações fundamentais.\\
A camada interpretativa é a que recebe o sinal de múltiplas redes neuronais internas, e traduz para o objectivo, usando Dense layers. \\

Esta abstração segue sempre esta ordem. As variações dentro de cada arquitetura dependem das governamentalizações das mesmas, ou de cada camada, ou então da repetição do circuito, em paralelo ou em série. ou uma combinação destes.\\

\subsection{Gerador de dados}

O gerador construido trata da formatação dos dados para entrada nos modelos. Formatação esse que se baseia nos valores de janelas temporais a usar, e na divisão treino/teste.\\
Esta ferramenta agrega os dados em tensores de formato \textit{(N, t, a)}, onde \textit{N} é o número de casos, \textit{t} é a janela temporal, e \textit{a} é o número de atributos.\\
A ferramenta permite também definir o tempo de salto entre cada entrada.\\
Usando como exemplo uma janela temporal de 168 (horas, uma semana) para treino, e 24 (horas) para o alvo. Com um salto temporal de 1 a primeira entrada teria como treino as primeiras 168 horas dos dados, e como alvo as 24 horas consequentes. A segunda entrada seria a partir da segunda hora dos dados, e assim consecutivamente. Para um caso em que o tempo de salto seria 24, a primeira entrada mantinha-se, mas a segunda começaria 24 horas depois, e não apenas uma.\\

Como estamos também a lidar com dados desfasados, o gerador permite um TODO: shift em atributos a especificar. No caso em estudo temos que os atributos são de DA (day-ahead), logo estão desfasados 24 horas. O que implica termos de aplicar este shift nos dados que não são DA, nomeadamente os dados alvo. Esta propriedade permite também o fácil uso da ferramenta noutros dados desfasados, como as previsões a 3 ou 8 horas.\\

\section{\href{https://github.com/alquimodelia/MuadDib}{MuadDib}\label{se:muaddib}}

Esta ferramenta criada para desenvolver as experiências desta dissertaçao, permite ao utilizador apenas com os dados que quer utilizar e a especificaçao das metricas pretendidas, facilmente ter um modelo optimizado para os seus dados e problema. \\
Criada especificamente no ambito deste trabalho, está focada em criaçao de previsões, podendo no entanto ser utilizada para outros fins. \\
A configuração é bastante facil de usar, e tem em conta um publico não especializado. O que significa que qualquer utilizador com um conjunto de dados pode fazer previsões utilizando machine learning.

