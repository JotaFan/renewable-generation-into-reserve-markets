\chapter{Contextos\label{ch:contextos}}

\section{Mercado de Serviços de Sistema \label{se:servicos_sistema}}
%\cite{Lopes2021}
%\cite{Watson1984}
%\citep{Schweppe1988}

O mercado de serviço de sistema é parte integrante dos mercados de energia e mantêm responsabilidade sobre a segurança do mesmo.\cite{dgegmss} \\
Serve para garantir o equilibrio entre a energia produzido e a consumida. Esta qualidade e segurança é controlada através da frequência e da potência activa, controlo de tensão e potência reactiva, arranque automático e outras técnicas de sistemas \cite{Rassid2017} \cite{Carneiro2016}. \\
Neste caso de estudo estamos interessados nos serviçoes de controlo de frequência. A nível europeu estes serviços são impostos pela ENTO-E (\textit{European Network of Transmission System Operators for Electricity}), e a operação dos mesmos é da responsabilidade dos TSO (\textit{Transmission System Operator} ou \textit{ Operador da Rede de Transporte}) nacionais.\\
Para manter o controlo de frequência o gestor de sistema deverá manter reservas para responder às diferenças entre a energia consumida e produzida na rede, que deve ser mantida em equilibrio. Quando o serviço de sistema precisa de actuar para manter a frequência no seu valor nominal, 50Hz na Europa, isto é feito alterando a potência activa dos geradores.  \\
Quando é necessário um aumento na potência chama-se a isto Banda de Reserva/Regulação a Subir, e quando é necessária uma diminuição chama-se à mesma a Descer. \\
Para isto, mo mercado ibérico, a tarefa é dividida em três reservas, primária, secundária e terciária. Esta divisão assenta no tempo de resposta que os sistemas precisam de ter, e na capacidade de actuação (MWh/Hz). \\

A reserva secundária, como sistema de segurança à reserva primária, regula-se pelo mercado de banda das reservas secundárias, que decorre no dia anterior ao que será necessário utilização da mesma. \\
Este valor alocado tem um custo para as operadoras, como tal a previsão do mesmo é importante para a gestão destes sistemas de segurança. Estas previsões são feitas através de estatiscas dos sistemas, e tendo em conta as areas de balanço que o mesmo têm. \\
Uma melhor previsão deste valor poderia levar a uma poupança, tanto financeira, como de recursos. \\

Estas previsões são feitas ao uso de formulas. Que por si só não preveêm a variabilidade dos sistemas de produção de energia renovável. Esta variabilidade sendo dificilmente previsivel, tem sido alvo de estudo com modelos de \textit{machine learning} \cite{} \\
Com bons resultados apresentados em estudo de energias renováveis, a aplicação dos mesmos metódos para as reservas de sistema parece um passo natural. \\



%\section{MIBEL \label{se:mibel}}
%\cite{Bessa2012}
%\cite{Carneiro2016}
%\cite{Fernandes2016}
%\citep{Agostini2021}

