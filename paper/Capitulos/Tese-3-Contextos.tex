\chapter{Contextos\label{ch:contextos}}

\section{Mercados de Energia}

\subsection{Mercado Ibérico de Electricidade \label{se:mibel}}

O \gls{MIBEL} é um exemplo de integração de mercados de energia entre países, funcionando como um elo entre os mercados de eletricidade de Portugal, \gls{OMIP} e Espanha, \gls{OMIE}. Este mercado grossista compreende diferentes formatos de negociação, cada um desempenhando um papel específico na gestão da compra e venda de eletricidade. \\
O \gls{OMIP} é responsável pela negociação a prazo de energia elétrica enquanto que o  \gls{OMIE} é responsável pela negociação diária de energia elétrica.\\
O \gls{MIBEL} é estruturado para fornecer uma plataforma eficiente e transparente para a transação de energia, garantindo a competitividade e a segurança de fornecimento. Vamos explorar os principais componentes deste modelo.\\


\subsubsection{Mercado em Bolsa (Mercado Spot) \label{se:mercado_bolsa}}
O mercado em bolsa, também conhecido como mercado spot, é uma das principais formas de negociação no \gls{MIBEL}. Este mercado é dividido em duas vertentes: o mercado diário e o mercado intradiário. No mercado diário, as propostas de compra e venda de eletricidade são apresentadas para o dia seguinte, permitindo que os agentes ajustem suas previsões de produção e consumo com base nas condições de mercado mais recentes. Já o mercado intradiário permite a negociação para as horas seguintes, oferecendo maior flexibilidade para ajustes de última hora, o que é especialmente útil para acomodar variações inesperadas na oferta e demanda. Este sistema dinâmico assegura que a eletricidade é negociada em tempo real, refletindo as necessidades e capacidades do sistema elétrico com um horizonte de curto prazo.\\


\subsubsection{Mercado de Contratação a Prazo \label{se:mercado_prazo}}
Além do mercado spot, o \gls{MIBEL} inclui o mercado de contratação a prazo, onde os agentes estipulam compromissos de compra e venda de eletricidade com semanas, meses, ou até anos de antecedência. Este mercado permite aos participantes fixar preços e volumes de energia para o futuro, mitigando os riscos associados à volatilidade dos preços no curto prazo. A contratação a prazo proporciona uma maior previsibilidade e estabilidade financeira para os produtores e consumidores de energia, permitindo um planejamento estratégico mais robusto. Os contratos podem variar em termos de maturidade, desde acordos de curto prazo até contratos de longo prazo, dependendo das necessidades e estratégias dos agentes envolvidos.\\


\subsubsection{Mercado Livre de Contratação Bilateral Física \label{se:mercado_bilateral}}
Outra componente importante do \gls{MIBEL} é o mercado livre de contratação bilateral física, onde os agentes negociam diretamente a compra e venda de eletricidade para uma determinada maturidade no futuro. Este formato permite uma maior personalização dos contratos, uma vez que as condições podem ser ajustadas diretamente entre as partes envolvidas, sem a intervenção de um mercado centralizado. Esse tipo de negociação é particularmente vantajoso para grandes consumidores e produtores que procuram acordos específicos para atender às suas necessidades operacionais ou estratégias de hedge contra flutuações de preços. A liberdade de negociação bilateral física oferece um nível adicional de flexibilidade e controle sobre as transações, promovendo uma maior eficiência no mercado.\\

\subsubsection{Mercado de Serviços de Sistema \label{se:servicos_sistema_mibel}}
Por fim, o mercado de serviços de sistema desempenha um papel crítico na manutenção do equilíbrio entre a produção e o consumo de energia elétrica em tempo real. Este mercado é responsável por garantir que a rede elétrica opere de forma segura e estável, ativando reservas e ajustando a produção conforme necessário para responder a variações inesperadas na demanda ou na oferta. O mercado de serviços de sistema engloba uma série de mecanismos, incluindo a ativação de reservas de frequência e o despacho de unidades geradoras flexíveis, que são essenciais para a gestão da estabilidade da rede. A participação neste mercado é muitas vezes obrigatória para certos tipos de geradores, especialmente aqueles que possuem a capacidade de resposta rápida, como hidroelétricas e centrais térmicas.\\
Os mercados de serviços de sistema, português e espanhol, são geridos independentemente, onde o \gls{GGS} é o operador do mercado no respectivo país. \gls{REN} em Portugal e \gls{REE} em Espanha.\\
\bigskip
\bigskip
Em resumo, o \gls{MIBEL} é um mercado complexo e multifacetado que oferece uma ampla gama de formatos de negociação para atender às diversas necessidades dos agentes de mercado. Desde a negociação em tempo real no mercado spot até compromissos de longo prazo no mercado de contratação a prazo e acordos personalizados no mercado bilateral, o \gls{MIBEL} proporciona um ambiente robusto para a transação de eletricidade, promovendo a eficiência, a flexibilidade e a segurança do fornecimento de energia na Península Ibérica.\cite{Rassid2017}\\



\begin{figure}[H]
	\centering
	\resizebox{\linewidth}{!}{\begin{tikzpicture} [ node distance = 2cm, auto, block/.style={ rectangle, draw, align=center, minimum width=2cm, minimum height=1cm }, line/.style={ draw, -latex' } ]

    \node [block] (top) {Mercados Organizados};
    \node [block, below=of top](spot) {Mercado Spot \gls{OMIE}}; 

    \node [block, left=of spot](prazo) {Contractos a prazo \gls{OMIP}}; 
    \node [block, right=of spot](ss) {Serviços de Sistema};

    \draw [line] (top) -- (prazo);
    \draw [line] (top) -- (spot);
    \draw [line] (top) -- (ss);
    
    \node[fit=(top)(spot)(prazo)(ss),  minimum width=10cm, minimum height=4cm] (group) {};

    \node[block, left=of group, minimum width=1cm, minimum height=4cm] (left_block) {Contractos Bilaterias};

    \node[fit=(left_block)(prazo)(spot),  minimum width=10cm] (group2) {};
    \node[block, below=of group2, minimum width=10cm] (group2) {Negociação Mibel};

    \node[block, below right=of ss] (ren) {Portugal \gls{REN}};
    \node[block, below=of ren] (ree) {Espanha \gls{REE}};
    \draw [line] (ss) |- (ren);
    \draw [line] (ss) |- (ree);



    \end{tikzpicture}}
	\caption{Organizaçao MIBEL. Adaptado de \cite{Rassid2017}}
	\label{fig:mibel_org}
\end{figure}




\subsection{Mercado de Serviços de Sistema \label{se:servicos_sistema}}
%\cite{Lopes2021}
%\cite{Watson1984}
%\citep{Schweppe1988}

% O mercado de serviço de sistema é parte integrante dos mercados de energia e mantêm responsabilidade sobre a segurança do mesmo.\cite{dgegmss}.
% Serve para garantir o equilíbrio entre a energia produzido e a consumida. Esta qualidade e segurança é controlada através da frequência e da potência activa, controlo de tensão e potência reactiva, arranque automático e outras técnicas de sistemas \cite{Rassid2017} \cite{Carneiro2016}. \\
% Neste caso de estudo estamos interessados nos serviços de controlo de frequência. A nível europeu estes serviços são impostos pela \gls{ENTSO-E}, e a operação dos mesmos é da responsabilidade dos \gls{TSO} nacionais.\\
% Para manter o controlo de frequência o gestor de sistema deverá manter reservas para responder às diferenças entre a energia consumida e produzida na rede, que deve ser mantida em equilíbrio. Quando o serviço de sistema precisa de actuar para manter a frequência no seu valor nominal, 50Hz na Europa, isto é feito alterando a potência activa dos geradores.  \\
% Quando é necessário um aumento na potência chama-se a isto Banda de Reserva/Regulação a Subir, e quando é necessária uma diminuição chama-se à mesma a Descer.
% Para isto, no mercado ibérico, a tarefa é dividida em três reservas, primária, secundária e terciária. Esta divisão assenta no tempo de resposta que os sistemas precisam de ter, e na capacidade de actuação (MWh/Hz). \\

% A reserva primária é a primeira a ser activada em resposta a distúrbios na rede, como desvios de frequência, e deve agir em questão de segundos para estabilizar o sistema. A reserva secundária, que funciona como um sistema de segurança para a reserva primária, é regulada pelo mercado de banda das reservas secundárias, e sua alocação ocorre no dia anterior à sua necessidade. A reserva terciária é usada para complementar as reservas anteriores e para lidar com desvios de potência activa de longa duração.
% O valor alocado de reserva secundária tem um custo para as operadoras, como tal a previsão do mesmo é importante para a gestão destes sistemas de segurança. Estas previsões são feitas através de estatísticas dos sistemas, e tendo em conta as áreas de balanço que os mesmos têm.\\
% Assim como no mercado de energia de balanço, onde a participação de \gls{vRES} é cada vez mais incentivada, também nos serviços de reserva há uma crescente necessidade de adaptar esses mercados para acomodar a variabilidade e a imprevisibilidade inerente das \gls{vRES}, como a energia eólica e solar. A falta de harmonização nos mercados de balanço, particularmente na forma como os preços de desequilíbrios são determinados, apresenta um desafio significativo. Embora a participação de produtores de energia renovável nesses mercados seja tecnicamente viável, existem restrições que visam garantir a segurança e a estabilidade da rede, além de questões económicas que precisam ser abordadas para tornar esses mercados mais atrativos para todos os participantes.\\

% Uma previsão precisa das necessidades de reservas secundárias é crucial, pois impacta diretamente os custos e recursos das operadoras.
% Estas previsões são feitas usando fórmulas. Que por si só não prevêem a variabilidade dos sistemas de produção de energia renovável. Esta variabilidade sendo dificilmente previsível, tem sido alvo de estudo com modelos de \textit{machine learning}.\\
% Com o aumento de dados históricos destes mercados começa a ser possível estudar os mesmos para criar métodos empíricos para estas previsões.\\
% Com bons resultados apresentados em estudo de energias renováveis, a aplicação dos mesmos métodos para as reservas de sistema parece um passo natural. \\


\subsubsection{Introdução ao Mercado de Serviços de Sistema \label{se:intro_servicos_sistema}}

O mercado de serviços de sistema é uma componente fundamental dos mercados de energia, desempenhando um papel crucial na manutenção da segurança e estabilidade das redes elétricas \cite{dgegmss}. Esses serviços são essenciais para garantir que a produção e o consumo de energia permaneçam em equilíbrio, um requisito vital para o funcionamento seguro e eficiente de qualquer sistema eléctrico. A principal função dos serviços de sistema é assegurar a qualidade da energia fornecida, monitorizando parâmetros críticos como a frequência, a potência activa e reactiva, controlando a tensão na rede, arranque automático e outras técnicas de sistemas. Esse controlo é realizado através da coordenação entre os geradores e os consumidores, com o objetivo de responder rapidamente a variações na oferta e na procura de energia \cite{Rassid2017} \cite{Carneiro2016}.\\
No contexto europeu, a regulação desses serviços é coordenada pela \gls{ENTSO-E}, que estabelece os requisitos e normas para a operação dos sistemas de energia, e a operação dos mesmos é da responsabilidade dos \gls{TSO} nacionais. Essas reservas são activadas conforme necessário para manter a frequência da rede no seu valor nominal de 50Hz, ajustando a potência activa dos geradores em resposta a variações imprevistas na demanda ou na oferta de energia.\\
As reservas de frequência são divididas em três categorias principais: primária, \gls{FCR}, secundária, \gls{aFRR}, e terciária, \gls{mFRR}. Cada uma com funções específicas e tempos de resposta distintos. A reserva primária é activada automaticamente e de forma quase instantânea, dentro de segundos após um distúrbio na rede, para estabilizar rapidamente a frequência. A reserva secundária entra em ação logo em seguida, substituindo gradualmente a reserva primária e ajustando a frequência de volta ao seu valor programado. Finalmente, a reserva terciária é utilizada para corrigir desvios de longo prazo e libertar as outras reservas para possíveis eventos futuros, completando o ciclo de controlo da frequência e assegurando que o sistema retorne a um estado de equilíbrio estável.\\
A harmonização dos mercados europeus de eletricidade, especialmente nos mercados diários, intradiários e de balanço, é uma realidade em desenvolvimento que busca reduzir custos e melhorar as condições de participação para todos os envolvidos \cite{Algarvio2019}. No entanto, a integração das \gls{vRES}, como a eólica e a solar, apresenta desafios adicionais devido à sua natureza intermitente e dependente de condições climáticas. Embora tecnicamente viável, devido a este paradigma de imprevisibilidade e ao facto de serem fontes não despacháveis, a participação dessas fontes nos mercados de balanço enfrenta restrições significactivas para garantir a segurança e a estabilidade da rede.\\
A atual infraestrutura dos mercados de serviços de sistema precisa, portanto, ser adaptada para acomodar essas novas fontes de energia. Uma parte essencial dessa adaptação é o desenvolvimento de métodos mais robustos para prever a necessidade de reservas, que levem em consideração a variabilidade das \gls{vRES}. Actualmente, as previsões são baseadas principalmente em fórmulas criadas pelas operadoras, mas esta abordagem muitas vezes falha em capturar a complexidade e a incerteza associadas à produção renovável. Assim, há uma crescente exploração de técnicas avançadas, como o uso de modelos de \textit{machine learning}, para melhorar a precisão das previsões e otimizar a gestão das reservas.
Além disso, a evolução para um mercado pan-europeu harmonizado de serviços de sistema envolve não apenas a uniformização de regras e requisitos técnicos, mas também a criação de incentivos económicos que tornem a participação atraente para todos os tipos de produtores de energia, incluindo os renováveis. Isso é particularmente importante, uma vez que os mercados de balanço são fundamentais para garantir que as redes elétricas possam operar de forma estável e segura, mesmo com altas penetrações de \gls{vRES}. Ao permitir que essas fontes renováveis participem de forma mais activa e competitiva nos mercados de balanço, espera-se não apenas reduzir os custos de operação dos sistemas eléctricos, mas também aumentar a viabilidade económica das \gls{vRES}.\\
Com a crescente dependência de fontes de energia renovável e a necessidade de sistemas eléctricos mais resilientes e flexíveis, o papel dos serviços de sistema continuará a expandir-se e a evoluir, exigindo inovações tanto na gestão técnica quanto na regulação econômica dos mercados de energia.\\


\subsubsection{Estrutura e Funcionamento das Reservas de Frequência \label{se:reservas_freq}}


A reserva primária, \gls{FCR}, é o primeiro nível de resposta e é acionada automaticamente em questão de segundos após a detecção de um desvio de frequência, que pode ocorrer devido a falhas na produção ou variações repentinas na procura. Esta reserva é activada até 15 segundos após o distúrbio e permanece activa por cerca de 30 segundos, ou até que a reserva secundária possa assumir o controlo. A \gls{FCR} é geralmente suportada por geradores que possuem capacidade técnica para resposta rápida, como hidroelétricas e algumas unidades térmicas. Este serviço é obrigatório para todos os geradores conectados à rede que possuem a capacidade técnica necessária, e não é remunerado em muitos mercados europeus, incluindo o mercado ibérico.\\
A reserva secundária, \gls{aFRR}, entra em ação logo após a activação da reserva primária, com o objetivo de restaurar a frequência da rede ao seu valor programado de 50 Hz e libertar a \gls{FCR} para responder a possíveis distúrbios subsequentes. A \gls{aFRR} é activada automaticamente até 30 segundos após o desvio inicial e pode levar até 15 minutos para corrigir completamente o desequilíbrio. Este tipo de reserva é contratado em mercados específicos de banda de reserva, nos quais os geradores submetem ofertas para fornecer a capacidade necessária.\\
A reserva terciária, \gls{mFRR}, é o último nível de resposta e é utilizada principalmente para corrigir desequilíbrios de longo prazo e libertar a \gls{aFRR} para outros usos. Ao contrário das reservas primária e secundária, a \gls{mFRR} é activada manualmente pelos \gls{TSO} e pode levar até 15 minutos a estar completamente activa. Esta reserva é frequentemente utilizada para ajustar a geração ou o consumo de energia de acordo com desvios significativos e prolongados, que não podem ser compensados de forma eficaz pelas reservas de resposta mais rápida. A \gls{mFRR} é geralmente suportada por geradores que podem oferecer flexibilidade em suas operações, como algumas centrais térmicas e hidroelétricas de grande porte.\\

\subsubsection{Previsão de Necessidades de Reservas \label{se:pred_impact_vres}}

A previsão das necessidades de reservas de frequência é uma componente essencial na gestão eficiente dos sistemas eléctricos, especialmente num cenário de crescente penetração das \gls{vRES}.\\
O uso de técnicas de \textit{machine learning} tem sido explorado como uma solução promissora para melhorar essas previsões. Estes modelos podem analisar grandes volumes de dados, identificar padrões complexos e ajustar previsões em tempo real, levando em consideração factores como mudanças nas condições meteorológicas e padrões de consumo de energia. Ao incorporar a variabilidade das \gls{vRES} nos modelos de previsão, é possível reduzir a incerteza e melhorar a alocação das reservas de frequência, resultando numa operação mais eficiente do sistema eléctrico.\\
Outro factor crítico na previsão das necessidades de reservas de frequência é a coordenação entre diferentes mercados e operadores de sistemas. A harmonização dos mercados europeus de balanço, incluindo a padronização das regras de oferta, leilão e remuneração, pode facilitar a integração das \gls{vRES} e melhorar a eficiência geral do sistema. Com regras claras e uniformes, os produtores de energia renovável têm maior incentivo para participar activamente dos mercados de reservas, fornecendo capacidade adicional para apoiar a estabilidade da rede. Isso é particularmente relevante em mercados onde as \gls{vRES} ainda enfrentam barreiras significactivas para a participação, como regras complexas de licitação ou altos requisitos de capacidade mínima para participação.\\
Apesar dos avanços na previsão de necessidades de reservas, ainda existem desafios consideráveis. A precisão das previsões pode ser limitada pela qualidade dos dados disponíveis, bem como pela capacidade dos modelos de capturar todas as variáveis relevantes que afetam a operação da rede. Além disso, a crescente interconexão dos sistemas eléctricos e o aumento da troca de energia entre países exigem uma abordagem coordenada e colaboractiva para a previsão de reservas, considerando tanto as condições locais quanto regionais.\\
O desenvolvimento contínuo de técnicas avançadas de previsão e a integração de soluções baseadas em dados serão fundamentais para enfrentar esses desafios. À medida que mais dados históricos se tornam disponíveis e os modelos de previsão evoluem, espera-se que a gestão das reservas de frequência se torne cada vez mais eficiente, contribuindo para um sistema eléctrico mais resiliente e capaz de integrar altos níveis de energias renováveis. Isso não apenas reduzirá os custos operacionais, mas também contribuirá para a segurança energética e para a transição para um sistema energético mais sustentável.\\


%\subsection{\gls{MIBEL} \label{se:mibel}}
%\cite{Bessa2012}
%\cite{Fernandes2016}
%\citep{Agostini2021}


\thispagestyle{plain}
\section{Arquitecturas de Modelos\label{se:arquiteturas_modelos}}

Grande parte da literatura sobre previsões em modelos de apredizagem apresenta as mesmas arquiteturas, sendo que são depois aprimoradas consoate os dados e o problema. \\
Apresento aqui as arquitecturas mais usadas em previsões, como também algumas usadas noutros ramos tentado prever a compatibilidade neste problema. \\
% TODO: meter citaçao para o uso de cada uma em previsões
Neste trabalho vamos usar arquiteturas de \gls{FCNN}, \gls{CNN}, \gls{LSTM} e Transformer.\\




\subsection{FCNN\label{se:fcnn_sec}}

A arquitetura mais simples \gls{FCNN}, Redes Neuronais Totalmente Conectadas , é constituída por camadas em que cada neurónio está ligado a todos os neurónios da camada seguinte. Isto significa que cada caraterística de entrada tem um peso associado, e esses pesos são aprendidos durante o treino. A saída de cada neurónio é calculada através da aplicação de uma função de ativação à soma ponderada das suas entradas.\\
Cada neurónio gera uma operação, inicialmente aleatória, para tentar reproduzir uma função que traduza a entrada na saída ideal.\\
Esta arquitectura tem como base o Perceptão inicialmente proposto em \cite{Rosenblatt1958}. Este apresentava um Perceptão que fazia uma decisão binária baseado nas somas pesadas de todas as entradas.\\
A ideia é a base utilizada actualmente, mas apresentava algumas limitações, e muita computação, o proposto por \cite{Minsky1969}, eleva a ideia com a introdução da função de activação e o bias. A utilização mais recorrente actual é a proposta em \cite{Haykin1999}.


\begin{figure}[H]
	\centering
	\resizebox{\linewidth}{!}{\begin{tikzpicture}[scale=2.4, transform shape]
    % Draw input nodes
    \foreach \h [count=\hi ] in {$x_2$,$x_1$}{%
          \node[input] (f\hi) at (0,\hi*1.25cm-1.5 cm) {\h};
        }
    % Dot dot dot ... x_n
    \node[below=0.62cm] (idots) at (f1) {\vdots};
    \node[input, below=0.62cm] (last_input) at (idots) {$x_n$};
    % Draw summation node
    \node[functions] (sum) at (4,0) {\Huge$\sum$};
    \node[below=0.69cm] at (sum) {$\sum_{i=0}^n w_ix_i$};
    % Draw edges from input nodes to summation node
    \foreach \h [count=\hi ] in {$w_2$,$w_1$}{%
          \path (f\hi) -- node[weights] (w\hi) {\h} (sum);
          \draw[->] (f\hi) -- (w\hi);
          \draw[->] (w\hi) -- (sum);
        }
    % Dot dot dot ... w_n
    \node[below=0.05cm] (wdots) at (w1) {\vdots};
    \node[weights, below=0.45cm] (last_weight) at (wdots) {$w_n$};
    % Add edges for last node and last weight etc
    \path[draw,->] (last_input) -- (last_weight);
    \path[draw,->] (last_weight) -- (sum);
    % Draw node for activation function
    \node[functions] (activation) at (7,0) {};
    \node[small_input, below=1cm] (bias) at (activation) {bias};
    \path[draw,->] (bias) -- (activation);

    % Place activation function in its node
    \begin{scope}[xshift=7cm,scale=1.25]
        \addaxes
        % flexible selection of activation function
        \relu
        % \stepfunc
    \end{scope}
    % Connect sum to relu
    \draw[->] (sum) -- (activation);
    \draw[->] (activation) -- ++(1,0);
    % Labels
    \node[above=1cm]  at (f2) {Entradas};
    \node[above=1cm] at (w2) {Pesos};
    \node[above=1cm] at (activation) {Função de activação};

    % Neuron
    \node[draw, dashed, fit= (w2) (last_weight) (activation) (bias), inner sep=0.5em] (square){};
    \node[below=1.5cm] at (square) {Neurónio};

\end{tikzpicture}}
	\caption{Ilustração de um neurónio. Adaptado de \cite{Haykin1999}}
	\label{fig:neuronio}
\end{figure}



\subsection{CNN\label{se:cnn_sec}}
% TODO: add cites do conv 
As Redes Neuronais Convolucionais (\gls{CNN}) diferem das \gls{FCNN} no sentido em que os filtros (neurónios) não são criados aleatoriamente, mas sim cada filtro trata de uma parte da camada de entrada. Nas convoluções é criada uma janela móvel que percorre a camada, criando um saída desse conjunto de pontos. Esta janela move-se sempre subsequentemente.\\
Esta operação é normalmente feita na dimensão (ou dimensões) em que queremos perceber padrões.\
Nos nossos dados a convolução será na dimensão temporal.\\
Se tivermos uma matriz com nove passos temporais (N,9,1), se o tamanho da janela de convolução for 3, teremos uma saída de tamanho 6 (N, 6, 1).\\
\begin{figure}[H]
	\centering
	\resizebox{\linewidth}{!}{
\begin{tikzpicture}
    % Time series
    \matrix (M1) [matrix of math nodes, nodes={draw, minimum size=1cm, anchor=center}, 
    column sep=-\pgflinewidth, row sep=-\pgflinewidth,
    ] {
        \node[fill=red, draw=red] (M1-1-1) {1}; & \node[fill=red, draw=red] (M1-2-1) {2};
        & \node[fill=red, draw=red] (M1-3-1) {3}; &
        4 & 5 & \node[draw=yellow] (M1-6-1) {6}; & \node[draw=yellow] (M1-7-1) {7};
        & \node[draw=yellow] (M1-8-1) {8}; & 9 \\
    };
    
    % Kernel
    \matrix (M2) [below=of M1, matrix of math nodes, nodes={draw, minimum size=1cm, anchor=center}, 
    column sep=-\pgflinewidth, row sep=-\pgflinewidth,
    ] {
        \node[fill=red, draw=red] (M2-1-1) {6}; & 9 & 12 & 15 & 18 & \node[draw=yellow] (M2-5-1) {21}; & 24 \\
    };

    \draw[dashed, red] (M1-1-1.north west) -- (M2-1-1.north west);
    \draw[dashed, red] (M1-3-1.north east) -- (M2-1-1.north east);
    \draw[dashed, red] (M1-1-1.south west) -- (M2-1-1.south west);
    \draw[dashed, red] (M1-3-1.south east) -- (M2-1-1.south east);

    \draw[dashed, yellow] (M1-6-1.north west) -- (M2-5-1.north west);
    \draw[dashed, yellow] (M1-8-1.north east) -- (M2-5-1.north east);
    \draw[dashed, yellow] (M1-6-1.south west) -- (M2-5-1.south west);
    \draw[dashed, yellow] (M1-8-1.south east) -- (M2-5-1.south east);

    
    % Titles
    \node [right=1cm, align=center, font=\bfseries] at (M1.east) {Série Temporal};
    \node [right=1cm, align=center, font=\bfseries] at (M2.east) {Filtro};
    \end{tikzpicture}}
	\caption{Ilustração da operação de Convolução}
	\label{fig:conv_layer1D}
\end{figure}

Anteriormente ignoramos o número de filtros. Mas as convoluções criam o número pedido de filtros para cada janela temporal. Aqui cada filtro vai funcionar como na camada \gls{FCNN}, onde cada um começa com uma operação pseudo aleatória. Esta operação normalmente é feita na dimensão dos atributos.\\
Ou seja, a quantidade de filtros que esta camada irá produzir por convolução.\\
Se tivermos a mesma entrada que anteriormente mas com 4 atributos (N, 9, 4), e se definir o número de filtros para 2 teremos uma saída (N, 6, 2).\\
Ou seja, dois filtros por cada janela temporal.\\


\begin{figure}[H]
	\centering
	\resizebox{\linewidth}{!}{\begin{tikzpicture}[scale=2]
    \matrix (M1) [matrix of math nodes, nodes={draw, minimum size=1cm, anchor=center}, 
    column sep=-\pgflinewidth, row sep=-\pgflinewidth,]
{
    \node[draw=red] (M1-1-1) {1}; & \node[draw=red] (M1-1-2) {2}; & \node[draw=red] (M1-1-3) {3}; & 4 & 5 & 6 & 7 & 8 & 9\\
    \node[draw=red] (M1-2-1) {4}; & \node[draw=red] (M1-2-2) {5}; & \node[draw=red] (M1-2-3) {6}; & 7 & 8 & 9 & 10 & 11 & 12\\
    \node[draw=red] (M1-3-1) {7}; & \node[draw=red] (M1-3-2) {8}; & \node[draw=red] (M1-3-3) {9}; & 10 & 11 & 12 & 13 & 14 & 15\\
\node[draw=red] (M1-4-1) {10}; & \node[draw=red] (M1-4-2) {11};
& \node[draw=red] (M1-4-3) {12}; & 13 & 14 & 15 & 16 & 17 & 18\\
};

\matrix (M2) [matrix of math nodes, nodes={draw, minimum size=1cm, anchor=center}, 
column sep=-\pgflinewidth, row sep=0.3cm,
    right=of M1]
{
    \node[fill=red, draw=red] (M2-1-1) {78}; & 90 & 102 & 114 & 126 & \node[draw=yellow] (M2-1-5) {138}; & 150 \\
    \node[fill=red, draw=red] (M2-2-1) {6}; & 9 & 12 & 15 & 18 & \node[draw=yellow] (M2-2-5) {21}; & 24 \\
    };

% Titles
\node [above=0.5cm, align=center, font=\bfseries] at (M1.north) {Série Temporal};
\node [above=0.5cm, align=center, font=\bfseries] at (M2.north) {Filtros};
 
\draw[dashed, red] (M1-1-1.north west) -- (M2-1-1.north west);
\draw[dashed, red] (M1-1-1.north west) -- (M2-2-1.north west);

\draw[dashed, red] (M1-1-3.north east) -- (M2-1-1.north east);
\draw[dashed, red] (M1-1-3.north east) -- (M2-2-1.north east);

\draw[dashed, red] (M1-4-1.south west) -- (M2-1-1.south west);
\draw[dashed, red] (M1-4-1.south west) -- (M2-2-1.south west);


\draw[dashed, red] (M1-4-3.south east) -- (M2-1-1.south east);
\draw[dashed, red] (M1-4-3.south east) -- (M2-2-1.south east);



% TODO: add legenda em pequeno
%\node [right=1cm, align=center, font=\bfseries] at (matrix1.west) {Atributos};
%\node [right=1cm, align=center, font=\bfseries] at (matrix1.south) {tempo};


\end{tikzpicture}}
	\caption{Ilustração da camada de Convolução}
	\label{fig:conv_layer}
\end{figure}

As convoluções podem realizar as operações em mais dimensões, é comum usar 2D para imagens, e 3D para vídeos. Neste trabalho apenas trabalhamos com convoluções 1D.\\

\subsubsection{UNET\label{se:unet_sec}}

Um desenho especial de \gls{CNN}, normalmente usando em modelação de imagens, e primeiro proposto em \cite{Shelhamer2014}, a arquitectura UNET passa por criar uma rede de expansão dos filtros, usando convoluções, e de seguida uma rede de contracção dos mesmo, até aos tamanhos pretendidos.\\
Nas suas ligações UNET junta informação de filtros passados (não de nível temporal mas de rede neuronal) para realçar informação já trabalhada, e assim identificar padrões de vários contextos diferentes.\\
É chamada assim pois é uma rede (NET) que forma um U na sua expansão, contracção e ligações entre estes.\\
Em cada camada de encoding vai usando convulucões para criar novos filtros e diminuir a dimensionalidade, enquanto que na fase de decoding vai usar convoluções para aumentar a dimensionalidade e diminuir o número de filtros, adicionando a camada decoder de tamanho análogo.\\

\begin{figure}[H]
	\centering
	\resizebox{\linewidth}{!}{\begin{tikzpicture}[ node distance = 2cm, auto, block/.style={ rectangle, draw, align=center, minimum width=2cm, minimum height=1cm }, line/.style={ draw, -latex' } ]
    % Encoder (Contracting Path)
    \node [block] (input) {Input};
    \node [block, below right=of input, xshift=-2cm] (enclayer1) {Enconding1};
    \node [block, below right=of enclayer1, xshift=-1.8cm] (enclayer2) {Enconding2};
    \node [block, below right=of enclayer2, xshift=-1.6cm] (enclayer3) {Enconding3};
    % (None, 168, 18) 
    
    \node [block, below right=of enclayer3] (up1) {Enconding4};
    
    % Decoder (Expanding Path)
    \node [block, above right=of up1] (declayer1) {Decoding1};
    \node [block, above right=of declayer1, xshift=-1.6cm] (declayer2) {Decoding2};
    \node [block, above right=of declayer2, xshift=-1.8cm] (declayer3) {Decoding3};
    \node [block, above right=of declayer3, xshift=-2cm] (output) {Output};
    
    % Skip Connection
    % \draw [line] (pool1) -- ++(0,-1) -| (up1);
    
    % Connections
    \draw [line] (input) -- (enclayer1);
    \draw [line] (enclayer1) -- (enclayer2);
    \draw [line] (enclayer2) -- (enclayer3);
    \draw [line] (enclayer3) -- (up1);
    \draw [line] (up1) -- (declayer1);
    
    
    \draw [line] (declayer1) -- (declayer2);
    \draw [line] (declayer2) -- (declayer3);
    \draw [line] (declayer3) -- (output);
    
    
    \draw [line] (input) -- (output);
    \draw [line] (enclayer1) -- (declayer3);
    \draw [line] (enclayer2) -- (declayer2);
    \draw [line] (enclayer3) -- (declayer1);
    
    
    \end{tikzpicture}}
	\caption{Ilustração uma rede UNET.}
	\label{fig:unet_graph}
\end{figure}


\subsection{RNN\label{se:rnn_sec}}

As Redes Neuronais Recorrentes (RNN) são projetadas para processar sequências de dados, onde a ordem dos elementos é importante. Elas funcionam passando informações de um neurónio para outro em uma cadeia, o que permite que cada neurónio seja influenciado pelo estado anterior da rede.\\
Isso é feito através de loops internos que permitem à rede "lembrar" informações de etapas anteriores. No entanto, as RNNs enfrentam dificuldades ao tentar lembrar informações de longo prazo, devido ao problema conhecido como desvanecimento do gradiente, onde os gradientes se tornam muito pequenos e impedem a atualização eficaz dos pesos da rede.\\

\subsubsection{LSTM\label{se:lstms_sec}}

As redes \gls{LSTM} são um tipo especial de RNN projetado para superar os problemas de memória de longo prazo encontrados nas RNNs. Isto é conseguido através de uma estrutura de célula que mantém informações ao longo do tempo, permitindo que a rede lembre detalhes importantes mesmo após muitos passos no tempo.\\
As \gls{LSTM}s usam mecanismos de portão para controlar o fluxo de informações, permitindo que elas ignorem informações irrelevantes e mantenham as informações relevantes. Isso torna-as particularmente eficazes em tarefas que exigem o entendimento de dependências de longo prazo em dados sequenciais.\\


O uso de \gls{LSTM} para previsões é uma área comum, mas aqui é seguido através das ideas partilhas em \cite{Hewamalage2021}, e reforçado pelo uso em previsões energéticas demonstados em \cite{Costa2022} \\


\subsection{Transformer\label{se:transformer_sec}}

Os Transformers são um tipo de arquitetura de modelo que utiliza mecanismos de atenção para pesar a importância de diferentes partes de um dado de entrada, primeiro apresentado em \cite{Vaswani2017}.\\
Em vez de processar os dados sequencialmente, como as RNNs, os Transformers processam todos os elementos do dado de entrada simultaneamente. Isso é feito através de um mecanismo de atenção que calcula uma pontuação de atenção para cada par de elementos no dado de entrada, indicando quão relevante um elemento é para o outro. Essas pontuações de atenção são então usadas para ponderar a contribuição de cada elemento ao resultado final. \\
Isso permite aos Transformers capturar dependências de longo alcance nos dados de forma eficiente, tornando-os extremamente eficazes para tarefas de processamento de linguagem natural, como tradução automática e sumarização de texto.\\
% TODO: ref para cahtgpt e dall-e e assim
Este tipo de desenho é a base para os modelos generativos mais conhecidos como o chatGPT para linguagem ou o Dall-E para imagens.\\



