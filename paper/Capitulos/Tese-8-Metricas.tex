\chapter{Métricas}

As métricas utilizadas serviram maioritariamente dois propósitos, com valorizações distintas na escolha de melhores modelos. \\
O primeiro intuito é de estudo de cada modelo, utilizando as métricas comuns de regressão linear, comparando os valores reais com os valores das previsões.\\
Outro objectivo das métricas aplicadas e este mais relevante, era o estudo comparativo do desempenho de cada modelo com o modelo de benchmark.\\

\begin{alignat*}{3} 
& t : \text{Valor real.} &\qquad& p : \text{Previsão} &\qquad& n : \text{número de amostras} \\
\end{alignat*}


\section{Métricas de modelo}

\bigskip
RMSE - Root Mean Squared Error \\

\begin{equation} \label{eq:rmse} 
    RMSE = \sqrt{\frac{1}{n} \sum_{i=1}^{n}(t_i - p_i)^2} 
\end{equation}
\smallskip

Métrica comum em problemas de regressão, dando mais peso a erros maiores, mas retorna um valor que pode ser diretamente comparado ao valor em estudo. Neste caso podemos considerar que o RMSE representa o erro quadrático em MWh. \\
\bigskip
SAE - Sum Abs Error \\


\begin{equation} \label{eq:sae} 
    SAE = \sum_{i=1}^{n}\left|t_i - p_i \right|
\end{equation}
\smallskip

Este simboliza a soma absoluta de todos os erros, dentro da janela temporal em questão. Que representa a quantidade total da energia alocada/não alocada em erro, este é também a soma das duas próximas métricas.\\
\bigskip
AllocF - Alocação em Falta \\

\begin{equation} \label{eq:allocf} 
    AllocF = 
    \begin{cases} 
        0 & , \text{se } p \geq t \\
        t - p  & , \text{se } p < t \\
    \end{cases} 
\end{equation}
\smallskip

Representa a soma total de toda a energia que faltou ser alocada. \\
\bigskip
AllocD - Alocação em Demasia \\

\begin{equation} \label{eq:allocd} 
    AllocD = 
    \begin{cases} 
        0 & , \text{se } p \leq t \\
        p - t  & , \text{se } p > t \\
    \end{cases} 
\end{equation}
\smallskip

Representa a soma total de toda a energia que for alocada em demasia. \\

\section{Métricas de comparação modelo/benchmark}

GPD - Ganho Percentual de Desempenho

\begin{equation} \label{eq:gpd} 
    GPD = \frac{SAE_{benchmark} - SAE_{modelo}}{SAE_{benchmark}} \times 100
\end{equation}
\smallskip

O Ganho Percentual de Desempenho é a nossa métrica basilar. Representa, dentro da janela temporal de validação, a percentagem de melhoria do modelo em relação ao benchmark. Isto é representa a percentagem de energia que foi melhor alocada que o modelo. Onde 100\% representa uma melhoria perfeita, onde o modelo não tem erro. E 0\% representa nenhuma melhoria, ou seja, igual ao benchmark. Pode também ter valores negativos, que representa a percentagem em que o modelo é pior que o benchmark, podendo ser infinitamente pior.\\
Esta métrica é representativa da totalidade de energia, tanto alocado como em falta. \\
As próximas métricas são variações desta que ajudam a escolher o melhor modelo em cada experiência, conseguindo distinguir entre alocação em falta e em demasia.\\
\bigskip
GPDF - Ganho Percentual de Desempenho (alocação em) Falta\\

\begin{equation} \label{eq:gpdf} 
    GPDF = \frac{AllocF_{benchmark} - AllocF_{modelo}}{AllocF_{benchmark}} \times 100
\end{equation}
\smallskip

O mesmo que o GPD mas apenas para as somas totais de alocação em falta.\\
\bigskip
GPDD - Ganho Percentual de Desempenho (alocação em) Demasia\\

\begin{equation} \label{eq:gpdd} 
    GPD-D = \frac{AllocD_{benchmark} - AllocD_{modelo}}{AllocD_{benchmark}} \times 100
\end{equation}
\smallskip

O mesmo que o GPD mas apenas para as somas totais de alocação em falta.\\
\bigskip
GPD Norm - Ganho Percentual de Desempenho Normalizado \\

\begin{equation} \label{eq:gpdnorm} 
    GPD Norm = \frac{GPDF + GPDD}{2}
\end{equation}
\smallskip

Aqui o GPD é calculado a partir dos já calculados GPDF e GPDD, sendo a média destes. Desta maneira conseguimos ter uma percentagem de melhoria em relação ao benchmark, onde a melhoria da alocação em demasia e a melhoria da alocação em falta têm o mesmo peso. \\

\bigskip
GPD $Norm^{2}$ - Ganho Percentual de Desempenho Normalizado (negativos) Quadrado

 GPD $Norm^{2}$=GPD norm mas os GPD são ao quadrado se forem negativos  


 \begin{equation} \label{eq:gpdnorm2} 
    GPD Norm^{2} = 
    \begin{cases} 
        GPD Norm & , \text{se } GPDF \text{ }\&\text{ } GPDD \geq 0 \\
        \frac{GPDF^{2} + GPDD}{2} & , \text{se } GPDF  < 0 \\
        \frac{GPDF + GPDD^{2}}{2} & , \text{se } GPDD < 0 \\
    \end{cases} 
\end{equation}
\smallskip


O mesmo que GPD norm mas os GPDF ou GPFD que sejam negativos o seu valor é ao quadrado e mantendo-se negativo. Serve para dar mais peso aos valores negativos, assim não tendo GPD altos mesmo se um dos GPD for negativo (pior que o benchmark). \\
Esta métrica é a principal na escolha do melhor modelo em cada experiência visto manter ambos os GPD mas penalizando se algum deles é negativo. \\
\bigskip
GPD Positivo  - Ganho Percentual de Desempenho Positivo

 \begin{equation} \label{eq:gpdpositivo} 
    GPD Positivo = 
    \begin{cases} 
        GPD & , \text{se } GPDF \text{ }\&\text{ } GPDD \geq 0 \\
        0 & , \text{se } GPDF \text{ }\|\text{ } GPDD < 0 \\
    \end{cases} 
\end{equation}
\smallskip


Esta métrica é igual a GPD mas apenas nos casos em que ambos são positivos, logo o modelo é melhor que o benchmark, senão é zero. Serve para medir o GPD real, mas apenas nos casos em que o modelo já surpassa o benchmark.\\
