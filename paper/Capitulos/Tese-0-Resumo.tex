\chapter{Resumo}
\justifying

A crescente penetração de fontes de energia renováveis variáveis no tempo, \gls{vRES}, no sistema eléctrico, como a solar e a eólica, está a transformar significativamente os mercados de eletricidade, devido à sua natureza intermitente e imprevisível. Esta imprevisibilidade torna as previsões de produção e consumo de energia mais desafiantes, especialmente porque os mercados de eletricidade fecham entre 1 e 37 horas antes da entrega real da energia, o que pode resultar em discrepâncias entre a energia contratada e a energia realmente produzida ou consumida. Manter o equilíbrio entre a oferta e a procura em tempo real é vital para a segurança e estabilidade da rede, função que recai principalmente sobre os operadores de redes de transporte (\gls{TSO}).\par
Os \gls{TSO} utilizam mercados de reserva de energia, onde adquirem de forma simétrica potência secundária ascendente e descendente, com base em previsões de procura para as horas subsequentes. No entanto, essa abordagem simétrica pode ser ineficaz diante das flutuações das energias renováveis, levando à necessidade de ajustes mais dinâmicos e precisos.\par
Neste contexto, o presente trabalho propõe primeiramente um estudo do parâmetro variável por hora da fórmula do \gls{TSO} português para a previsão de reserva necessária ($\rho$), onde usando os dados históricos horários no período 2010 a 2019, é calculado o $\rho$ horário que resulte num menor erro médio na previsão, onde são obtidos resultados com erro inferior a 5\%.\par
Este trabalho propõe de seguida um modelo baseado em técnicas de \textit{machine learning} para calcular dinamicamente as reservas de potência secundária, tanto ascendente como descendente, integrando despachos programados para o dia seguinte, previsões de produção de \gls{vRES}, a procura prevista e outras características operacionais.\par
Utilizando dados operacionais abertos do \gls{TSO} espanhol, o modelo foi treinado com dados no período de 2014 a 2023,  e validado com dados de referência de 2019 a 2022. A metodologia proposta demonstra uma melhoria significativa na utilização das reservas de potência secundária, com um aumento de aproximadamente 43\% na eficiência das reservas ascendentes e cerca de 36\% nas reservas descendentes. Este avanço contribui para uma gestão mais eficiente e equilibrada do sistema elétrico, especialmente em cenários com elevada penetração de \gls{vRES}.\par




\vspace{0.5cm} %adiciona um espaço de 0.5cm entre o texto e as palavras chave.

\keywordsP{sistemas de reserva, mercados de energia, redes neuronais, previsões}
%reparem que # necessita de um \ para que o latex o interprete correctamente como um caracter especial. Isto também acontece com % por exemplo.