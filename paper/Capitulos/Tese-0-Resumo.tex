\chapter{Resumo}

A penetração na rede e o investimento crescente em fontes de energia renováveis variáveis estão a mudar a forma como os mercados da eletricidade se comportam. Os intervenientes baseiam-se em previsões para participar nos mercados, que fecham entre 1 e 37 horas antes das transacções de energia em tempo real.\\
Normalmente, os operadores de redes de transmissão (TSO) utilizam uma aquisição simétrica para a atribuição de potência secundária ascendente e descendente com base na procura prevista para os consumos horários do dia seguinte. Este trabalho utiliza técnicas de \emph{machine learning} que calculam dinamicamente as reservas de energia secundária a montante e a jusante, utilizando os despachos programados para o dia seguinte e os despachos esperados de energias renováveis variáveis, a procura e outras caraterísticas.\\
O modelo introduz dados abertos operacionais do TSO espanhol de 2014 a 2023 para treino. Os dados de referência e de teste datam de 2019 a 2022. A metodologia proposta melhora a utilização da potência reservada secundária descendente em cerca de 36\% e da potência reservada secundária ascendente em quase 43\%. \\


\vspace{0.5cm} %adiciona um espaço de 0.5cm entre o texto e as palavras chave.

\keywordsP{sistemas de reserva, mercados de energia, redes neuronais, previsões}
%reparem que # necessita de um \ para que o latex o interprete correctamente como um caracter especial. Isto também acontece com % por exemplo.