\chapter{Contexto e revisão bibliográfica}
\section{Revisão bibliográfica}

A análise de séries temporais é um tema recorrente em pesquisa. Especialmente para previsões.\par
Desde as previsões para mercados de acções \cite{Bhandari2022}, fenómenos meteorológicos \cite{Wang2019}, e especialmente mercados energéticos, onde se quer ter em consideração o impacto das gerações mais voláteis.\par
As energias renováveis, devido à sua natureza, são as produções mais voláteis, logo alvo de estudo ideal para estas tecnologias \cite{Lu2015}, energia eólica \cite{Sun2022}, energia solar \cite{Rajasundrapandiyanleebanon2023}, aplicabilidade dos vários sistemas \cite{Ahmad2020}, procura \cite{Antonopoulos2020}.
Sendo que cada problema já apresenta arquiteturas e soluções diferentes, como a geração de energia fotovoltaica em casas pode ser melhor prevista com \gls{LSTM}\cite{Costa2022} mas também com uso de \gls{SVM}\cite{Meenal2018}. As várias faces destas tecnologias estão optimamente apresentadas em \cite{Benti2023}.\par


Para o estudo de previsões de séries temporais chega a ser o caso se pesquisar primeiramente com \textit{deep learning}, antes de procurar outras soluções. Em \cite{Elsayed} é visto o impacto dessa decisão, e se realmente compensa emergir em \text{machine learning}. O trabalho conclui que modelos simples, com alguma engenharia de atributos inteligente, conseguem competir, ou mesmo passar as qualidades de redes neuronais profundas.\par
Esta conclusão mostra também que por vezes a procura por modelos mais complexos não compensa, e que cada problema/\textit{dataset} deve ter a sua própria investigação e conclusão, consoante a quantidade/qualidade de recursos disponíveis.\par
Mas mesmo sem \textit{machine learning} alguma fórmulas de previsão usadas podem ser melhoradas apenas pela extrapolação de parâmetros a partir dos dados históricos.\par  
No caso do \gls{TSO} portugês, a \gls{REN}, para a previsão de bandas de reserva usa um modelo preditivo baseado na fórmula publicada pela \gls{ENTSO-E}, multiplicando a esta um rácio horário.\par
Esta fórmula já foi alvo de estudo em \cite{Carneiro2016}, onde todos os parâmetros foram testados com os dados históricos, de modo a optimizar os mesmos. Neste trabalho apenas o rácio horário é posto em causa, e onde os valores apresentados em \cite{Carneiro2016} apresentam erros médios por hora na casa dos 25\%, este trabalho apresenta todos os erros médios por hora abaixo dos 5\%. De notar que ambos os métodos são otimizações estatísticas baseadas em dados históricos, e como tal de 2016 a 2023 existe um grande aumento de dados disponíveis.\par
A previsão de bandas de reserva já foi alvo de estudo também através de redes neuronais profundas, como em \cite{miota2023} que estuda o mercado espanhol. Este trabalho propõe prever o custo da alocação e faz usando um elevado número de atributos disponíveis pela \gls{ESIOS}, 32 variáveis. Apresentado bons resultados, este trabalho é um bom indicador que modelos de \text{machine learning} podem trazer melhores previsões que os métodos tradicionais.\par


No âmbito de energias vários trabalhos vieram mostrar que o uso de \textit{machine learning} para previsões energéticas tem aplicabilidade \cite{Stassen} e muitos casos resultados melhores do que usados na indústria corrente. \cite{Ahmad2020} \cite{Antonopoulos2020} \par
Como muitos trabalhos apresentados em \cite{Benti2023}, o uso destas técnicas está a crescer e a produzir frutos. Como concluído neste trabalho as várias arquitecturas e modelos comuns de \textit{machine learning} já foram aplicados em energia, especialmente nas áreas de consumo e produção.\par
No caso do estudo de alocação necessária podemos ver que em \cite{Algarvio2024} houve já uma melhoria de 13\% e 8\% em relação ao método usado pelo TSO, usando fórmulas dinâmicas de alocação, e neste trabalho usando \textit{machine learning} são apresentadas melhorias de alocação em 44\% e 38\%.\par






