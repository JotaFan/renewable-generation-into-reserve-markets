\part[]{}{}

\chapter{Conclusões e sugestões futuras}

Primeiramente podemos ver pela análise estatística \ref{tab:statsres} e aplicando a ideia \cite{Elsayed}, simples modelos estatísticos conseguiriam baixar bastante o erro de previsão, melhor do que o que é utilizado actualmente \ref{tab:benchmarkmetrics}, embora tenham aumentado a necessidade de alocação na reserva terciária.\\
E se considerarmos ainda que os modelos estatísticos apresentados que apresentam estes resultados, utilizam apenas a variável em questão, e não todos os outros atributos, a nível de aplicabilidade já são uma melhoria. \\
Em relação a métodos \textit{machine learning}, com poucos recursos computacionais, conseguimos já modelos que superam o método actual.\\
Com modelos relativamente simples conseguimos melhorias muito grandes na alocação de energia, em relação à alocada actualmente. Estes métodos podem criar grandes ganhos financeiros, e diminuir a quantidade de recursos desperdiçados, logo têm um efeito positivo no mercado de reservas.\\
Os resultados aqui apresentados provam que vários tipos de modelos de \textit{machine learning} conseguem realizar previsões bem mais exactas, e que diminuem os recursos usados. Para uso em mercado real estes podem ser adaptados para responder ao mercado em questão, e ao contrário de uma fórmula, podem ir “aprendendo” e melhorando com o passar do tempo.\\
O que mostra que usando estes métodos dinâmicos podemos sim reduzir as incertezas da penetração das \gls{vRES} na alocação de energia secundária.\\
O futuro da indústria pode passar por este tipo de metodologias. Uma maneira de melhorar ainda mais estes resultados seria o uso de outras variáveis para o modelo. Variáveis essas como os dados de reserva primária, dados meteorológicos e principalmente dados não de \gls{DA} mas reais.\\
Um aumento computacional poderia também ter um aumento significativo nas previsões, usando mais quantidade de dados, usando modelos mais pesados e complexos, mais dados históricos, modelações com dados de mercados diferentes com convergência para o mercado necessário.\\
Outra possibilidade pode ser o uso de \textit{machine learning} para a reparametrização de novas fórmulas baseadas nas já existentes e em uso.\\
Vários caminhos e maneiras podem surgir para aplicar o uso de modelos de \textit{machine learning} em alocação dinâmica de reservas, onde operadores diferentes podem ter arquiteturas e modelos completamente diferentes.
