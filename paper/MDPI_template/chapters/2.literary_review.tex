The increasing integration of \gls{vRES} in power systems has created significant challenges for electricity markets and ancillary services. Traditionally, \gls{TSO} rely on symmetric allocation of upward and downward reserves based on deterministic forecasts of demand. However, with \gls{vRES} like wind and solar introducing substantial variability and unpredictability, these conventional methods have proven inefficient in addressing the real-time balancing needs of modern power grids.\par
Numerous studies highlight the limitations of static reserve procurement methods under high vRES penetration. The ENTSO-E framework \cite{handbook2009policy} outlines standardized methodologies for reserve sizing, such as \gls{FCR} and \gls{aFRR}, but these often fail to adapt dynamically to changing system conditions. In Portugal, for example, the secondary reserve allocation formula used by the \gls{TSO} employs a fixed ratio applied to expected demand, resulting in excessive allocation and energy waste %[2]
. Similar inefficiencies are observed in the Spanish market, where reserve procurement lacks symmetry and adaptability to \gls{vRES} production.\par %[3]
Dynamic procurement of secondary reserves has been proposed as a solution to address these inefficiencies \cite{Algarvio2022}, with an improvement in 13\% and 8\% for up and down secondary capacities. By incorporating real-time or near real-time forecasts of demand and renewable generation, dynamic methodologies aim to optimize reserve allocation, reducing both operational costs and resource wastage. Machine learning techniques have emerged as a powerful tool to support this transition. Studies such as \cite{DeVos2019} and \cite{Kruse2022} demonstrate the potential of predictive models to estimate reserve needs with greater accuracy, leading to significant reductions in over-procurement.\par
The literature also underscores the importance of enhancing forecast accuracy for \gls{vRES} generation and consumption patterns. Traditional statistical models, including ARMA and ARIMA, have been widely used for time series forecasting. However, recent advancements in machine learning, such as \gls{LSTM} networks and \gls{CNN}, have shown superior performance in capturing the nonlinear and temporal characteristics of renewable energy data \cite{Benti2023}. These models can adapt to complex patterns and improve prediction accuracy, enabling more efficient management of reserves.\par
In summary, the literature identifies three key areas of focus: the inefficiency of static reserve allocation methods, the potential of machine learning to improve forecasting accuracy, and the need for market design adaptations to support dynamic reserve procurement. This paper builds upon these insights by applying machine learning techniques to optimize secondary reserve allocation in the Spanish electricity market, addressing both forecast uncertainty and market inefficiencies.\par