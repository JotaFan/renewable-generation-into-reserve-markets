The dynamic procurement of secondary reserves represents a significant step forward in addressing the inefficiencies inherent in traditional static allocation methods. Unlike static reserve procurement, which relies on fixed ratios or historical averages, dynamic approaches incorporate real-time forecasts and system conditions to adjust reserve requirements. This adaptability is particularly critical for modern electricity systems with high penetration of variable renewable energy sources (vRES), where forecasting uncertainty and rapid changes in generation output challenge grid stability.

Dynamic reserve procurement involves estimating upward and downward reserve needs based on the expected deviations between day-ahead scheduled generation and real-time demand. By leveraging advanced forecasting tools, such as machine learning models, it becomes possible to predict these deviations with greater accuracy, optimizing the allocation of secondary reserves. Historical data on vRES production, system demand, and grid imbalances serve as inputs to these models, allowing the identification of patterns and trends that inform reserve procurement decisions.

Machine learning techniques, including Long Short-Term Memory (LSTM) networks and other time-series forecasting models, have demonstrated significant potential for improving reserve predictions. These models can capture the nonlinear and temporal dependencies present in renewable energy data, outperforming traditional statistical methods such as ARIMA. By incorporating real-time weather forecasts, generation data, and demand profiles, dynamic approaches ensure that reserve procurement aligns more closely with actual system needs, reducing both over-procurement and under-procurement of reserves.

The dynamic approach also allows for asymmetrical procurement of upward and downward reserves, which is particularly relevant in systems with variable renewable generation. For instance, during periods of high solar generation, upward reserves may be less critical, whereas downward reserves become essential to accommodate excess production. Conversely, during low renewable output, upward reserves are prioritized to address potential generation shortfalls.

In summary, dynamic procurement of secondary reserves offers a more efficient and adaptive solution to balancing challenges in modern electricity systems. By leveraging machine learning techniques and real-time forecasts, this approach enhances reserve allocation, reduces operational costs, and