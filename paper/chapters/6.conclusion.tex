The results of the case study validate the effectiveness of machine learning techniques in improving the accuracy of reserve forecasts and optimizing reserve allocation. By dynamically adjusting upward and downward reserves based on real-time forecasts, the proposed methodology addresses the inefficiencies of static procurement methods. The observed cost savings and improved reserve utilization demonstrate the practical benefits of this approach for systems with high renewable penetration.

Additionally, the case study highlights the potential for asymmetrical reserve procurement to better reflect system needs, particularly during periods of extreme renewable generation variability. The integration of weather forecasts into the predictive models further enhances their reliability, ensuring that reserve procurement decisions are informed by real-time conditions.

In conclusion, the case study illustrates that dynamic reserve procurement, supported by machine learning techniques, can significantly improve the efficiency and cost-effectiveness of balancing services in modern electricity systems. These findings provide a strong foundation for further research and potential implementation in other power markets with similar challenges.
