The European Union's energy and climate goals for 2030 and 2050 emphasize the transition to a carbon-neutral energy system, driven by the large-scale integration of \gls{vRES}, such as wind and \gls{PV} technologies \cite{Perissi2022}. While \gls{vRES} are critical for achieving sustainability targets, such as \gls{SDG} 7 (Affordable and Clean Energy) and \gls{SDG} 13 (Climate Action), their stochastic and intermittent nature poses significant challenges to power system operations, particularly in balancing energy supply and demand efficiently.\cite{Ocker2017}\cite{Frade2019_wind}  \par
The increasing penetration of \gls{vRES} introduces greater uncertainty into energy markets, particularly in \gls{DA}  forecasts, which are essential for the allocation of secondary reserves. These reserves, procured to address real-time imbalances between generation and consumption, often suffer from inefficient allocation methodologies. \gls{DA} predictions frequently diverge from real-time conditions, leading to both over-allocation and under-allocation of reserves. This inefficiency not only results in higher operational costs but also compromises the optimal utilization of resources, thereby undermining the economic and energy efficiency of the system.\par
This paper focuses on enhancing the accuracy of \gls{DA}  forecasts for secondary reserve allocation, addressing the inefficiencies caused by \gls{vRES} uncertainty. By leveraging machine learning techniques, this work develops predictive models that incorporate historical data on \gls{vRES} generation, demand patterns, and system behavior. The objective is to dynamically adjust reserve allocations, ensuring that grid stability is maintained while minimizing excess reserve procurement.\par
The work present on this paper analyzes the benefits of using of machine learning techniques for an idependent, up and down, forecast of capacity procurement of secondary reserves. Publicly available operational data from the Spanish \gls{TSO} was utilized, ensuring the replicability of the analysis. Typically, \gls{TSO}s rely on bilateral agreements to acquire additional reserves, which can drive up costs. Analyzing the underutilization of secondary reserves and the frequent need for extraordinary capacity in Portugal and Spain highlights the inefficiencies in current methods for determining reserve requirements. \cite{EstanqueiroD5} \par
The remainder of this paper is structured as follows. Section 2 provides an overview of wholesale energy markets and reserve systems, highlighting existing inefficiencies. Section 3 outlines the proposed methodology for dynamic reserve allocation using machine learning techniques. Section 4 presents case studies and evaluates the performance of the developed models. Finally, Section 5 summarizes the findings and discusses the implications for future energy systems.\par

% \par
% \par
% ++++++++++++++++++++++++++++++++++++++++++++++++++++++++++++++++++++++++++++++++++++++++++++++++++++++++++++++++++++++++++++++++++++++
% The growing integration of variable renewable energy sources (vRES), such as wind and solar photovoltaic (PV) technologies, is transforming electricity systems across Europe. While \gls{vRES} play a crucial role in achieving carbon neutrality goals and addressing climate targets set for 2030 and 2050, their intermittent and stochastic nature introduces significant operational challenges. These challenges are particularly evident in the accurate forecasting of energy generation and consumption, which is critical for efficient reserve management.\par
% Transmission System Operators (TSOs) rely on \gls{DA}  markets to procure secondary reserves, balancing supply and demand in real time. However, the volatility of \gls{vRES} production often leads to discrepancies between forecasted and actual conditions, resulting in over-allocation or under-allocation of reserves. Inaccurate allocation increases operational costs, reduces system efficiency, and hampers the optimal use of resources. Addressing this issue is essential for improving grid stability and ensuring the cost-effectiveness of energy markets.\par
% Dynamic allocation of secondary reserves has emerged as a promising solution to mitigate these inefficiencies, particularly when supported by advanced predictive tools. Recent studies have demonstrated the potential of machine learning techniques to enhance forecast accuracy, enabling better alignment of reserve procurement with real-time system needs. By analyzing historical operational data and identifying patterns in \gls{vRES} generation, demand, and system behavior, machine learning models can significantly reduce forecasting errors and improve the efficiency of reserve allocation.\par
% This paper proposes a machine learning-based methodology for the dynamic procurement of secondary reserves, using historical and operational data from the Spanish electricity market as a case study. The approach aims to optimize reserve allocation, minimize inefficiencies caused by \gls{vRES} variability, and improve the overall management of balancing services. The results demonstrate the potential of machine learning techniques to enhance system reliability and reduce unnecessary reserve costs in markets with high renewable penetration.\par
% The remainder of this paper is structured as follows. Section II reviews the literature on electricity markets and reserve systems. Section III introduces the proposed dynamic allocation methodology. Section IV presents the case study and evaluates the results. Finally, Section V concludes with a discussion on the implications and potential for future work.\par