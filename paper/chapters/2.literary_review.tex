The increasing integration of \gls{vRES} in power systems has created significant challenges for electricity markets and ancillary services. Traditionally, \gls{TSO}s rely on symmetric allocation of upward and downward reserves based on deterministic forecasts of demand. However, with \gls{vRES} like wind and solar introducing substantial variability and unpredictability, these conventional methods have proven inefficient in addressing the real-time balancing needs of modern power systems.\par
The \gls{ENTSO-E} framework \cite{handbook2009policy} outlines standardized methodologies for reserve sizing, which are \gls{FCR}, \gls{aFRR}, and \gls{mFRR}, that often fail to adapt dynamically to changing system conditions. In Portugal, for example, the secondary reserve allocation formula used by the \gls{TSO} employs a fixed ratio applied to expected demand, resulting in excessive allocation and energy waste \cite{Frade:19c,Perissi2022}. Similar inefficiencies are observed in the Spanish procurement of secondary (\gls{aFRR}) reserve capacity, where it lacks symmetry and adaptability to \gls{vRES} production \cite{Algarvio:24,Cardo-Miota:23}.
%
\textcolor[rgb]{0,0,0.5}{Indeed, after the European gas crisis, the Spanish TSO changed the allocation of reserve capacity in the middle of 2022 to a more conservative, practically symmetrical procurement. The procurement of secondary capacity increased to a similar level across the day and is typical of peak procurement. The procurement of secondary down capacity significantly increased to values close to up capacity. Since the middle of 2022, the procurement of secondary capacity has been less dynamic and volatile throughout the day, decreasing the usage percentage of secondary capacity \cite{Algarvio:24,Cardo-Miota:23}.}
%
 %[3]

Numerous studies highlight the limitations of static reserve procurement methods under high vRES penetration.
%
The majority of the literature focuses on using historical data to compute the procurement of secondary reserves \cite{Knorr:19,Frade2019_market,Papavasiliou:21}
%
Operational methodologies are needed to be used by TSOs.
%
Dynamic procurement of secondary reserves has been proposed to address these inefficiencies, with an improvement of 13\% and 8\% for up and down secondary capacities by 2022 \cite{Algarvio2024}. By incorporating real-time or near real-time forecasts of demand and renewable generation, dynamic methodologies aim to optimize reserve allocation, reducing operational costs and resource wastage.
%
Furthermore, five different mechanisms for procuring secondary power in Spain were analyzed for the Spanish power system by 2030, with renewable penetrations higher than 70\% \cite{Algarvio:24}. The dynamic procurement methodology proposed in this study enables cost reductions for Spanish secondary power by 27\% when using block bids and 34\% when using flexible bids. These results highlight the increasing importance of dynamic reserve procurements with the rising uncertainty of higher penetrations of \gls{vRES}.  

Machine learning techniques have emerged as a powerful tool to support this transition. Studies such as \cite{DeVos2019} and \cite{Kruse2022} demonstrate the potential of predictive models to estimate reserve needs with greater accuracy, leading to significant reductions in over-procurement. 
De Vos et \textit{al.} proposed a machine learning approach to estimate imbalance uncertainty, to adjust the size of Belgium's operating reserves from an annual to a daily basis, resulting in a 5\% reduction \cite{DeVos2019}. 

Kruse, Sch\"{a}fer, and Witthaut introduced an ex-post machine learning method to determine the appropriate size for secondary reserves. They identified key variables that most accurately estimate errors, essential for detecting when secondary control is activated. Additionally, to enhance the efficiency of cross-border capacity allocation in balancing service exchanges, legislation recommended coupling balancing mechanisms, as demonstrated in the Nordpool market \cite{Frade:19c,Khodadadi:20}.\par
Cardo-Miota et \textit{al.} identified the benefit of using machine learning techniques to predict the prices of \gls{aFRR} in Spain \cite{Cardo-Miota:23}.
The literature also underscores the importance of enhancing forecast accuracy for \gls{vRES} generation and consumption patterns. Traditional statistical models, including ARMA and ARIMA, have been widely used for time series forecasting. However, recent advancements in machine learning, such as \gls{LSTM} networks and \gls{CNN}, have shown superior performance in capturing the nonlinear and temporal characteristics of renewable energy data \cite{Couto:21,Benti2023}. These models can adapt to complex patterns and improve prediction accuracy, enabling more efficient management of reserves.\par

\textcolor[rgb]{0,0,0.5}{So, most studies use machine learning techniques to improve the market outcomes of their participants. Moreover, on the subject of the presented article, most studies use historical data analysis to define the size of the secondary capacity. This study presents an operational open-source methodology that can be tested and adapted by \gls{TSO}s. 
Against this background, while the literature presents rich forecasting approaches to accurately predict \gls{vRES} production using machine learning techniques, \gls{TSO}s still use deterministic conservative methodologies to procure secondary capacity. Indeed, vRES and demand-side players are the main sources of real-time deviations, and forecast methodologies are being improved to reduce the penalties they pay for imbalances. So, \gls{TSO}s shall also consider the stochastic nature of those imbalances while predicting the secondary energy needs.}

In summary, has been identified in the literature three key areas of focus: the inefficiency of static reserve allocation methods, the potential of machine learning to improve forecasting accuracy, and the need for market design adaptations to support dynamic reserve procurement. This paper builds upon these insights by applying machine learning techniques to optimize secondary reserve allocation, addressing forecast uncertainty and market inefficiencies.\par