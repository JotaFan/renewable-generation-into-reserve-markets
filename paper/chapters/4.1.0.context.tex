
\subsection{Previsão de Necessidades de Reservas \label{se:pred_impact_vres}}

A previsão das necessidades de reservas de frequência é uma componente essencial na gestão eficiente dos sistemas eléctricos, especialmente num cenário de crescente penetração das \gls{vRES}.\par
O uso de técnicas de \textit{machine learning} tem sido explorado como uma solução promissora para melhorar essas previsões. Estes modelos podem analisar grandes volumes de dados, identificar padrões complexos e ajustar previsões em tempo real, considerando factores como mudanças nas condições meteorológicas e padrões de consumo de energia. Ao incorporar a variabilidade das \gls{vRES} nos modelos de previsão, é possível reduzir a incerteza e melhorar a alocação das reservas de frequência, resultando numa operação mais eficiente do sistema eléctrico.\par
Outro factor crítico na previsão das necessidades de reservas de frequência é a coordenação entre diferentes mercados e operadores de sistemas. A harmonização dos mercados europeus de balanço, incluindo a padronização das regras de oferta, leilão e remuneração, pode facilitar a integração das \gls{vRES} e melhorar a eficiência geral do sistema. Com regras claras e uniformes, os produtores de energia renovável têm maior incentivo para participar activamente dos mercados de reservas, fornecendo capacidade adicional para apoiar a estabilidade da rede. Esta questão é particularmente relevante em mercados onde as \gls{vRES} ainda enfrentam barreiras significativas para a participação, como regras complexas de licitação ou altos requisitos de capacidade mínima para participação.\par
Apesar dos avanços na previsão de necessidades de reservas, ainda existem desafios consideráveis. A precisão das previsões pode ser limitada pela qualidade dos dados disponíveis, bem como pela capacidade dos modelos de capturar todas as variáveis relevantes que afetam a operação da rede. Além disso, a crescente interconexão dos sistemas eléctricos e o aumento da troca de energia entre países exigem uma abordagem coordenada e colaborativa para a previsão de reservas, considerando tanto as condições locais como as condições regionais.\par
O desenvolvimento contínuo de técnicas avançadas de previsão e a integração de soluções baseadas em dados serão fundamentais para enfrentar esses desafios. À medida que mais dados históricos se tornam disponíveis e os modelos de previsão evoluem, espera-se que a gestão das reservas de frequência se torne cada vez mais eficiente, contribuindo para um sistema eléctrico mais resiliente e capaz de integrar altos níveis de Tal desenvolvimento, não apenas reduzirá os custos operacionais, mas também contribuirá para a segurança energética e para a transição para um sistema energético mais sustentável.\par

\subsubsection{Previsão de Banda Secundária no Mercado Ibérico de Electricidade \label{se:pred_mibel}}
A nível Europeu a \gls{ENTSO-E} providencia várias metodologias para o dimensionamento das reservas de controlo descritas em \cite{handbook2009policy}. A quantidade mínima recomendada de alocação necessária para a reserva de controlo secundária pode ser descrita da seguinte forma:\par

\begin{equation} \label{eq:BRENTSOE} 
    BR = \sqrt{a \times  L_{max} + b^{2}} - b 
\end{equation}
onde:
\begin{itemize}
  \item $BR$: Banda de Reserva de regulação secundária mínima necessária (MW).
  \item $a$ e $b$: Coeficientes empiricos, $a$=10MW e $b$=150MW .
  \item $L_{max}$: Consumo máximo antecipado (MW).
\end{itemize}


\paragraph{Portugal \label{se:prev_portugal}}
\text{ }  \par

No mercado português para dimensionar a \gls{aFRR} a \gls{REN} utiliza por base a equação \ref{eq:BRENTSOE} multiplicando um parâmetro horário, $\rho$:

\begin{equation} \label{eq:BRREN} 
    BR = \rho \times \sqrt{a \times  L_{max} + b^{2}} - b 
\end{equation}
onde:
\begin{itemize}
  \item $\rho$: Paramêtro horário.
\end{itemize}


Na equação \ref{eq:BRREN} BR equivale à banda a subir, sendo a banda a descer metade da banda a subir. De notar que em \cite{Carneiro2016} BR é a banda de reserva, que equivale à soma da banda a subir e banda a descer, onde aí é sempre considerado que banda a subir são $\frac{2}{3}$ da Banda de Reserva total e a banda a descer é o restante $\frac{1}{3}$.\par
Este método de cálculo permite manter as reservas a corresponder às necessidades do sistema, mas têm uma uma alocação  em excesso. Podemos verificar que no período 2013 a 2023, inclusive, as médias por hora têm cerca de 437\% de alocação em excesso, o que corresponde, em média, a cerca de 221 MWh desperdiçados a cada hora. \par

\begin{table}[H]
	\centering
    \caption{Média das Bandas Alocada e Usada (REN)}    
    \resizebox{!}{!}{\begin{tabular}{rrrr}
\toprule
Banda de Reserva Alocada & Banda Reserva Activada & erro & erro \% \\
\midrule
271.57 & 50.53 & 221.04 & 437.43 \\
\bottomrule
\end{tabular}
}
    \label{tab:media_bandas_pt}
    \end{table}

Estando actualmente o \gls{TSO} português a utilizar esta fórmula, e a obter estes resultados, este é um bom caso de estudo de optimização dos parâmetros da fórmula. Sendo que \textit{a} e \textit{b} são dados pela entidade europeia, propõe-se o estudo do parâmetro horário de modo a corresponder a banda de reserva calculada ao consumo real.\par

\paragraph{Espanha \label{se:prev_espanha}}
\text{ }  \par

No mercado espanhol não encontramos directivas de uso de uma fórmula como no caso português. Nem encontramos uma simetria directa entre as bandas a subir e a descer. Contudo, podemos verificar que a média horária dentro do mesmo período apresenta disparidades ainda maiores em quantidade média de energia alocada desperdiçada.\par

\begin{table}[H]
	\centering
    \caption{Média das Bandas Alocada e Usada (REE)}    
    \resizebox{!}{!}{\begin{tabular}{lrrrr}
\toprule
 & Banda de Reserva Alocada & Banda Reserva Activada & erro & erro \% \\
\midrule
Banda a Subir & 662.94 & 158.10 & 504.84 & 319.32 \\
Banda a Descer & 549.27 & 168.20 & 381.07 & 226.55 \\
\bottomrule
\end{tabular}
}
    \label{tab:media_bandas_es}
    \end{table}


Como temos uma boa quantidade de dados históricos e uma falta de definição e formulação exacta da necessidade, o caso espanhol é um bom caso de estudo para previsões usando \textit{machine learning}.\par


