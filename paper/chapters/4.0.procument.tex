

This study proposes a dynamic procurement based on machine learning techniques trained with historical hourly data. With custom made model architectures



\subsection{Methodology Implementation}

The methodology aplied was a "brute force" choosing of better model, which can lead to better fine tuning results than a more complex architecture as shown in \cite{Liu2022}. With multiple model related variables in study:\par

\begin{table}[H] 
    \caption{This is a table caption. Tables should be placed in the main text near to the first time they are~cited.\label{tab1}}
    \newcolumntype{C}{>{\centering\arraybackslash}X}
    \begin{tabularx}{\textwidth}{CC}
    \toprule
    \textbf{Variables} & \textbf{options} \\
        \midrule
            \multirow[m]{5}{*}{architecture}	& CNN\\
                                                & LSTM\\
                                                & RNN\\
                                                & UNET\\
                                                & Transformer\\
        \midrule
            \multirow[m]{2}{*}{Advance Loss function}	& Mirror Weights\\
                                                & N/A \\
        \midrule
            \multirow[m]{3}{*}{Loss function}	& \gls{MAE}\\
                                                & \gls{MSE}\\
                                                & \gls{MSLE}\\
        \midrule
            \multirow[m]{3}{*}{Activation}	& linear\\
                                                & relu\\
                                                & gelu\\
        \midrule
            \multirow[m]{3}{*}{Weights}	& Temporal\\
                                                & Distance to mean \\
                                                & No Weights\\    
    \bottomrule
    \end{tabularx}
    % \noindent{\footnotesize{\textsuperscript{1} Tables may have a footer.}}
\end{table}

Where each of the model variables in study where a layer of training, giving the best model within that scope we would go to the next variable with the given best option so far. Going back and forward as to not loose best possible choices.\par


\begin{figure}[H]
	\centering
	\resizebox{\linewidth}{!}{\begin{tikzpicture}[ node distance = 1cm, auto, block/.style={ rectangle, draw, align=center, minimum width=1cm, minimum height=1cm }, line/.style={ draw, -latex' } ]
    % Encoder (Contracting Path)
    \node [block] (archs) {Architectures};
    \node [block,  right=of archs] (advance_loss_function) {Advance Loss Function};
    \node [block,  below=of advance_loss_function] (loss_function) {Loss Functions};
    
    \node[block, fit=(advance_loss_function)(loss_function)] (loss) {};

    
    \node [block,  right=of loss] (activations) {Activations};

    \node [block,  right=of activations] (weights) {Weights};

    % \node [block, below right=of enclayer1, xshift=-1.8cm] (enclayer2) {Enconding2};
    % \node [block, below right=of enclayer2, xshift=-1.6cm] (enclayer3) {Enconding3};
    % % (None, 168, 18) 
    
    % \node [block, below right=of enclayer3] (up1) {Enconding4};
    
    % % Decoder (Expanding Path)
    % \node [block, above right=of up1] (declayer1) {Decoding1};
    % \node [block, above right=of declayer1, xshift=-1.6cm] (declayer2) {Decoding2};
    % \node [block, above right=of declayer2, xshift=-1.8cm] (declayer3) {Decoding3};
    % \node [block, above right=of declayer3, xshift=-2cm] (output) {Output};
    
    % % Skip Connection
    % % \draw [line] (pool1) -- ++(0,-1) -| (up1);
    
    % Connections
    \draw [line] (archs) -- (advance_loss_function);
    \draw [line] (advance_loss_function) -- (loss_function);
    \draw [line] (loss) -- (activations);
    \draw [line] (activations) -- (weights);

    \draw [line, bend right=30] (advance_loss_function) to (archs);



    \draw [line, bend right=30] (weights) to (archs);

    % \draw [line, bend left=30] (loss_function) to (archs);
    % \draw [line] (up1) -- (declayer1);
    
    
    % \draw [line] (declayer1) -- (declayer2);
    % \draw [line] (declayer2) -- (declayer3);
    % \draw [line] (declayer3) -- (output);
    
    
    % \draw [line] (input) -- (output);
    % \draw [line] (enclayer1) -- (declayer3);
    % \draw [line] (enclayer2) -- (declayer2);
    % \draw [line] (enclayer3) -- (declayer1);
    
    
    \end{tikzpicture}}
	\caption{Model choice method scheme.}
	\label{fig:method_training}
\end{figure}

For the porpuse of controling and processing this experiment three python packages were created.

\begin{itemize}
    \item \textbf{\href{https://github.com/alquimodelia/alquimodelia}{Alquimodelia}}: A keras based model builder package, to create the necessary models with each different arch and variable.
    \item \textbf{\href{https://github.com/alquimodelia/alquitable}{Alquitable}}: A keras based workshop package, to create custom callbacks, loss functions, data generators.
    \item \textbf{\href{https://github.com/alquimodelia/MuadDib}{MuadDib}}: A machine learning framework that uses alquimodelia to test and choose best models on given conditions automatically.
\end{itemize}

The experiments were done using \href{https://keras.io/}{keras}>=3 with a torch backend on a CPU laptop.

% \subsubsection{Advance Loss Function}


% Para escolher a melhor maneira de distribuir pesos foi criada uma função de perda com diferentes regras, que distribuem o peso da amostra:
% \href{https://github.com/alquimodelia/alquitable/blob/main/alquitable/advanced_losses.py#L33}{Mirror Weights (Pesos Espelhados)},
% que vai distribuir os pesos da amostra consoante um rácio predefinido e o próprio erro da amostra.\par
% Os pesos nas amostras vão ser divididos entre os erros negativos (alocação em demasia) e os positivos (alocação em falta). Consoante uma variável lógica,  uns terão peso 1 e os outros serão o próprio erro em absoluto. Dando assim um peso equivalente ao erro, quanto maior o erro maior o peso da amostra na função de perda, do lado da amostra escolhido (em demasia ou em falta).\par
% O rácio pode ser multiplicado um rácio tanto a um dos pesos como a outro, sendo estes rácios que irão equilibrar as diferenças entre a alocação em falta e a em demasia. Refira-se que o sinal do rácio influencia qual o lado a ser multiplicado.\par
% Este pesos são passados directamente à função de perda em uso.\par


% \begin{figure}[H]
%     \centering
%     \includegraphics[width=\textwidth]{plots/ratio_mw.png}
%     \caption{Resultados de alocações totais em diferentes rácios}
%     \label{fig:resexpratiomw}
%   \end{figure}

% Estas variações no rácio produzem diferentes dimensões nas alocações, modificando assim a sua posição em relação ao \textit{benchmark}. Aqui para cada arquitetura o rácio ideal para o melhor GPD Positivo diferencia ligeiramente, tendo sido procurado com tentativa/erro baseado em assunções perante a aparente distribuição rácio/alocações.\par


\subsection{Metrics}

With distinct valorizations, the metrics, are used to choose best model on each iteration, and they can be divided into two groups:
Model metric, where we just use the usual regression metrics adding a metric for how much did the model missed in alocating for the validation periodo. And the comparative metrics, where we assert percentual gains over the current allocation method. \par 
\begin{alignat*}{3} 
& t : \text{Real value.} &\qquad& p : \text{Forecast} &\qquad& n : \text{number of samples} \\
\end{alignat*}


\subsubsection{Model Metrics}
\begin{linenomath}
    \begin{equation}\label{eq:rmse}
        RMSE = \sqrt{\frac{1}{n} \sum_{i=1}^{n}(t_i - p_i)^2}
    \end{equation}
    \end{linenomath}

\begin{linenomath}
    \begin{equation}\label{eq:SAE}
        SAE = \sum_{i=1}^{n}\left|t_i - p_i \right|
    \end{equation}
    \end{linenomath}
SAE can be divide into the following metrics, where we obtain the error, within the time period, of allocated energy not enough for the needs, and too much energy allocated, separatly.\\
AllocM - Missing Allocation: \\
\begin{linenomath}
    \begin{equation}\label{eq:AllocM}
        AllocM = 
        \begin{cases} 
            0 & , \text{if } p \geq t \\
            t - p  & , \text{if } p < t \\
        \end{cases} 
        \end{equation}
    \end{linenomath}
AllocS - Surplus Allocation \\
\begin{linenomath}
    \begin{equation}\label{eq:AllocS}
        AllocS = 
        \begin{cases} 
            0 & , \text{if } p \leq t \\
            p - t  & , \text{if } p > t \\
        \end{cases} 
            \end{equation}
    \end{linenomath}
We need these metrics because we are not just looking to get a better error than the benchmark, we want to have both instances of less wasted allocated energy (AllocS), and less occurences of missing allocation (AllocS).\par


\subsubsection{Model/benchmark comparative metrics}

PPG - Performance Percentual Gain
\begin{linenomath}
    \begin{equation}\label{eq:PPG}
        PPG = \frac{SAE_{benchmark} - SAE_{modelo}}{SAE_{benchmark}} \times 100
    \end{equation}
    \end{linenomath}
PPG is the percentage of how much better is the model over the benchmark. The following metrics are the same but for only missing allocation and surplus allocation.\\
PPGM - Performance Percentual Gain Missing (allocation)\\
\begin{linenomath}
    \begin{equation}\label{eq:PPGM}
        PPGM = \frac{AllocM_{benchmark} - AllocM_{modelo}}{AllocM_{benchmark}} \times 100
    \end{equation}
    \end{linenomath}
PPGS - Performance Percentual Gain Surplus (allocation)\\
\begin{linenomath}
    \begin{equation}\label{eq:PPGS}
        PPGS = \frac{AllocS_{benchmark} - AllocS_{modelo}}{AllocS_{benchmark}} \times 100
    \end{equation}
    \end{linenomath}

The next metric is showing how much better is the model over the benchmark, but only if both condition are met. PPGM and PPGS positive.
PPG Positive  - Performance Percentual Gain Positive
\begin{linenomath}
    \begin{equation}\label{eq:PPGPositive}
        PPG Positive = 
        \begin{cases} 
            PPG & , \text{if } PPGM \text{ }\&\text{ } PPGS \geq 0 \\
            0 & , \text{if } PPGM \text{ }\|\text{ } PPGS < 0 \\
        \end{cases} 
        \end{equation}
    \end{linenomath}



% To address the challenges introduced by \gls{vRES}, dynamic reserve procurement methods have been proposed. Unlike static methods, dynamic approaches consider real-time or near real-time forecasts of energy demand and renewable generation, allowing \gls{TSO}s to adjust reserve allocations accordingly. This adaptability reduces over-procurement and minimizes costs, improving the efficiency of reserve markets.

% The adoption of advanced forecasting tools, particularly machine learning techniques, is central to enabling dynamic reserve procurement. By leveraging historical and operational data, machine learning models can predict reserve needs with greater accuracy, addressing the uncertainties associated with \gls{vRES} generation. Studies have shown that these models outperform traditional statistical methods, offering significant improvements in reserve management and cost reduction.

% In conclusion, the evolving electricity markets and ancillary services frameworks must adapt to the challenges posed by high \gls{vRES} penetration. Dynamic reserve procurement, supported by advanced forecasting techniques and market design improvements, offers a path toward more efficient and reliable power systems.


% The dynamic procurement of secondary reserves represents a significant step forward in addressing the inefficiencies inherent in traditional static allocation methods. Unlike static reserve procurement, which relies on fixed ratios or historical averages, dynamic approaches incorporate real-time forecasts and system conditions to adjust reserve requirements. This adaptability is particularly critical for modern electricity systems with high penetration of \gls{vRES}, where forecasting uncertainty and rapid changes in generation output challenge grid stability.

% Dynamic reserve procurement involves estimating upward and downward reserve needs based on the expected deviations between day-ahead scheduled generation and real-time demand. By leveraging advanced forecasting tools, such as machine learning models, it becomes possible to predict these deviations with greater accuracy, optimizing the allocation of secondary reserves. Historical data on vRES production, system demand, and grid imbalances serve as inputs to these models, allowing the identification of patterns and trends that inform reserve procurement decisions.

% Machine learning techniques, including \gls{LSTM} networks and other time-series forecasting models, have demonstrated significant potential for improving reserve predictions \cite{Costa2022}\cite{Benti2023}. These models can capture the nonlinear and temporal dependencies present in renewable energy data, outperforming traditional statistical methods such as ARIMA. By incorporating real-time weather forecasts, generation data, and demand profiles, dynamic approaches ensure that reserve procurement aligns more closely with actual system needs, reducing both over-procurement and under-procurement of reserves.

% The dynamic approach also allows for asymmetrical procurement of upward and downward reserves, which is particularly relevant in systems with variable renewable generation. For instance, during periods of high solar generation, upward reserves may be less critical, whereas downward reserves become essential to accommodate excess production. Conversely, during low renewable output, upward reserves are prioritized to address potential generation shortfalls.

% In summary, dynamic procurement of secondary reserves offers a more efficient and adaptive solution to balancing challenges in modern electricity systems. By leveraging machine learning techniques and real-time forecasts, this approach enhances reserve allocation, reduces operational costs, improving penetration of \gls{vRES}.

% 
\subsection{Previsão de Necessidades de Reservas \label{se:pred_impact_vres}}

A previsão das necessidades de reservas de frequência é uma componente essencial na gestão eficiente dos sistemas eléctricos, especialmente num cenário de crescente penetração das \gls{vRES}.\par
O uso de técnicas de \textit{machine learning} tem sido explorado como uma solução promissora para melhorar essas previsões. Estes modelos podem analisar grandes volumes de dados, identificar padrões complexos e ajustar previsões em tempo real, considerando factores como mudanças nas condições meteorológicas e padrões de consumo de energia. Ao incorporar a variabilidade das \gls{vRES} nos modelos de previsão, é possível reduzir a incerteza e melhorar a alocação das reservas de frequência, resultando numa operação mais eficiente do sistema eléctrico.\par
Outro factor crítico na previsão das necessidades de reservas de frequência é a coordenação entre diferentes mercados e operadores de sistemas. A harmonização dos mercados europeus de balanço, incluindo a padronização das regras de oferta, leilão e remuneração, pode facilitar a integração das \gls{vRES} e melhorar a eficiência geral do sistema. Com regras claras e uniformes, os produtores de energia renovável têm maior incentivo para participar activamente dos mercados de reservas, fornecendo capacidade adicional para apoiar a estabilidade da rede. Esta questão é particularmente relevante em mercados onde as \gls{vRES} ainda enfrentam barreiras significativas para a participação, como regras complexas de licitação ou altos requisitos de capacidade mínima para participação.\par
Apesar dos avanços na previsão de necessidades de reservas, ainda existem desafios consideráveis. A precisão das previsões pode ser limitada pela qualidade dos dados disponíveis, bem como pela capacidade dos modelos de capturar todas as variáveis relevantes que afetam a operação da rede. Além disso, a crescente interconexão dos sistemas eléctricos e o aumento da troca de energia entre países exigem uma abordagem coordenada e colaborativa para a previsão de reservas, considerando tanto as condições locais como as condições regionais.\par
O desenvolvimento contínuo de técnicas avançadas de previsão e a integração de soluções baseadas em dados serão fundamentais para enfrentar esses desafios. À medida que mais dados históricos se tornam disponíveis e os modelos de previsão evoluem, espera-se que a gestão das reservas de frequência se torne cada vez mais eficiente, contribuindo para um sistema eléctrico mais resiliente e capaz de integrar altos níveis de Tal desenvolvimento, não apenas reduzirá os custos operacionais, mas também contribuirá para a segurança energética e para a transição para um sistema energético mais sustentável.\par

\subsubsection{Previsão de Banda Secundária no Mercado Ibérico de Electricidade \label{se:pred_mibel}}
A nível Europeu a \gls{ENTSO-E} providencia várias metodologias para o dimensionamento das reservas de controlo descritas em \cite{handbook2009policy}. A quantidade mínima recomendada de alocação necessária para a reserva de controlo secundária pode ser descrita da seguinte forma:\par

\begin{equation} \label{eq:BRENTSOE} 
    BR = \sqrt{a \times  L_{max} + b^{2}} - b 
\end{equation}
onde:
\begin{itemize}
  \item $BR$: Banda de Reserva de regulação secundária mínima necessária (MW).
  \item $a$ e $b$: Coeficientes empiricos, $a$=10MW e $b$=150MW .
  \item $L_{max}$: Consumo máximo antecipado (MW).
\end{itemize}


\paragraph{Portugal \label{se:prev_portugal}}
\text{ }  \par

No mercado português para dimensionar a \gls{aFRR} a \gls{REN} utiliza por base a equação \ref{eq:BRENTSOE} multiplicando um parâmetro horário, $\rho$:

\begin{equation} \label{eq:BRREN} 
    BR = \rho \times \sqrt{a \times  L_{max} + b^{2}} - b 
\end{equation}
onde:
\begin{itemize}
  \item $\rho$: Paramêtro horário.
\end{itemize}


Na equação \ref{eq:BRREN} BR equivale à banda a subir, sendo a banda a descer metade da banda a subir. De notar que em \cite{Carneiro2016} BR é a banda de reserva, que equivale à soma da banda a subir e banda a descer, onde aí é sempre considerado que banda a subir são $\frac{2}{3}$ da Banda de Reserva total e a banda a descer é o restante $\frac{1}{3}$.\par
Este método de cálculo permite manter as reservas a corresponder às necessidades do sistema, mas têm uma uma alocação  em excesso. Podemos verificar que no período 2013 a 2023, inclusive, as médias por hora têm cerca de 437\% de alocação em excesso, o que corresponde, em média, a cerca de 221 MWh desperdiçados a cada hora. \par

\begin{table}[H]
	\centering
    \caption{Média das Bandas Alocada e Usada (REN)}    
    \resizebox{!}{!}{\begin{tabular}{rrrr}
\toprule
Banda de Reserva Alocada & Banda Reserva Activada & erro & erro \% \\
\midrule
271.57 & 50.53 & 221.04 & 437.43 \\
\bottomrule
\end{tabular}
}
    \label{tab:media_bandas_pt}
    \end{table}

Estando actualmente o \gls{TSO} português a utilizar esta fórmula, e a obter estes resultados, este é um bom caso de estudo de optimização dos parâmetros da fórmula. Sendo que \textit{a} e \textit{b} são dados pela entidade europeia, propõe-se o estudo do parâmetro horário de modo a corresponder a banda de reserva calculada ao consumo real.\par

\paragraph{Espanha \label{se:prev_espanha}}
\text{ }  \par

No mercado espanhol não encontramos directivas de uso de uma fórmula como no caso português. Nem encontramos uma simetria directa entre as bandas a subir e a descer. Contudo, podemos verificar que a média horária dentro do mesmo período apresenta disparidades ainda maiores em quantidade média de energia alocada desperdiçada.\par

\begin{table}[H]
	\centering
    \caption{Média das Bandas Alocada e Usada (REE)}    
    \resizebox{!}{!}{\begin{tabular}{lrrrr}
\toprule
 & Banda de Reserva Alocada & Banda Reserva Activada & erro & erro \% \\
\midrule
Banda a Subir & 662.94 & 158.10 & 504.84 & 319.32 \\
Banda a Descer & 549.27 & 168.20 & 381.07 & 226.55 \\
\bottomrule
\end{tabular}
}
    \label{tab:media_bandas_es}
    \end{table}


Como temos uma boa quantidade de dados históricos e uma falta de definição e formulação exacta da necessidade, o caso espanhol é um bom caso de estudo para previsões usando \textit{machine learning}.\par



% \input{chapters/4.1.1.model_archs.tex}
