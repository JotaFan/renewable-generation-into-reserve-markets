% \section{Métodos}

Iniciamos a metodologia com a definição de dois problemas de previsão. Um de optimização de uma variável numa fórmula, com o caso do mercado português; e o uso de \textit{machine learning} para fazer a previsão apenas usando os dados disponíveis.\par

\section{Optimização}

Queremos optimizar o parâmetro $\rho$ presente na equação \ref{eq:BRREN}. Parâmetro esse com os seguintes valores para cada hora apresentados em \cite{Carneiro2016}, que usaremos como medida de comparação:\par
\begin{table}[H] \centering \caption{Valores de $\rho$ apresentado em \cite{Carneiro2016}} \begin{tabular}{ll}
\toprule
Hora & $\rho$ \\
\midrule
1/2/8/9/24 & 1,6 \\
3/7/10/11/19/20 & 1,4 \\
4 & 1,3 \\
5/6/12/13/14/15/16/17/18/21/22/23 & 1,2 \\
\bottomrule
\end{tabular}
 \end{table}

Vamos extrapolar o $\rho$ através dos valores históricos de consumo real, ao invés do consumo previsto, e do valor de banda calculada, onde para cada entrada horária aplicamos a seguinte fórmula:\par
\begin{equation} \label{eq:rhoproposed} 
    \rho  = \frac{(BR + b)}{\sqrt{a \times Consumo + b^{2}}}
\end{equation}

Para aproximar os $\rho$s em cada um das horas, com o menor erro possível, vamos testar com aplicação de várias aproximações: média, mediana, média ponderada à banda, ou ao consumo.\par
O erro será diferença entre a \textit{BR} calculada a partir do $\rho_{proposto}$ e do consumo real, e a \text{BR} calculada pela \gls{REN}. A aproximação que apresentar um menor erro é escolhida.\par


\section{\textit{Machine Learning}}

Na tentativa de solucionar este problema propomos-nos a realizar várias experiências criando modelos com as várias arquiteturas apresentadas, e experimentado diversas parametrizações das mesmas.\par
Além dos métodos de redes neuronais vamos testar métodos de previsão estatísticos, de modo a termos também um ponto de comparação usando métodos interpretáveis e transparentes.\par
O objectivo é conseguir um modelo que dentro do período de validação, 2019 a 2022 inclusive, consiga prever no mínimo a alocação necessária, mas tendo um erro inferior ao da alocação feita pelo \gls{TSO} espanhol.\par


% \thispagestyle{plain}
% \subsubsection{Estatisticos}

Em estatística conseguimos encontrar vários métodos de estudo de séries temporais. Estes métodos são normalmente usados como primeira abordagem para fazer previsões.\par
Estes modelos podem ser \gls{AR}, que fazem previsões baseados num número \textit{(p)} de dados anteriores. Estes modelos são construídos com a noção de que um valor é linearmente dependente de \textit{p} valores anteriores numa série temporal.\par

\begin{alignat*}{2} 
    & X_{t} : \text{Valor no } t \text{ a prever.} &\quad& p : \text{O número observaçoes anteriores.} \\
    & \varphi_{i} : \text{Coeficiente na observação } i. &\quad& q : \text{O número observaçoes anteriores.} \\
    & \epsilon_{i} : \text{Erro na observação } i. &\quad& \theta_{i} : \text{Coeficiente na observação } i \text{.} \\ 
    & \mu : \text{Média dos valores } X \text{.} 
\end{alignat*}

\bigskip
\gls{AR} \\

\begin{equation} \label{eq:ar} 
    X_{t} = \sum_{i=1}^{p}\varphi_{i} X_{t-i} 
\end{equation}
\smallskip

Outra família destes modelos são os de \gls{MA}, onde a média de um número de observações \textit{(q)} em conjunto com os erros \textit{($\epsilon$)} e os coeficientes \textit{($\theta$)} é usada para prever os valores seguintes.\par
\bigskip
\gls{MA} \\

\begin{equation} \label{eq:ma} 
    X_{t} = \mu + \sum_{i=1}^{q}(\theta_{i} \epsilon_{t-i}) + \epsilon_{t}
\end{equation}
\smallskip
Estes dois tipos de modelos podem ser utilizados em conjunto, criando os modelos \gls{ARMA}, que incorpora as capacidades de ambos os modelos.\par

\bigskip
\gls{ARMA} \\

\begin{equation} \label{eq:arma} 
    X_{t} = \sum_{i=1}^{p}\varphi_{i} X_{t-i}  + \mu + \sum_{i=1}^{q}(\theta_{i} \epsilon_{t-i}) + \epsilon_{t}
\end{equation}
\smallskip

Existem mais modelos de previsão estatística baseados nestes com algumas variações, mas para este trabalho, e apenas como ponto de comparação às redes neuronais, ficamos apenas por estes.\par
As variáveis em estudo por tipo de modelo foram retiradas das autocorrelações temporais usando os métodos de sugestão da ferramenta \hyperref[se:muaddib]{MuadDib}:\\


\begin{table}[h] \centering
\begin{tabular}{lrrrr}
    \toprule
     & p & q \\
    \midrule
    \gls{AR} & 1/ 2 / 23 / 24 / 25 / 48 / 144 / 168 / 192 / 336 & NA \\
    \gls{MA} & NA & 1 / 24 \\
    \gls{ARMA} & 1 & 1 \\
    \bottomrule
    \end{tabular}
    \label{tab:varsstats} 
    \caption{Variáveis de estudo dos modelos AR/MA}
\end{table}


Todos estes modelos foram testados usando o software disponível na package de \textit{python} \href{https://www.statsmodels.org/stable/index.html}{\textit{statsmodel}}, com a classe \href{https://www.statsmodels.org/stable/generated/statsmodels.tsa.arima.model.ARIMA.html}{ARIMA}.
 \label{se:metstats}


% \newpage
\thispagestyle{plain}
\subsubsection{Redes Neuronais}

As redes neuronais podem ser descritas como uma função desconhecida \textit{f(x)=y} onde durante o treino a função \textit{f} é criada através da manipulação dos pesos da sua arquitetura usando os dados de treino, x, de forma a diminuir ao máximo uma função de perda . Sendo \textit{f'(x)=y'} um modelo já treinado onde \textit{y'} é a previsão, a função de perda \textit{fp(y, y')} idealmente igual a 0, com \textit{y'=y.}.\par
Neste trabalho o \textit{x} são todos os dados apresentados no capitulo \hyperref[ch:estudo_2]{Estudo 2}, em grupos de 128 (horas), e o \textit{y} é a energia usada, "UpwardUsedSecondaryReserveEnergy" no modelo de previsão de energia a subir e "DownwardUsedSecondaryReserveEnergy" no modelo de previsão de energia a descer, nas 24 horas subsequentes. A \textit{fp} é um dos factores de estudo, assim como outros parâmetros dentro das arquiteturas de modelos, \textit{f}.\par
Assim utilizamos os 168 horas (1 semana) para prever as 24 horas seguintes. As 24 horas seguintes são o objectivo do estudo, energia a alocar no dia seguinte. As 168 horas são escolhidas graças às \hyperref[tab:tempcorr]{maiores autocorrelações temporais}, de onde as maiores fora das primeiras 48 horas são 144, 168, 192 horas ou seja, 6, 7 e 8 dias respectivamente, onde em ambos os casos 7 dias era o valor com maior correlação.\par
As condições em estudo são feitas através da ferramenta \hyperref[se:muaddib]{MuadDib}, seguindo vários percursos entre as combinações possíveis, de modo a conseguir a combinação óptima.\par

%TODO: check google doc
\paragraph{Arquitecturas}
\text{ }  \par

\gls{FCNN}, \gls{CNN}, RNN são as arquitecturas mais simples que vamos estudar. Estas vão apenas pegar nos blocos e vamos criar as mesmas "Vanila" e "Stacked" com 2 blocos (ex: StackedCNN) ou 6 blocos (ex: Stacked6CNN).\par
UNET, \gls{LSTM} são arquiteturas mais complexas e pesadas. Como descrito anteriormente uma mais utilizada em análise de imagens, e outra em análise de texto respectivamente.\par
Transformers são as mais pesadas qualidade comum da família de "generative AI".

\paragraph{Função de Perda}
\text{ }  \par

Nos primeiros testes mais simples foi imediato a discrepância entre os erros da energia alocada em demasia e em falta. Sendo que estes erros estão em dimensões completamente diferentes.
\begin{figure}[H]
    \centering
    \includegraphics[width=\textwidth]{plots/allocs_results_shadow.png}
    \caption{Resultados de alocações totais em diferentes arquiteturas}
    \label{fig:resexparchs}
  \end{figure}

Na energia em falta, estamos a lidar com valores na dimensão de $10^{6}$ nos resultados, sendo que o benchmark está nos $10^{5}$. Logo estão bastante acima do que queremos. Por outro lado na Energia em Demasia temos resultados na ordem dos $10^{6}$ e o benchmark está na ordem dos $10^{7}$. Isto dá-nos espaço para aumentar os resultados da Energia em Demasia mantendo-os ainda abaixo do benchmark para diminuir os resultados da Energia em Falta com objectivo de a ter também abaixo do benchmark.\par
Para combater esta desigualdade foram criadas várias funções de perda para atribuir melhor peso a ambas de modo a atingir melhor o objectivo geral.\par
De maneira que partimos esta experiência em duas partes. A primeira parte, Função de Perda Avançada, vai estudar diferentes maneiras de distribuir pesos entre a energia alocada em demasia e a em falta. A segunda vai escolher qual a melhor função de perda a aplicar nessa distribuição de pesos, ou vice-versa.\par


\subparagraph{Funções de Perda}
\text{ }  \par
Depois de escolhidos os pesos nos diferentes grupos são testadas as funções a aplicar. Aqui vamos manter simples e testar apenas as mais comuns em problemas de regressão linear: \gls{MAE}, \gls{MSE}, \gls{MSLE}.\par
\gls{MAE} é usada no geral em problemas em que os dados têm um histograma linear, e um erro normalmente distribuído.\par
\gls{MSE} é usado para atribuir maior peso aos erros maiores, do que no \gls{MAE}. Fazendo com que o modelo se concentre mais em aprender a diminuir erros maiores.\par
\gls{MSLE} é sugerido em dados que têm uma histograma exponencial.\par

% TODO: meter formulas? depende do espaço


\subparagraph{Função de Perda Avançada}\label{se:advancedloss}
\text{ }  \par
Para escolher a melhor maneira de distribuir pesos foi criada uma função de perda com diferentes regras, que distribuem o peso da amostra.\par
\href{https://github.com/alquimodelia/alquitable/blob/main/alquitable/advanced_losses.py#L33}{Mirror Weights (Pesos Espelhados)}.\par
Que vai distribuir os pesos da amostra consoante um rácio predefinido e o próprio erro da amostra.\par
Os pesos nas amostras vão ser divididos entre os erros negativos (alocação em demasia) e os positivos (alocação em falta). Consoante uma variável lógica,  uns terão peso 1 e os outros serão o próprio erro em absoluto. Dando assim um peso equivalente ao erro, quanto maior o erro maior o peso da amostra na função de perda, do lado da amostra escolhido (em demasia ou em falta).\par
Pode ser multiplicado um rácio tanto a um dos pesos como a outro, sendo estes rácios que irão equilibrar as diferenças entre a alocação em falta e a em demasia. E o sinal do rácio influencia qual o lado a ser multiplicado.\par
Este pesos são passados directamente à função de perda em uso.\par

% TODO: meter formulas? depende do espaço

\begin{figure}[H]
    \centering
    \includegraphics[width=\textwidth]{plots/ratio_mw.png}
    \caption{Resultados de alocações totais em diferentes rácios}
    \label{fig:resexpratiomw}
  \end{figure}

Estas variações no rácio produzem diferentes dimensões nas alocações, modificando assim a sua posição em relação ao benchmark. Aqui para cada arquitetura o rácio ideal para o melhor GPD Positivo diferencia ligeiramente, tendo sido procurado com tentativa/erro baseado em assunções perante a aparente distribuição rácio/alocações.\par


\paragraph{Função de Activação}
\text{ }  \par

Como mostrado em \cite{Vaswani2017}, e \cite{Liu2022}, o uso de uma activação mais apropriada aos dados pode ser crucial para um salto na qualidade do modelo.\par
Vamos dividir as função de activação usadas nas camadas intermédias e a usada na camada final. Isto porque as camadas intermédias tendem a funcionar melhor com a mesma activação e a final é que mais define o valor que sai do modelo.\par
Esta experiência vai testar a combinações das seguintes activações nas duas variáveis descritas anteriormente: linear, relu, gelu.\par


\paragraph{Pesos}
\text{ }  \par

Esta experiência serve para testar diferentes pesos por amostra, não por grupo como na experiência anterior. Aqui os pesos são aplicados no momento da função de perda final.\par
Normalmente é usado para dar mais pesos a amostras com menor amostragem. Mais facilmente aplicável em modelos de classificação. Com este é um problema de regressão linear com séries temporais vamos testar aplicar os seguintes pesos, ou nenhum peso.\par
Este peso é multiplicado pelo peso em \hyperref[se:advancedloss]{peso espelhados}.


\subparagraph{Temporais}
\text{ }  \par
Aqui a primeira amostra tem o menor valor de peso (1) e todas as amostras seguintes incrementam 1. Dando mais peso consecutivamente a amostras mais recentes. É testado em vários casos de séries temporais onde o objectivo é prever o futuro. Podendo assim dar mais peso a tendências e valores mais recentes.\par

\subparagraph{Distância à média}
\text{ }  \par
Neste peso cada amostra tem como valor a sua distância à média total dos dados. Vai servir para o modelo conseguir criar pesos relevantes a valores mais distantes à média.\par
Logo as amostras que tenham picos de valores têm um peso maior, forçando o modelo a aprender melhor estas ocasiões.

 \label{se:metneuralnet}

% \thispagestyle{plain}
% \input{Capitulos/Tese-5.2.3-Metodos-validacao_benchmark}

