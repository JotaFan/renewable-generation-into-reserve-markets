\section{Estimativa do parâmetro $\rho$ da fórmula da REN}


Com o $\rho_{proposto}$, calculado através da fórmula \ref{eq:rhoproposed}, arredondado a uma casa decimal, podemos verificar no histograma, apresentado na figura \ref{fig:histograma_parametro_p}, uma diferença considerável entre as contagens de ambos os conjuntos de valores $\rho$ apresentados.\par


% \begin{figure}[H]
%     \centering
%     \includegraphics[width=0.7\textwidth]{plots/histograma_parâmetro_ρ.png}
%     \caption{Histograma $\rho$}
%     \label{fig:histograma_parametro_p}
%   \end{figure}


% \begin{figure}[H]
%     \centering
%     \includegraphics[width=0.65\textwidth]{plots/valor_do_parametro_ρ_(hora).png}
%     \caption{Valor do paramêtro $\rho$ (hora)}
%     \label{fig:valor_do_parametro_ρ_hora}
%   \end{figure}

\begin{figure}[H]
    \centering
    % First figure in a minipage
    \begin{minipage}[b]{0.49\textwidth}
        \centering
        \includegraphics[width=\textwidth]{plots/histograma_parâmetro_ρ.png}
        \caption{Histograma $\rho$}
        \label{fig:histograma_parametro_p}
    \end{minipage}
    % Second figure in a minipage
    \begin{minipage}[b]{0.49\textwidth}
        \centering
        \includegraphics[width=\textwidth]{plots/valor_do_parâmetro_ρ_(hora).png}
        \caption{Valor do parâmetro $\rho$ (hora)}
        \label{fig:valor_do_parametro_ρ_hora}
    \end{minipage}
\end{figure}

Como podemos ver na figura \ref{fig:valor_do_parametro_ρ_hora}, acima apresentada, o $\rho_{proposto}$ apresenta um grande variabilidade em todas as horas, embora de notar que em todas tem um maior peso perto da mediana. O $\rho$ de comparação embora sempre dentro da distribuição note-se que cai quase sempre em zonas com pouco peso nestes dados históricos.\par
Calculamos $\rho$ possíveis para proposta final usando as seguintes aproximações: média, mediana, \gls{mp-consumo} e \gls{mp-BS}.\par

As distribuições por hora são as apresentadas na seguinte figura:

\begin{figure}[H]
    \centering
    \includegraphics[height=0.42\textwidth]{plots/comparação_de_ρ_propostos_por_hora.png}
    \caption{Comparação $\rho$ por hora}
    \label{fig:comparação_de_ρ_propostos_por_hora}
\end{figure}

Todas seguem um percurso semelhante ao longo do dia, o qual também pode ser extrapolado para Carneiro2016. A média e mediana destacam-se seguindo muito parecidas, enquanto que as ponderadas também parecidas entre elas são bastante mais discretas.\par
Para a escolha da aproximação deste parâmetro à Hora, estudou-se o erro entre a Banda Reserva calculada através das aproximações feitas e a Banda Reserva disponível nos dados.\par


\begin{figure}[H]
    \centering
    \includegraphics[width=0.7\textwidth]{plots/comparação_das_métricas_de_erro.png}
    \caption{Comparaçao dos erros por $\rho$}
    \label{fig:comparação_das_métricas_de_erro}
\end{figure}


\begin{table}[H]
    \centering
    \caption{Erros de Banda de Reserva por método de normalização $\rho$}    
    \resizebox{0.65\linewidth}{!}{\begin{tabular}{lrrrr}
\toprule
 & MAE (MW) & RMSE (MW) & MedianAE (MW) & MAPE (\%)\\
Normalização &  &  &  &  \\
\midrule
Carneiro2016 & 53.07 & 66.54 & 44.53 & 18.70 \\
média & 30.94 & 39.19 & 25.38 & 11.58 \\
mediana & 30.85 & 39.20 & 25.17 & 11.51 \\
média ponderada banda & 32.15 & 40.61 & 26.45 & 12.19 \\
média ponderada consumo & 31.54 & 39.91 & 26.20 & 11.73 \\
\bottomrule
\end{tabular}
}
    \label{fig:tabela_estudo_1_medias}
    \end{table}


A normalização com erros mais baixos é a média. Com um erro médio (de todo o histórico) para o consumo real de 8.8\% o que comparando com o \textit{benchmark} de 29.36\% é uma melhoria  bastante considerável. Comparando as bandas calculadas a uma média em cada hora:\par

\begin{figure}[H]
    \centering
    \includegraphics[width=0.75\textwidth]{plots/média_historica_de_banda_de_reserva.png}
    \caption{Média historica de Banda de Reserva}
\end{figure}

Podemos ver que em termos de média horária, a Banda de Reserva calculada através do $\rho_{proposto}$ apresenta quase uma sobreposição por inteiro ao valor médio real.\par

Retiramos as médias dos erros percentuais e podemos observar: \\

\begin{figure}[H]
    \centering
    \includegraphics[width=0.75\textwidth]{plots/erro_médio_por_hora_banda_de_reserva.png}
    \caption{Erro médio por hora Banda de Reserva}
\end{figure}

Em termos de média diária o erro pelo método proposto está bem abaixo da margem de erro do 5\% na banda, em todas as horas. E na outra tese apenas 10\% cai dentro dessa margem de erro.\par

Como tal o $\rho_{proposto}$ a partir do estudo dos dados  históricos é: \

\begin{table}[H] \centering \caption{Valores de $\rho$ propostos} \begin{tabular}{rr}
\toprule
Hora & $\rho$ \\
\midrule
1 & 1.621694 \\
2 & 1.576623 \\
3 & 1.486929 \\
4 & 1.364176 \\
5 & 1.313958 \\
6 & 1.318832 \\
7 & 1.504499 \\
8 & 1.612361 \\
9 & 1.638188 \\
10 & 1.613728 \\
11 & 1.601277 \\
12 & 1.485861 \\
13 & 1.451995 \\
14 & 1.457233 \\
15 & 1.440454 \\
16 & 1.421988 \\
17 & 1.424636 \\
18 & 1.420682 \\
19 & 1.553086 \\
20 & 1.588201 \\
21 & 1.480219 \\
22 & 1.478815 \\
23 & 1.474412 \\
24 & 1.635658 \\
\bottomrule
\end{tabular}
 \end{table}


% \begin{table}[H]
%     \caption{Valores de $\rho$ propostos}    
%     \resizebox{\linewidth}{!}{\begin{tabular}{rr}
\toprule
Hora & $\rho$ \\
\midrule
1 & 1.621694 \\
2 & 1.576623 \\
3 & 1.486929 \\
4 & 1.364176 \\
5 & 1.313958 \\
6 & 1.318832 \\
7 & 1.504499 \\
8 & 1.612361 \\
9 & 1.638188 \\
10 & 1.613728 \\
11 & 1.601277 \\
12 & 1.485861 \\
13 & 1.451995 \\
14 & 1.457233 \\
15 & 1.440454 \\
16 & 1.421988 \\
17 & 1.424636 \\
18 & 1.420682 \\
19 & 1.553086 \\
20 & 1.588201 \\
21 & 1.480219 \\
22 & 1.478815 \\
23 & 1.474412 \\
24 & 1.635658 \\
\bottomrule
\end{tabular}
}
%     \end{table}

% Em relação a perdas por arredondamento, apresento o resultado dos erros por arrendamento em cada um da casas possíveis, concluindo que até à primeira casa decimal, pode ser feito arredondamento do parâmetro $\rho$, sem influenciar muito o erro: \\


% \begin{figure}[H]
%     \centering
%     \includegraphics[width=\textwidth]{plots/Erro_médio_por_hora_Banda_de_Reserva_Arredondamento.png}
%     \caption{Erro médio por hora Banda de Reserva (Arredondamento)}
% \end{figure}



Neste estudo podemos comprovar que usando um $\rho$ extrapolado dos dados históricos, e um $L_{max}$ sendo o consumo real e não o consumo máximo calculado, os erros médios por hora ficam abaixo dos 5\%.\par