\chapter{Resumo}
\justifying

A crescente penetração de fontes de energia renováveis variáveis no tempo, vRES, no sistema eléctrico, como a solar ou eólica, está a transformar significativamente os mercados de eletricidade, devido à sua natureza intermitente e imprevisível. Isso torna as previsões de produção e consumo de energia mais desafiantes, especialmente porque os mercados fecham entre 1 e 37 horas antes da entrega real da energia, podendo originar discrepâncias entre as energias contratadas e necessárias. Manter o equilíbrio entre a oferta e a procura em tempo real é vital para a segurança e estabilidade da rede, função que recai principalmente sobre os operadores de redes de transporte (TSO).\par
Os TSO utilizam mercados de reserva de energia, onde adquirem de forma simétrica potência secundária ascendente e descendente, com base em previsões de procura para as horas subsequentes. No entanto, essa abordagem é ineficaz face às flutuações das renováveis, levando à necessidade de ajustes mais dinâmicos e precisos.\par
Este trabalho propõe um estudo de parâmetros fórmula do TSO português para a previsão de reserva necessária ($\rho$), onde, usando os dados históricos horários no período de 2010 a 2019, é calculado o $\rho$ que apresente menor erro na previsão, atingindo erros inferiores a 5\%.\par
O presente trabalho propõe também um modelo \textit{machine learning} para calcular dinamicamente as reservas de potência secundária, utilizando dados operacionais abertos do TSO espanhol. O modelo foi treinado com dados no período de 2014 a 2023, e validado com dados de referência de 2019 a 2022. A metodologia proposta demonstra uma melhoria significativa na utilização das reservas de potência secundária, com um aumento de aproximadamente 47\% na eficiência das reservas ascendentes e cerca de 42\% nas reservas descendentes. Este avanço contribui para uma gestão mais eficiente e equilibrada do sistema elétrico, especialmente em cenários com elevada penetração de vRES.\par




\vspace{0.5cm} %adiciona um espaço de 0.5cm entre o texto e as palavras chave.

\keywordsP{sistemas de reserva, mercados de energia, redes neuronais, previsões}
%reparem que # necessita de um \ para que o latex o interprete correctamente como um caracter especial. Isto também acontece com % por exemplo.