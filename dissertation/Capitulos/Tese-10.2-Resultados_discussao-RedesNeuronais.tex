\subsection{Redes Neuronais \label{se:resml}}

Os vários métodos percorreram muitos tipos de modelos diferentes. Na tabela seguinte apresentamos apenas os melhores resultados baseados em GPD Positivo\par


\begin{table}[H]
    \caption{Resultados métricas Modelos Neuronais}    
    \resizebox{\linewidth}{!}{
\begin{table}[H] 
    \begin{adjustwidth}{-\extralength}{0cm}
    \caption{Metric Results for Up and Down Forecast. \label{res_linear_forecast}}
    \newcolumntype{C}{>{\centering\arraybackslash}X}
    \begin{tabularx}{\fulllength}{CCCCCC}
    \toprule
    &  & RMSE & SAE & AllocM & AllocS  \\
    & Architecture &  & $\times10^{6}$ & $\times10^{5}$ & $\times10^{6}$  \\



    \midrule
            \multirow[m]{5}{*}{Up Allocation}	& StackedFCNN200 & 558.73 & 4.45 & 0.41 & 4.41 \\
                                                & StackedCNN200 & 241.95 & 1.52 & 10.68 & 0.45 \\
                                                & UNET200 & 242.62 & 1.55 & 10.75 & 0.48 \\
                                                & VanillaCNN200 & 233.11 & 1.63 & 6.42 & 0.99 \\
                                                & Transformer & 267.64 & 8.28 & 6.37 & 7.65 \\
           
        \midrule
            \multirow[m]{5}{*}{Down Allocation}	& StackedFCNN200 & 674.12 & 5.64 & 0.14 & 5.62 \\
                                                & StackedCNN200 & 196.73 & 1.20 & 7.24 & 0.47 \\
                                                & UNET200 & 187.44 & 1.11 & 6.78 & 0.43 \\
                                                & VanillaCNN200 & 664.59 & 5.41 & 0.21 & 5.39 \\
                                                & Transformer & 351.15 & 10.69 & 4.59 & 10.23  \\
    \bottomrule
    \end{tabularx}
    \end{adjustwidth}
    % \noindent{\footnotesize{\textsuperscript{1} Tables may have a footer.}}
\end{table}




\begin{table}[H]
    \begin{adjustwidth}{-\extralength}{0cm}
    \caption{Comparative Metric Results for Up and Down Forecast. \label{res_comparative_forecast}}
    \newcolumntype{C}{>{\centering\arraybackslash}X}
    \begin{tabularx}{\fulllength}{CCCCCC}
    \toprule
    &  & PPG & PPG M & PPG S &  PPG Positive \\
    & Architecture & \% & \% & \% & \% \\



    \midrule
            \multirow[m]{5}{*}{Up Allocation}	& StackedFCNN200 & 22.67 & 1.02 & 22.83 & 22.67 \\
                                                & StackedCNN200 & 73.69 & -2500.21 & 92.18 & 0.00 \\
                                                & UNET200 & 73.14 & -2516.72 & 91.74 & 0.00 \\
                                                & VanillaCNN200 & 71.72 & -1463.78 & 82.75 & 0.00 \\
                                                & Transformer &  52.27 & -317.32 & 55.55  & 0.00 \\
           
        \midrule
            \multirow[m]{5}{*}{Down Allocation}	& StackedFCNN200 & 14.08 & 6.01 & 14.10 & 14.08 \\
                                                & StackedCNN200 & 81.81 & -4721.00 & 92.84 & 0.00 \\
                                                & UNET200 & 83.09 & -4413.30 & 93.41 & 0.00 \\
                                                & VanillaCNN200 & 17.48 & -40.50 & 17.61 & 0.00 \\
                                                & Transformer & 5.63 & 2.18 & 5.15  & 5.63 \\
                                                
    \bottomrule
    \end{tabularx}
    \end{adjustwidth}
\end{table}

}
    \label{tab:mlresmetrics}
    \end{table}

O melhor modelo para alocação a Descer apresenta um ganho de desempenho em relação ao \textit{benchmark} de 42\%, e o a Subir de 47\% na soma da janela temporal de validação.\par
Estes modelos têm ambas as alocações e os erros menores que o \textit{benchmark}. Considerando que os dados que permitem quantificar a mais valia económica de reduzir a alocação de reserva secundária em falta devido não são dados públicos, o objetivo passa por manter esta alocações com valores mais baixos que o \textit{benchmark} (GPDF positivo mas próximo de 0) e minimizar a alocação em excesso, maximizando o GPDD, ou juntando as condições maximizando o GPD Positivo. Desta forma a primeira arquitetura de cada tabela é aquela que apresenta melhores resultados quantificáveis quer do ponto de vista operacional como económico.\par
Escolhendo o modelo com melhores resultados em GPD Positivo podemos ver algumas janelas temporais.\par


\begin{figure}[H]
    \centering
    \includegraphics[width=\textwidth]{plots/alocacoes_temporais_upward_prediction_gpd_p.png}
    \caption{Janelas temporais energia a subir}
    \label{fig:mltimewindowsup}
\end{figure}


\begin{figure}[H]
    \centering
    \includegraphics[width=\textwidth]{plots/alocacoes_temporais_downward_prediction_gpd_p.png}
    \caption{Janelas temporais energia a descer}
    \label{fig:mltimewindowsdown}
\end{figure}

É visualmente notável que o modelo mantém uma previsão mais perto da energia usada do que o \textit{benchmark}. Mesmo nas piores janelas temporais, o erro de previsão acumulado é claramente menor que o do método actual.\par
Atente-se no facto de as previsões seguirem bastante mais fielmente as curvas e picos apresentados nos valores de alocação reais, especialmente nas janelas de mês onde temos mais amostras. É possível perceber que o modelo quase sempre acompanha picos da energia usada voltado a baixar quando estes também baixam, destacando-se assim do actual método que mantém uma linha de base bastante mais elevada (desperdiçando mais recursos) e com flutuações que não descrevem tão bem a realidade.\par
Esta flexibilidade no modelo de redes neuronais permite ao operador ter um sinal muito mais flexível diminuindo, deste modo, a alocação desperdiçada.\par

\begin{figure}[H]
    \centering
    \includegraphics[width=\textwidth]{plots/alocation_sum_over_time.png}
    \caption{Soma de Banda Secundária}
    \label{fig:mltimewindowssum}
\end{figure}

Os gráficos anteriores vêm realçar esta mesma ideia. Analisando a energia cumulativa dentro janelas em destaque percebemos que o método proposto mantém quase sempre uma melhoria relativamente ao método utilizado. Esta melhoria é igualmente visível mesmo quando passamos a janelas diárias e semanais, embora haja um aumento considerável das vezes em que o método proposto não é melhor que o actual. E mais importante, o desenho das flutuações é bastante mais fiel ao real.\par


\begin{table}[H]
    \centering
    \caption{Resultados Modelos}    
    \resizebox{0.8\linewidth}{!}{\begin{tabular}{llllll}
\toprule
 &  & média & desvio padrão & min & max \\
\midrule
\multirow[t]{2}{*}{Alocação a Descer (MW)} & benchmark & 884.58 & 165.39 & 720.00 & 1708.00 \\
 & modelo & 800.39 & 250.90 & 234.16 & 1778.46 \\
\cline{1-6}
\multirow[t]{2}{*}{Alocação a Subir (MW)} & benchmark & 882.33 & 165.17 & 719.00 & 1694.00 \\
 & modelo & 793.58 & 164.50 & 243.14 & 1285.83 \\
\cline{1-6}
\multirow[t]{2}{*}{Capacidade Horária (MW)} & benchmark & 1766.92 & 330.54 & 1439.00 & 3399.00 \\
 & modelo & 1593.97 & 340.53 & 787.13 & 2868.49 \\
\cline{1-6}
\multirow[t]{2}{*}{Energia a Descer Extraordinária (MWh)} & benchmark & 203.98 & 261.78 & 1.10 & 1214.00 \\
 & modelo & 175.44 & 278.59 & 6.48 & 1449.46 \\
\cline{1-6}
\multirow[t]{2}{*}{Energia a Subir Extraordinária (MWh)} & benchmark & 166.41 & 183.68 & 1.30 & 1054.80 \\
 & modelo & 146.08 & 179.98 & 1.44 & 1311.24 \\
\cline{1-6}
\bottomrule
\end{tabular}
}
    \label{tab:mlres}
    \end{table}



\begin{table}[H]
    \caption{$\Delta$\% das médias dos Modelos}    
    \resizebox{\linewidth}{!}{\begin{tabular}{rrrrr}
\toprule
Alocação a Descer & Alocação a Subir & Capacidade Horária & Energia a Descer Extraordinária & Energia a Subir Extraordinária \\
\midrule
-9.52 & -10.06 & -9.79 & -13.99 & -12.22 \\
\bottomrule
\end{tabular}
}
    \label{tab:mlres_deltas}
    \end{table}

O método proposto apresenta uma melhoria total, durante o período de validação, de \textasciitilde47\% na alocação a subir e \textasciitilde42\% na alocação a descer face ao método usado no mercado. As melhorias médias são de \textasciitilde37\% e \textasciitilde29\% respectivamente, o que também é uma melhoria face ao estado da arte \cite{Algarvio2024} com 13\% e 8\%.\par
O método proposto liberta em média \textasciitilde33\% dos recursos horários, e baixando a necessidade de activar a reserva terciária em \textasciitilde52\% e \textasciitilde59\%.\par

As correlações entre o modelo e a realidade são também mais elevadas que entre \hyperref[fig:featurecorrelation]{modelo e \textit{benchmark}}.


\begin{figure}[H]
    \centering
    \includegraphics[width=0.8\textwidth]{plots/heatmap_correlation_pred.png}
    \caption{Correlação entre previsão e real}
    \label{fig:predcorrelation}
  \end{figure}

Este mapa de correlações é quase o oposto do apresentado pelo \hyperref[fig:benchmarkcorr]{\textit{benchmark}}.\par
Aqui as correlações maiores são, como seria de esperar, entre a energia usada e a sua alocaçao. Com 48\% na energia a subir e 58\% a descer. E as energias alocadas têm uma correlaçao baixa.\par 