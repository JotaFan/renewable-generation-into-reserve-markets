\subsection{Estatísticos \label{se:resstats}}
Como ponto inicial de resultados os modelos estatísticos apresentam melhorias em relação à alocação em demasia, mas perdas significativas em relação a alocação em falta.\par

\begin{table}[H]
    \caption{Resultados métricas Modelos Estatísticos}    
    \resizebox{\linewidth}{!}{\begin{tabular}{llrrrrrrrrr}
\toprule
 &  & RMSE & SAE & AllocF & AllocD & GPD & GPD F & GPD D & GPD norm & GPD Positivo \\
 & Arquitetura &  &  &  &  &  &  &  &  &  \\
\midrule
\multirow[t]{3}{*}{Alocação a Subir} & ar & 169.21 & 4352584.52 & 2136545.80 & 2216038.73 & 74.92 & -1299.37 & 87.12 & -606.13 & 0.00 \\
 & arma & 181.33 & 4783841.06 & 2187173.52 & 2596667.54 & 72.44 & -1332.53 & 84.91 & -623.81 & 0.00 \\
 & ma & 183.10 & 4940770.16 & 2066116.05 & 2874654.11 & 71.54 & -1253.24 & 83.29 & -584.97 & 0.00 \\
\cline{1-11}
\multirow[t]{3}{*}{Alocação a Descer} & ar & 198.75 & 5265558.19 & 2624914.00 & 2640644.18 & 59.44 & -447.78 & 78.88 & -184.45 & 0.00 \\
 & arma & 218.76 & 5847476.54 & 2876213.76 & 2971262.78 & 54.96 & -500.22 & 76.23 & -211.99 & 0.00 \\
 & ma & 217.53 & 5869239.18 & 2871295.12 & 2997944.06 & 54.79 & -499.20 & 76.02 & -211.59 & 0.00 \\
\cline{1-11}
\bottomrule
\end{tabular}
}
    \label{tab:statsmetrics}
    \end{table}

Estes valores, a nível operacional, podem ser equiparáveis a alocar pouca ou nenhuma energia. Não correndo riscos de alocar em demasia. O que melhora bastante o desempenho em relação ao benchmark a nível de valor de energia absoluta desperdiçada mas derrota o propósito das reservas de energia.\par

\begin{figure}[H]
    \centering
    \includegraphics[width=\textwidth]{plots/alocacoes_temporais_upward_prediction_gpd_stats.png}
    \caption{Janelas temporais de modelos estatísticos energia a subir}
    \label{fig:statstimewindowsup}
\end{figure}


\begin{figure}[H]
    \centering
    \includegraphics[width=\textwidth]{plots/alocacoes_temporais_downward_prediction_gpd_stats.png}
    \caption{Janelas temporais de modelos estatísticos energia a descer}
    \label{fig:statstimewindowsdown}
\end{figure}

Estas figuras mostram que os modelos conseguem até acompanhar o real, podendo até ser um caminho a seguir com algum trabalho específico, mas perdem por manterem-se quase sempre abaixo do necessário, não dando assim a operacionalidade necessária à rede.\par
As médias horárias são:\\
\begin{table}[H]
    \centering
    \resizebox{0.8\linewidth}{!}{\begin{tabular}{llllll}
\toprule
 &  & média & desvio padrão & min & max \\
\midrule
\multirow[t]{2}{*}{Alocação a Descer (MW)} & benchmark & 542.59 & 126.09 & 363.00 & 946.00 \\
 & modelo & 200.14 & 103.62 & 0.00 & 915.37 \\
\cline{1-6}
\multirow[t]{2}{*}{Alocação a Subir (MW)} & benchmark & 623.68 & 152.39 & 419.00 & 958.00 \\
 & modelo & 160.49 & 77.05 & 0.00 & 765.82 \\
\cline{1-6}
\multirow[t]{2}{*}{Capacidade Horária (MW)} & benchmark & 1166.27 & 250.19 & 816.00 & 1891.00 \\
 & modelo & 360.63 & 109.25 & 45.53 & 1039.76 \\
\cline{1-6}
\multirow[t]{2}{*}{Energia a Descer Extraordinária (MWh)} & benchmark & 169.93 & 153.95 & 0.10 & 1226.40 \\
 & modelo & 192.13 & 168.57 & 0.00 & 1481.53 \\
\cline{1-6}
\multirow[t]{2}{*}{Energia a Subir Extraordinária (MWh)} & benchmark & 139.31 & 136.45 & 0.40 & 922.80 \\
 & modelo & 180.43 & 164.43 & 0.01 & 1508.90 \\
\cline{1-6}
\bottomrule
\end{tabular}
}
    \caption{Resultados Modelos Estatísticos}
    \label{tab:statsres}
    \end{table}



\begin{table}[H]
    \centering
    \resizebox{\linewidth}{!}{\begin{tabular}{rrrrr}
\toprule
Alocação a Descer & Alocação a Subir & Capacidade Horária & Energia a Descer Extraordinária & Energia a Subir Extraordinária \\
\midrule
-63.11 & -74.28 & -69.08 & 13.06 & 29.65 \\
\bottomrule
\end{tabular}
}
    \caption{$\Delta$\% das médias dos Modelos Estatísticos}    
    \label{tab:statsres_deltas}
    \end{table}

As médias de alocação são bem mais baixas que o benchmark, mas este modelos têm bastante falta de energia alocada em ambas, logo não respondem à premissa base de ter menos energia em falta e em demasia, inclusive, têm um aumento de necessidade de uso de reserva terciária.\par
