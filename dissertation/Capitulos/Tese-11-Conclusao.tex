\chapter{Conclusões e sugestões futuras}

Em primeiro lugar, pela análise estatística \ref{tab:statsres} e aplicando a ideia \cite{Elsayed}, é possível verificar que, simples modelos estatísticos conseguiriam baixar bastante o erro de previsão, melhor do que o que é utilizado actualmente \ref{tab:benchmarkmetrics}, embora tenham aumentado a necessidade de alocação na reserva terciária.\par
Se considerarmos ainda que nos modelos estatísticos apresentados é apenas utilizada a variável em estudo, e não todos os outros atributos, à semelhança do que sucede nos modelos de redes neuronais, é clara a melhoria ao nível da sua aplicabilidade.\par
Em relação a métodos \textit{machine learning}, com poucos recursos computacionais, conseguimos no presente estudo modelos que superam o método actual.\par
Com modelos relativamente simples conseguimos melhorias muito significativas na alocação de energia, em relação à energia alocada actualmente. Os métodos apresentados no presente estudo são suscetíveis de diminuir significativamente os custos inerentes à alocação de energia desperdiçada, e simultaneamente diminuir a quantidade de recursos em excesso, tendo, naturalmente, um efeito positivo no mercado de reservas.\par
Os resultados aqui apresentados provam que vários tipos de modelos de \textit{machine learning} conseguem realizar previsões bem mais exactas, e que diminuem os recursos usados. Para uso em mercado real, atendendo à sua flexibilidade, estes modelos podem ser adaptados para responder às necessidades do mercado em questão, ao contrário de uma fórmula estática, podem ir evoluindo de acordo com os dados disponibilizados aos longo do tempo.\par
Mostrando que usando estes métodos dinâmicos podemos sim reduzir as incertezas da penetração das \gls{vRES} no mercado de reservas de energia secundária.\par
O futuro de alocação de reservas nos mercados de sistemas poderá passar por este tipo de metodologias. Uma forma de melhorar os resultados alcançados com o presente estudo, seria através do uso de outras variáveis para o modelo, nomeadamente, dados de reserva primária, dados meteorológicos e, principalmente, dados reais e não de \gls{DA}.\par
Um aumento computacional poderia também ter um aumento significativo na qualidade das previsões, através do uso de mais dados e de  modelos mais pesados e complexos, mais dados históricos e modelações com dados de mercados diferentes com convergência para o mercado necessário.\par
Outra possibilidade pode ser o uso de \textit{machine learning} para a reparametrização de novas fórmulas baseadas nas já existentes e em uso.\par
Vários são os caminhos e as formas que podem ainda surgir e que visem a aplicação de modelos de \textit{machine learning} em alocação dinâmica de reservas, mediante os quais os diferentes operadores podem ter arquiteturas e modelos completamente distintos, mas com um fim comum \– o de optimizar recursos
