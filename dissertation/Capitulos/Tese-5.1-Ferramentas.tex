Para a realização do presente estudo, foi necessário o desenvolvimento e utilização de ferramentas específicas, com o objetivo de facilitar a modelação, análise e experimentação. Estas ferramentas, desenvolvidas em \textit{python}, em código aberto, possuem funcionalidades distintas e complementares, permitindo uma abordagem sistemática e eficiente. As principais ferramentas utilizadas neste estudo são: Alquimodelia, Alquitable e MuadDib. \par


\section{\href{https://github.com/alquimodelia/alquimodelia}{Alquimodelia}\label{se:alquimodelia}}

A biblioteca \href{https://github.com/alquimodelia/alquimodelia}{Alquimodelia} foi concebida para automatizar o processo de construção de modelos baseados nas arquiteturas apresentadas neste estudo. A ferramenta permite que o utilizador especifique os parâmetros desejados para o modelo sem a necessidade de reescrever o código para cada arquitetura ou hiperparâmetro. Isso facilita a realização de testes com diferentes arquiteturas, como \gls{FCNN}, \gls{CNN}, \gls{RNN}, \gls{LSTM} e Transformers. \par
\href{https://github.com/alquimodelia/alquimodelia}{Alquimodelia} é composta por um construtor de modelos que permite a criação rápida e eficiente de redes neuronais adaptadas aos objetivos específicos do estudo. Este construtor aceita diferentes entradas, tais como o número de camadas, tipos de funções de ativação e dimensões dos dados. Além disso, a ferramenta está preparada para lidar com arquiteturas mais complexas, como UNETs, ampliando as possibilidades de aplicação em diversos contextos. \par


\section{\href{https://github.com/alquimodelia/alquitable/blob/main/alquitable}{Alquitable}}

A biblioteca \href{https://github.com/alquimodelia/alquitable/blob/main/alquitable}{Alquitable} é uma extensão personalizada do Keras, com camadas, funções de perda, callbacks e geradores de dados. O seu objetivo principal é criar funcionalidades nos modelos em Keras que possam não ser nativas a essa biblioteca. \par

\subsection{\href{https://github.com/alquimodelia/alquitable/blob/main/alquitable/generator.py}{Gerador de dados}}

O gerador construido trata da formatação dos dados para entrada nos modelos. Formatação esse que se baseia nos valores de janelas temporais a usar, e na divisão treino/teste.\par
Esta ferramenta agrega os dados em tensores de formato \textit{(N, t, a)}, onde \textit{N} é o número de casos, \textit{t} é a janela temporal, e \textit{a} é o número de atributos e permite igualmente definir o tempo de salto entre cada entrada.\par
Considere-se como exemplo uma janela temporal de 168 (horas, uma semana) para treino, e 24 (horas) para o alvo. Com um salto temporal de 1 a primeira entrada teria como treino as primeiras 168 horas dos dados, e como alvo as 24 horas consequentes. A segunda entrada seria a partir da segunda hora dos dados, e assim consecutivamente. Para um caso em que o tempo de salto seria 24, a primeira entrada mantinha-se, mas a segunda começaria 24 horas depois, e não apenas uma.\par

Como estamos também a lidar com dados desfasados, o gerador atribui este desfasamento em atributos a especificar. No caso em estudo temos que os atributos são de \gls{DA}, logo estão desfasados 24 horas. O que implica termos de aplicar este desfasamento nos dados que não são \gls{DA}, nomeadamente os dados alvo. Esta propriedade permite também o fácil uso da ferramenta noutros dados desfasados, como as previsões a 3 ou 8 horas.\par

\section{\href{https://github.com/alquimodelia/MuadDib}{MuadDib}\label{se:muaddib}}

% Esta ferramenta criada para desenvolver as experiências desta dissertação, permite ao utilizador apenas com os dados que quer utilizar e a especificação das métricas pretendidas, facilmente ter um modelo optimizado para os seus dados e problema.\par
% Ao usar a ferramenta o utilizador consegue testar vários modelos e hiper parametrizações diferentes, mantendo a vantagem de escrever código ao mínimo.\par

A ferramenta \href{https://github.com/alquimodelia/MuadDib}{MuadDib} foi criada com o intuito de otimizar o processo de experimentação, oferecendo um ambiente integrado para testes e validação de modelos. Com esta biblioteca, é possível testar diferentes combinações de hiperparâmetros e avaliar o desempenho de vários modelos de \text{machine learning}. \par
\href{https://github.com/alquimodelia/MuadDib}{MuadDib} é altamente customizável e oferece suporte para métricas de avaliação, permitindo que o utilizador identifique rapidamente as configurações mais eficientes para o problema em questão. Além disso, a ferramenta é capaz de comparar modelos baseados em redes neuronais com métodos estatísticos, fornecendo uma visão abrangente dos resultados obtidos.\par
A integração com outras ferramentas, como \href{https://github.com/alquimodelia/alquimodelia}{Alquimodelia} e \href{https://github.com/alquimodelia/alquitable/blob/main/alquitable}{Alquitable}, garante um fluxo de trabalho coeso, desde a preparação dos dados até a avaliação final dos modelos. \par


\bigskip
Estas ferramentas não só facilitam a reprodução dos resultados deste estudo, mas também oferecem uma base para o desenvolvimento de projetos futuros em contextos semelhantes. A sua documentação e código aberto estão disponíveis, permitindo a expansão e adaptação conforme as necessidades específicas de cada utilizador. \par
